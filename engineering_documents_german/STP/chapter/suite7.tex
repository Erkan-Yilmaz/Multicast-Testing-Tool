\chapter{Testsuite /TS70/}
Dieser Anhang enthält die Testfälle der Testsuite /TS70/.\\
Die zugehörige Requirement ID lautet /QP30/.\\
\newline
Die Testfälle dieser Testsuite sollen die korrekte und einfache Installation der Applikation sowohl unter Windows als auch unter Linux sicherstellen.\\

\begin{table}[h]
\caption{/TC7001/}
\label{tab:TC7001}
\begin{center}
\begin{tabular}{|p{3.5cm}|p{12cm}|}
\hline
\textbf{Testfall Id} & /TC7001/\\
\hline
\textbf{Testfall Name} & Installation unter Windows\\
\hline
\textbf{Requirement ID} & /QP30/\\
\hline
\textbf{Beschreibung} & Dieser Testfall verifiziert die, für einen erfahrenen Benutzer, einfache Installation der Applikation.\\
\hline
\end{tabular}
\begin{tabular}{|p{2.5cm}|p{5cm}|p{7.55cm}|}
\multicolumn{3}{|c|}{\textbf{Einzelschritte des Testfalls}} \\
\hline
\textbf{Schritt} & \textbf{Aktion} & \textbf{Ergebnis}\\
\hline
Installer herunterladen & Herunterladen des Installers & Die URL ist valide und der Download startet. Der Download muss auch ohne eine Anmeldung auf der Seite funktionieren.\\
\hline
Installation starten & Ausführen des heruntergeladenen Installers. & Ein Dialog öffnet sich. \\
\hline
Konfiguration der Installation & Der Installationswizzard wird verwendet. & Wenn Konfigurationsmöglichkeiten vorhanden sind, so beschränken sie sich nur auf allgemein bekannte Einstellungen, wie das Angeben eines Installationspfads.\\
\hline
Installation & Die Applikation wird installiert. & Der Fortschritt der Installation wird angezeigt. \\
\hline
Starten der Applikation & Die nun installierte Applikation wird über das Start Menü gestartet. & Die Applikation startet. \\
\hline
\end{tabular}
\end{center}
\label{default}
\end{table}

\begin{table}[h]
\caption{/TC7002/}
\label{tab:TC7002}
\begin{center}
\begin{tabular}{|p{3.5cm}|p{12cm}|}
\hline
\textbf{Testfall Id} & /TC7002/\\
\hline
\textbf{Testfall Name} & Installation unter Linux (Ubuntu)\\
\hline
\textbf{Requirement ID} & /QP30/\\
\hline
\textbf{Beschreibung} & Dieser Testfall verifiziert die korrekte Installation unter Linux (Ubuntu).\\
\hline
\end{tabular}
\begin{tabular}{|p{2.5cm}|p{5cm}|p{7.55cm}|}
\multicolumn{3}{|c|}{\textbf{Einzelschritte des Testfalls}} \\
\hline
\textbf{Schritt} & \textbf{Aktion} & \textbf{Ergebnis}\\
\hline
Debian Paket herunterladen & Herunterladen des Debian Pakets. & Die URL ist valide und der Download startet. Der Download muss auch ohne eine Anmeldung auf der Seite funktionieren.\\
\hline
Debian Paket installieren (GUI) & Ein Doppelklick auf das Icon der heruntergeladenen Datei. & Start eines Installationsdialogs. \\
\hline
Debian Paket installieren (CLI) & Ausführen des Befehls sudo dpkg -i [file] & Installation der Applikation.\\
\hline
Installation & Die Applikation wird installiert. & Der Fortschritt der Installation wird angezeigt. \\
\hline
Starten der Applikation & Die nun installierte Applikation wird über das Start Menü gestartet. & Die Applikation startet. \\
\hline
\end{tabular}
\end{center}
\label{default}
\end{table}
