\chapter{Testsuite /TS10/}
Dieser Anhang enthält die Testfälle der Testsuite /TS10/.\\
Die zugehörige Requirement Id lautet /UC10/.\\
\newline
Die Testfälle dieser Testsuite sollen die korrekte Konfigurationsmöglichkeit der Sender über die GUI sicherstellen.\\
Dazu zählt zum Einen die korrekte Übernahme der eingegebenen Werte, die Fehlerbehandlung, sowie die Zusammenarbeit mit dem Rest des Programms im Betrieb.
Zur Konfiguration gehört das Anlegen und Löschen bestehender Sender, das Aktivieren und Deaktivieren von Streams, sowie das Bearbeiten von bereits angelegten Streams.

\begin{table}[h]
\caption{/TC1001/}
\label{tab:TC1001}
\begin{center}
\begin{tabular}{|p{3.5cm}|p{9cm}|}
\hline
\textbf{Testfall Id} & /TC1001/\\
\hline
\textbf{Testfall Name} & Testen auf valide Werte\\
\hline
\textbf{Requirement Id} & /VA0100/, /VA0500/, /VA0700/\\
\hline
\textbf{Beschreibung} & Der Test stellt sicher, dass ein Sender mit validen
Werten über die GUI erstellt werden kann. \\
\hline
\end{tabular}
\begin{tabular}{|p{2.5cm}|p{5cm}|p{4.55cm}|}
\multicolumn{3}{|c|}{\textbf{Einzelschritte des Testfalls}} \\
\hline
\textbf{Schritt} & \textbf{Aktion} & \textbf{Ergebnis}\\
\hline
Programm starten (0) & Das Programm starten & GUI erscheint
 \\
\hline
Dialog öffnen (1) & Betätigen des 'Neu' Buttons im Senderbereich. & Der Dialog
 zum Hinzufügen eines Senders erscheint.
\\
\hline
IP Adressen Test (2) & Eine valide Testgruppen Adresse wird eingestellt, die
 restlichen Einstellungen behalten die Standardwerte bei. Anschließend wird 'OK' betätigt.
 & Ein neuer Sender mit den zuvor eingestellten Werten erscheint im Bereich
 Empfänger.
 \\
\hline
Alle validen Gruppen testen (3) & Wiederholen von 1 und 2 mit allen validen
Testwerten. &
\\
\hline
Dialog öffnen (4) & Betätigen des 'Neu' Buttons im Senderbereich & Der Dialog
 zum Hinzufügen eines Senders erscheint.
\\
\hline
Port Test (5) & Den Port auf einen validen Port setzen und den Dialog mit 'OK'
 beenden. & Ein neuer Sender mit den Eingestellten Werten erscheint im
 Senderbereich.
\\
\hline
Testen aller validen Ports & Wiederhole 4 und 5 mit allen validen
 Konfigurationen. &
\\
\hline
Dialog öffnen (6) & Betätigen des 'Neu' Buttons im Senderbereich & Der Dialog
 zum Hinzufügen eines Senders erscheint.
\\
\hline
\end{tabular}
\end{center}
\end{table}

\begin{table}[h]
\begin{center}
\begin{tabular}{|p{2.5cm}|p{5cm}|p{4.55cm}|}
\hline
Paketraten Test (7) & Die Packetrate auf einen validen Wert setzen und den
 Dialog mit 'OK' beenden. & Ein neuer Sender mit den eingestellten Werten
 erscheint im Senderbereich.
\\
\hline
Testen aller validen Paketraten & Wiederhole 6 und 7 mit allen validen
 Konfigurationen. &
\\
\hline
Dialog öffnen (8) & Betätigen des 'Neu'-Buttons im Senderbereich & Der Dialog
 zum hHinzufügen eines Senders erscheint.
\\
\hline
Paketgrößentest (9) & Die Paketgröße auf einen validen Wert setzen und den
 Dialog mit OK beenden. & Ein neuer Sender mit den eingestellten Werten erscheint im
 Senderbereich.
\\
\hline
 Testen aller validen Paketgrößen & Wiederhole 8 und 9 mit allen validen
 Konfigurationen. &
\\
\hline
Dialog öffnen (10) & Betätigen des 'Neu' Buttons im Senderbereich & Der Dialog
 zum Hinzufügen eines Senders erscheint.
\\
\hline
TTL Test (11) & Die 'Time To Live' auf einen validen Wert setzen und den Dialog
 mit OK beenden. & Ein neuer Sender mit den Eingestellten Werten erscheint im
 Senderbereich.
\\
\hline
Testen aller validen TTLs & Wiederhole 10 und 11 mit allen validen
 Konfigurationen. &
\\
\hline
\end{tabular}
\end{center}
\end{table}

\begin{table}[h]
\caption{/TC1002/}
\label{tab:TC1002}
\begin{center}
\begin{tabular}{|p{3.5cm}|p{9cm}|}
\hline
\textbf{Testfall Id} & /TC1002/\\
\hline
\textbf{Testfall Name} & Testen Aktivierung der Sender.
\\
\hline
\textbf{Requirement Id} & /VA0100/, /VA0500/, /VA0700/\\
\hline
\textbf{Beschreibung} & Der Test stellt sicher, dass angelegte Sender aktiviert
werden können.
\\
\hline
\end{tabular}
\begin{tabular}{|p{2.5cm}|p{5cm}|p{4.55cm}|}
\multicolumn{3}{|c|}{\textbf{Einzelschritte des Testfalls}} \\
\hline
\textbf{Schritt} & \textbf{Aktion} & \textbf{Ergebnis}\\
\hline
TC1001 (1) & Anlegen von Sendern nach TC1001 (1) & Die angelegten Sender
 erscheinen.
\\
\hline
Aktivieren eines Senders (2) & Der erste Sender wird ausgewählt und mittels
 'Aktivieren'-Button gestartet. & Das rote Viereck wird mit einem grünen Dreieck
 ersetzt, was bedeutet, dass der Sender aktiv ist.
\\
\hline
Aktivieren mehrerer Sender gleichzeitig (3) & Die restlichen Sender werden
markiert und über die 'Aktivieren'-Schaltfläche gestartet. & Die Sender werden aktiviert,
was die GUI durch das Ersetzen des roten Vierecks mit einem grünen Dreieck
anzeigt.
\\
\hline
Aktivieren bei Erstellung (4) & Der Erstellen Dialog wird über die 'Neu'
 Schaltfläche geöffnet. Die Standardwerte werden beibehalten und die 'Aktivieren'
 Checkbox gesetzt. Dann wir der Dialog über OK geschlossen. & Ein bereits
 aktivierter Sender erscheint neu in der Senderliste. Dies ist an dem grünen
 Dreieck vor der Zeile des Senders zu erkennen.
\\
\hline
\end{tabular}
\end{center}
\end{table}

\begin{table}[h]
\caption{/TC1003/}
\label{tab:TC1003}
\begin{center}
\begin{tabular}{|p{3.5cm}|p{9cm}|}
\hline
\textbf{Testfall Id} & /TC1003/\\
\hline
\textbf{Testfall Name} & Testen auf Veränderbarkeit der Werte
\\
\hline
\textbf{Requirement Id} & /VA0100/, /VA0500/, /VA0700/\\
\hline
\textbf{Beschreibung} & Der Test stellt sicher, dass die Eigenschaften der
Sender verändert werden können.
\\
\hline
\end{tabular}
\begin{tabular}{|p{2.5cm}|p{5cm}|p{4.55cm}|}
\multicolumn{3}{|c|}{\textbf{Einzelschritte des Testfalls}} \\
\hline
\textbf{Schritt} & \textbf{Aktion} & \textbf{Ergebnis}\\
\hline
Wiederholen von TC1002 Schritte 1-3 & & Eine Liste von Sendern ist
 angelegt. Der erste Sender ist aktiviert, alle anderen sind deaktiviert.
\\
\hline
Öffnen des Bearbeiten-Dialogs (1) & Ein inaktiver Sender wird markiert und die
 'Bearbeiten'-Schaltfläche betätigt. & Der Dialog zum Bearbeiten erscheint.
 \\
\hline
Bearbeiten eines inaktiven Senders (2) & Das Analyseverhalten wird auf lazy 
gesetzt und OK betätigt. & Der Dialog schließt sich.
 \\
\hline
Wiederhole 1 & &.
\\
\hline
Änderung überprüfen (3) &  & Das Analyseverhalten ist 'lazy'.
\\
\hline
Öffnen des Bearbeiten-Dialogs (4) & Der aktive Sender wird markiert und die
 Bearbeiten Schaltfläche betätigt. & Der Dialog zum Bearbeiten erscheint.
\\
\hline
Bearbeiten eines aktiven Senders (5) & Das Analyseverhalten wird auf eager
gesetzt und mit 'OK' betätigt. & Der Dialog schließt sich.
\\
\hline
Wiederhole 4 & &.
\\
\hline
Änderung überprüfen (6) &  & Das Analyseverhalten ist 'eager'.
\\
\hline
\end{tabular}
\end{center}
\end{table}

\begin{table}[h]
\begin{center}
\begin{tabular}{|p{2.5cm}|p{5cm}|p{4.55cm}|}
\hline
Öffnen des Bearbeiten-Dialogs (7) & Ein inaktiver Sender wird markiert und die
 'Bearbeiten'-Schaltfläche betätigt. & Der Dialog zum Bearbeiten erscheint.
\\
\hline
Bearbeiten eines inaktiven Senders (8) & Der Packettype wird auf 'Hirschmann
Packet Format' gesetzt und mit 'OK' betätigt. & Der Dialog schließt sich.
\\
\hline
Wiederhole 7 & &.
\\
\hline
Änderung überprüfen (9) &  & Der Packettype ist 'Hirschmann
Packet Format'.
\\
\hline
\end{tabular}
\end{center}
\end{table}

\begin{table}[h]
\caption{/TC1004/}
\label{tab:TC1004}
\begin{center}
\begin{tabular}{|p{3.5cm}|p{9cm}|}
\hline
\textbf{Testfall Id} & /TC1004/\\
\hline
\textbf{Testfall Name} & Testen auf Deaktivieren von Sendern.
\\
\hline
\textbf{Requirement Id} & /VA0100/, /VA0500/, /VA0700/\\
\hline
\textbf{Beschreibung} & Der Test stellt sicher, dass aktive Sender deaktiviert
werden können.
\\
\hline
\end{tabular}
\begin{tabular}{|p{2.5cm}|p{5cm}|p{4.55cm}|}
\multicolumn{3}{|c|}{\textbf{Einzelschritte des Testfalls}} \\
\hline
\textbf{Schritt} & \textbf{Aktion} & \textbf{Ergebnis}\\
\hline
TC1001 (1) & Anlegen und Aktivieren von Sendern nach TC1002 (1) & Eine Liste
 mit aktiven Sendern liegt vor.
\\
\hline
Deaktivieren eines Senders & Der erste Sender wird ausgewählt und mittels
 'Deaktivieren'-Button gestoppt. & Das grüne Dreieck wird mit einem roten Viereck
 ersetzt, was bedeutet, dass der Sender inaktiv ist.
 \\
\hline
Deaktivieren mehrerer Sender gleichzeitig (3) & Die restlichen Sender werden
markiert und über die Deaktivieren Schaltfläche gestoppt. & Die Sender werden
deaktiviert, was die GUI durch das Ersetzen des grünen Dreiecks mit einem roten
Viereck anzeigt.
 \\
\hline
\end{tabular}
\end{center}
\end{table}

\begin{table}[h]
\caption{/TC1005/}
\label{tab:TC1005}
\begin{center}
\begin{tabular}{|p{3.5cm}|p{9cm}|}
\hline
\textbf{Testfall Id} & /TC1005/\\
\hline
\textbf{Testfall Name} & Testen auf Entfernen von Sendern
\\
\hline
\textbf{Requirement Id} & /VA0100/, /VA0500/, /VA0700/\\
\hline
\textbf{Beschreibung} & Der Test stellt sicher, dass Sender aus der Senderliste
entfernt werden können.
\\
\hline
\end{tabular}
\begin{tabular}{|p{2.5cm}|p{5cm}|p{4.55cm}|}
\multicolumn{3}{|c|}{\textbf{Einzelschritte des Testfalls}} \\
\hline
\textbf{Schritt} & \textbf{Aktion} & \textbf{Ergebnis}\\
\hline
TC1001 (1) & Anlegen von Sendern nach TC1001 (1) & Eine Liste
 von Sendern liegt vor.
\\
\hline
Entfernen eines Senders & Der erste Sender wird ausgewählt und mittels
 'Löschen'-Button entfernt. & Der entsprechende Sender ist nicht mehr in der Liste
 vorhanden.
\\
\hline
Entfernen mehrerer Sender gleichzeitig (3) & Die restlichen Sender werden
markiert und über die 'Löschen'-Schaltfläche gestoppt. & Die Sender
werden entfernt und die Senderliste ist leer.
\\
\hline
\end{tabular}
\end{center}
\end{table}

\begin{table}[h]
\caption{/TC1006/}
\label{tab:TC1006}
\begin{center}
\begin{tabular}{|p{3.5cm}|p{9cm}|}
\hline
\textbf{Testfall Id} & /TC1006/\\
\hline
\textbf{Testfall Name} & Testen auf invalide Eingaben
\\
\hline
\textbf{Requirement Id} & /VA0100/, /VA0500/, /VA0700/\\
\hline
\textbf{Beschreibung} & Der Test stellt sicher, dass das Programm invalide
Eingaben erkennt und abfängt.\\
\hline
\end{tabular}
\begin{tabular}{|p{2.5cm}|p{5cm}|p{4.55cm}|}
\multicolumn{3}{|c|}{\textbf{Einzelschritte des Testfalls}} \\
\hline
\textbf{Schritt} & \textbf{Aktion} & \textbf{Ergebnis}\\
\hline
Programm starten (0) & Das Programm starten & GUI erscheint
\\
\hline
Dialog öffnen (1) & Betätigen des 'Neu'-Buttons im Senderbereich & Der Dialog
 zum Hinzufügen eines Senders erscheint.
\\
\hline
IP Adressen Test (2) & Eine invalide Testgruppen Adresse wird eingestellt, die
 restlichen Einstellungen behalten die Standardwerte bei. Anschließend wird 'OK' betätigt.
 & Es erscheint zusätzlich ein Dialog, welcher auf eine Fehleingabe hinweist.
 Der 'Erstellen'-Dialog bleibt offen.
\\
\hline
Alle invaliden IP-Adressen testen (3) & Wiederholen von 1 und 2 mitallen unter
'Definitionen' definierten invaliden Werten. &
\\
\hline
Dialog öffnen (4) & Betätigen des 'Neu'-Buttons im Senderbereich & Der Dialog
 zum Hinzufügen eines Senders erscheint.
\\
\hline
\end{tabular}
\end{center}
\end{table}

\begin{table}[h]
\begin{center}
\begin{tabular}{|p{2.5cm}|p{5cm}|p{4.55cm}|}
\hline
Port Test (5) & Den Port auf einen invaliden Port setzen und 'OK'
 betätigen. & Es erscheint zusätzlich ein Dialog, welcher auf eine Fehleingabe
 hinweist. Der Erstellen Dialog bleibt offen.
\\
\hline
Alle invaliden Ports testen (3) & Wiederholen
von 1 und 2 mitallen unter 'Definitionen' definierten invaliden Werten. &
\\
\hline
Dialog öffnen (6) & Betätigen des 'Neu' Buttons im Senderbereich & Der Dialog
 zum Hinzufügen eines Senders erscheint.
\\
\hline
Paketraten Test (7) & Die Packetrate auf einen invaliden Wert setzen und OK
 betätigen. & Es erscheint zusätzlich ein Dialog, welcher auf eine Fehleingabe
 hinweist. Der Erstellen Dialog bleibt offen.
\\
\hline
Alle invaliden Paketraten testen (3) & Wiederholen von 1 und 2 mitallen unter
'Definitionen' definierten invaliden Werten. &
\\
\hline
Dialog öffnen (8) & Betätigen des 'Neu' Buttons im Senderbereich & Der Dialog
 zum Hinzufügen eines Senders erscheint.
\\
\hline
Paketgrößentest (9) & Die Paketgröße auf einen invaliden Wert setzen und OK
 betätigen. &  Es erscheint zusätzlich ein Dialog, welcher auf eine Fehleingabe
 hinweist. Der Erstellen Dialog bleibt offen.
\\
\hline
Alle invaliden Paketgrößen testen (3) & Wiederholen von 1 und 2 mitallen unter
'Definitionen' definierten invaliden Werten. &
\\
\hline
Dialog öffnen (10) & Betätigen des 'Neu' Buttons im Senderbereich & Der Dialog
 zum Hinzufügen eines Senders erscheint.
\\
\hline
\end{tabular}
\end{center}
\end{table}

\begin{table}[h]
\begin{center}
\begin{tabular}{|p{2.5cm}|p{5cm}|p{4.55cm}|}
\hline
TTL Test (11) & Die 'Time To Live' auf einen invaliden Wert setzen und 'OK'
 betätigen. & Es erscheint zusätzlich ein Dialog, welcher auf eine Fehleingabe
 hinweist. Der 'Erstellen'-Dialog bleibt offen.
\\
\hline
Alle invaliden TTLs testen (3) & Wiederholen von 1 und 2 mitallen unter
'Definitionen' definierten invaliden Werten.
\\
\hline
\end{tabular}
\end{center}
\end{table}
