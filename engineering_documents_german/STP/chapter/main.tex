
\chapter{Scope}
Der Systemtestplan spezifiziert die Teststrategie und den Testumfang zur
Verifikation des Pflichtenheftes (SRS). Testfälle werden referenziert.
Er bildet die Basis für den Systemtestbericht (STR) der zusätzlich noch die
Testergebnisse auflistet.
\newline
Damit setzt er die Qualitätsanforderung /QW20/ (/LQW20/) um.
Danach muss jedes einzelne Module testbar, sowie Fehlerzustände
analysierbar sein. Ein Testplan, sowie die Testergebnisse sind schriftlich vorzulegen.

\chapter{Definitionen}
\section{Wertebereiche}
Die folgenden Wertebereiche werden in diesem Dokument benutzt:\\

\subsection{IP-Addressen}
$ V4 := \{ p_0.p_1.p_2.p_3 | p_i \in [0|_{10},255|_{10}] , i=0,\ldots,3 \} $\\
$ V6 := \{ p_0:p_1:p_2:p_3:p_4:p_5:p_6:p_7 | p_i \in [0|_{16},ffff|_{16}] ,
i=0,\ldots,7 \} $\\
$ V4MC := \{ p_0.p_1.p_2.p_3 | p_0 \in [224|_{10},239|_{10}] \wedge p_{1,2} \in
[0|_{10},255|_{10}] \wedge p_3 \in [1|_{10},254|_{10}] \} $
$ V6MC := \{ p_0:p_1:p_2:p_3:p_4:p_5:p_6:p_7 | p_0 \in [ff00|_{16},ffff|_{16}]
\wedge p_{1,\ldots,6} \in [0|_{16},ffff|_{16}] \wedge p_7 \in
[1|_{16},fffe|_{16}] \} $\\

\subsection{Ports}
$ PORTS := [0,65535] $

\subsection{Analyze-Granularitäten}
$ABEHAVIOURS := \{ 'lazy', 'default', 'eager' \} $

\subsection{Sender-Spezifisch}
$PACKETSIZES := [0,9000]$\\
$PACKETRATES := [1,1000]$\\
$PACKETRATES_{reachable} := \{ x \in PACKETRATES | \frac{1000}{x} \in \mathbb{N}
\} $\\
$PACKETTYPES := \{ 'spam', 'hmann' \}$
$TTLS := [1,255]$

\section{Testwerte}
\subsection{IP Multicast Groups}

\begin{table}[H]
\center
\begin{tabular}{l | l | l}
\textbf{Wert} & \textbf{Bezeichnung} & \textbf{Klasse}\\
\hline \hline
224.0.0.1 & V4 Infimum & $\in IPV4$\\
239.255.255.254 & V4 Supremum & $\in IPV4$\\
225.1.1.1 & Standard-GUI-Wert & $\in IPV4$\\
ff00:::::::1 & V6 Infimum & $\in IPV6$\\
ffff:ffff:ffff:ffff:ffff:ffff:ffff:fffe & V6 Supremum & $\in IPV6$\\
\end{tabular}
\caption{Valide IP Multicast-Group Testwerte}
\end{table}

\begin{table}[H]
\center
\begin{tabular}{l | l | l}
\textbf{Wert} & \textbf{Bezeichnung} & \textbf{Klasse} \\
\hline \hline
223.255.255.254 & Keine Multicast Addresse & $\in \frac{V4}{V4MC} $\\
224.0.0.0 & Supremum des ersten invaliden IP-Bereichs & $\in \frac{V4}{V4MC}$\\
239.255.255.255 & Infimum des zweiten invaliden IP Bereichs & $\in
\frac{V4}{V4MC}$\\
240.0.0.1 & Keine Multicast Address & $\in \frac{V4}{V4MC}$\\
feff:::::::1 & Keine Multicast Addresse & $\in \frac{V6}{V6MC} $\\
ff00:::::::0 & Supremum des ersten invaliden IP-Bereichs & $\in
\frac{V6}{V6MC}$\\
ffff:ffff:ffff:ffff:ffff:ffff:ffff:ffff & Infimum des zweiten invaliden IP
Bereichs & $\in \frac{V6}{V6MC}$\\
:::::: & Keine IP Addresse & $\ni V4 \cup V6$\\
0.0.0 & Keine IP Addresse & $\ni V4 \cup V6$\\
'totallywrong' & Keine IP Addresse & $\ni V4 \cup V6$
\end{tabular}
\caption{Invalide IP Multicast-Group Testwerte}
\end{table}

\subsection{Ports}

\begin{table}[H]
\center
\begin{tabular}{l | l | l}
\textbf{Wert} & \textbf{Bezeichnung} & \textbf{Klasse}\\
\hline \hline
0 & Infimum & $\in PORTS$\\
65535 & Supremum & $\in PORTS$\\
1025 & User-Mode Infimum & $\in PORTS$
\end{tabular}
\caption{Valide Port Testwerte}
\end{table}

\begin{table}[H]
\center
\begin{tabular}{l | l | l}
\textbf{Wert} & \textbf{Bezeichnung} & \textbf{Klasse} \\
\hline \hline
-1 & Kein Port & $\ni PORTS $\\
65536 & Kein Port & $\ni PORTS $\\
'einszwei' & Kein Port & $\ni PORTS $
\end{tabular}
\caption{Invalide Port Testwerte}
\end{table}

\section{CLI Befehle}

\begin{table}[H]
\center
\begin{tabular}{p{5cm} | p{10cm}}
\textbf{Befehl} & \textbf{Beschreibung}\\
\hline \hline
-nogui & Startet das Tool ohne GUI\\
-cli & Startet den Logger mit\\
-profile <PROFILNAME> & Läd ein oder mehrere Profile anhand des
Profilnamens (getrennt durch Leerzeichen)\\
-path <PATH> & Läd ein oder mehrere Profile anhand eines Pfades (getrennt durch
Leerzeichen)\\
-startall & Startet alle Sender/Empfänger der geladenen Konfigurationen\\
-startnone & Startet keine der Sender/Empfänger der geladenen Konfigurationen\\
-restore & Startet alle Sender/Empfänger in dem Status (aktiv/inaktiv), wie im
Profil gespeichert (default)\\
\end{tabular}
\caption{Startparamter des Programms}
\end{table}

\chapter{Produktname und weitere Eigenschaften}
Dieser System Test Plan bezieht sich einzig und allein auf das Programm "`Multicast Test Tool"'
, welches von der Firma SPAM Software Programming And More für die Firma Net-Tools entwickelt wird.

\chapter{Zu testende Funktionen, Qualitätsanforderungen und Features}
Die folgenden, tabellarisch aufgelisteten Funktionen, Qualitätsanforderungen und
Features müssen getestet werden, soweit sie nicht als "`nicht zu testen"' markiert sind.

\begin{table}[h]
\caption{Funktionen, Qualitätsanforderungen und Features}
\label{tab:features}
\begin{center}
\begin{tabular}{|c|p{7cm}|c|c|}
\hline
\textbf{Requirement ID} & \textbf{Funktionalität} & \textbf{Priorität} & \textbf{Testsuite Id}\\
\hline
/UC10/ & Konf. des MC Senders & A & /TS10/\\
\hline
/UC10/ & Konf. des MC Empfängers & A & /TS20/\\
\hline
/UC10/ & Laden und Speichern der Konfiguration & B & /TS30/\\
\hline
/UC20/ & Laden einer Konfiguration und Starten von Streams über CLI & B & /TS40/\\
\hline
/UC30/ & Fehlerfreier Austausch von MC Paketen & A & /TS50/\\
\hline
/UC40/ & Analyse und Anzeige von Messdaten & A & /TS60/\\
\hline
/QP30/ & Installation der Applikation & B & /TS70/\\
\hline
\end{tabular}
\end{center}
\label{default}
\end{table}

\chapter{Test-Vorbereitungs-Strategie}
Um Testsuiten und damit verbundene Testfälle zu erstellen werden die Use-Cases aus der SRS aufgegriffen.\\
Zu jedem Use-Case wird mindestens eine Testsuite erstellt. Je nach Umfang des Use-Cases können auch mehrere Testsuiten entstehen.\\
Bei der anschließenden Erstellung der Testfälle wird versucht zu dem jeweiligen Use-Cases zugehörige Funktionale Anforderungen zu finden.\\
Diese werden bei der Erstellung der Testfälle berücksichtigt. Dies ermöglicht ein tiefgreifendes Testen der Use-Cases.\\
\newline
Die Testsuiten werden zunächst erstellt, anschließend überprüft und zuletzt getestet.\\
Zum Testen muss zunächst alles Testequipment angeschafft werden. Weitere Vorbereitungen sind nicht nötig.\\
Es wird davon ausgegangen, dass alle Test-Ersteller und Tester mit dem Produkt und deren Funktionen vertraut sind.

\chapter{Test-Durchführungs-Strategie}
Bei der Durchführung der Tests muss nicht viel berücksichtigt werden.
Da es in etwa gleich viele Tester wie Testsuiten gibt übernimmt jeder Tester genau eine Testsuite. Testsuite 7 ist recht klein und wird zusätzlich vom Testmanager übernommen.
Jede Testsuite besteht im Schnitt aus 5-6 Testfällen, was keine Priorisierung nötig macht.\\
Die Testfälle werden von dem jeweiligen Tester in beliebiger Reihenfolge ausgeführt. Schlägt ein Testfall fehl, so wird das Problem sofort
an den zuständigen Entwickler gemeldet und von diesem behoben. Nach Behebung des Fehlers wird erneut der Testfall getestet.
Sobald dieser Testfall Positiv ist, werden erneut alle bisher getesteten Testfälle innerhalb der Testsuite getestet, um
eine eventuell aufgedeckte "`Fehlermaskierung"' zu erkennen. Ist die Testsuite erfolgreich getestet wird der Erfolg an den Test-Manager, RS,
gemeldet. Sind alle Testsuiten als positiv gemeldet, werden erneut alle Testsuiten in einem Zug getestet. Treten Fehler auf,
so werden diese behoben und so lange der Schritt "`Korrigieren-Alle Testen"' wiederholt, bis keine Fehler mehr auftreten.
Alle Testsuiten sind nun fehlerfrei. Die Testfortschritte bzw. Testergebnisse werden in dem STR festgehalten.

\chapter{Test-Equipment}
Als Test Equipment werden 2 Rechner mit sowohl Linux als auch Windows Betriebsystem benötigt. Hardwareanforderungen finden sich in der SRS bzw. im PP.
Für Linux wird Ubuntu 10.10 x86 bzw. 11.04 x86 gewählt. Als Windows Vertreter wird Windows XP und Windows 7 in der 32 Bit Variante verwendet.\\
Zum Testen der Netzwerk-Infrastruktur werden vom Auftraggeber entsprechende Multicast-fähige Router/Switches und Kabel bereit gestellt.\\
Als Testsoftware wird zusätzlich nur das Programm "`Wireshark"' benötigt. Weiteres Equipment ist nicht vorgesehen.

\chapter{Test-Zeitplanung und -Budget}
Das Budget für die Testphase wurde bereits im Projektplan mit eingerechnet und ist daher an dieser Stelle nicht von Relevanz.\\
Für die Testplan-Erstellung wird eine Woche veranschlagt.\\ Der Review der Testpläne wird ebenfalls auf eine Dauer von einer Woche veranschlagt.\\
Für das Testen wird ebenfalls eine Woche eingeplant. Die Arbeit der Ersteller
erfolgt hierbei parallel zu ihrer normalen Tätigkeit.\\ An einem abschließenden
Testtag (14.05.2011) werden alle Testsuiten erneut getestet, bis keine Testsuite mehr fehlschlägt.\\

\chapter{Test-Planung}
\begin{table}[h]
\caption{Funktionen, Qualitätsanforderungen und Features}
\label{tab:features}
\begin{center}
\begin{tabular}{|c|p{6cm}|c|c|c|}
\hline
\textbf{Testsuite Id} & \textbf{Test Ziel} & \textbf{Ersteller} & \textbf{Reviewer} & \textbf{Tester}\\
\hline
/TS10/ & Senderkonfiguration sicherstellen & RS & DH & JJ\\
\hline
/TS20/ & Empfängerkonfiguration sicherstellen & JJ & RS & DH\\
\hline
/TS30/ & Konfigurationsverwaltung sicherstellen & KW & JJ & TSC\\
\hline
/TS40/ & Automatisierung per CLI sicherstellen & TSC & KW & TST\\
\hline
/TS50/ & Multicast-Fähigkeit sicherstellen & TST & TSC & KW\\
\hline
/TS60/ & Auswertung der Daten sicherstellen & DH & TST & RS\\
\hline
/TS70/ & Installation der Applikation & KW & DH & RS\\
\hline
\end{tabular}
\end{center}
\label{default}
\end{table}
\chapter{Referenzen, Standards}
PP - Projektplan\\
SRS - System Requirement Specification

\begin{appendix}

\chapter{Testsuite /TS10/}
Dieser Anhang enthält die Testfälle der Testsuite /TS10/.\\
Die zugehörige Requirement Id lautet /UC10/.\\
\newline
Die Testfälle dieser Testsuite sollen die korrekte Konfigurationsmöglichkeit der Sender über die GUI sicherstellen.\\
Dazu zählt zum Einen die korrekte Übernahme der eingegebenen Werte, die Fehlerbehandlung, sowie die Zusammenarbeit mit dem Rest des Programms im Betrieb.
Zur Konfiguration gehört das Anlegen und Löschen bestehender Sender, das Aktivieren und Deaktivieren von Streams, sowie das Bearbeiten von bereits angelegten Streams.

\begin{table}[h]
\caption{/TC1001/}
\label{tab:TC1001}
\begin{center}
\begin{tabular}{|p{3.5cm}|p{9cm}|}
\hline
\textbf{Testfall Id} & /TC1001/\\
\hline
\textbf{Testfall Name} & Testen auf valide Werte\\
\hline
\textbf{Requirement Id} & /VA0100/, /VA0500/, /VA0700/\\
\hline
\textbf{Beschreibung} & Der Test stellt sicher, dass ein Sender mit validen
Werten über die GUI erstellt werden kann. \\
\hline
\end{tabular}
\begin{tabular}{|p{2.5cm}|p{5cm}|p{4.55cm}|}
\multicolumn{3}{|c|}{\textbf{Einzelschritte des Testfalls}} \\
\hline
\textbf{Schritt} & \textbf{Aktion} & \textbf{Ergebnis}\\
\hline
Programm starten (0) & Das Programm starten & GUI erscheint
 \\
\hline
Dialog öffnen (1) & Betätigen des 'Neu' Buttons im Senderbereich. & Der Dialog
 zum Hinzufügen eines Senders erscheint.
\\
\hline
IP Adressen Test (2) & Eine valide Testgruppen Adresse wird eingestellt, die
 restlichen Einstellungen behalten die Standardwerte bei. Anschließend wird 'OK' betätigt.
 & Ein neuer Sender mit den zuvor eingestellten Werten erscheint im Bereich
 Empfänger.
 \\
\hline
Alle validen Gruppen testen (3) & Wiederholen von 1 und 2 mit allen validen
Testwerten. &
\\
\hline
Dialog öffnen (4) & Betätigen des 'Neu' Buttons im Senderbereich & Der Dialog
 zum Hinzufügen eines Senders erscheint.
\\
\hline
Port Test (5) & Den Port auf einen validen Port setzen und den Dialog mit 'OK'
 beenden. & Ein neuer Sender mit den Eingestellten Werten erscheint im
 Senderbereich.
\\
\hline
Testen aller validen Ports & Wiederhole 4 und 5 mit allen validen
 Konfigurationen. &
\\
\hline
Dialog öffnen (6) & Betätigen des 'Neu' Buttons im Senderbereich & Der Dialog
 zum Hinzufügen eines Senders erscheint.
\\
\hline
\end{tabular}
\end{center}
\end{table}

\begin{table}[h]
\begin{center}
\begin{tabular}{|p{2.5cm}|p{5cm}|p{4.55cm}|}
\hline
Paketraten Test (7) & Die Packetrate auf einen validen Wert setzen und den
 Dialog mit 'OK' beenden. & Ein neuer Sender mit den eingestellten Werten
 erscheint im Senderbereich.
\\
\hline
Testen aller validen Paketraten & Wiederhole 6 und 7 mit allen validen
 Konfigurationen. &
\\
\hline
Dialog öffnen (8) & Betätigen des 'Neu'-Buttons im Senderbereich & Der Dialog
 zum hHinzufügen eines Senders erscheint.
\\
\hline
Paketgrößentest (9) & Die Paketgröße auf einen validen Wert setzen und den
 Dialog mit OK beenden. & Ein neuer Sender mit den eingestellten Werten erscheint im
 Senderbereich.
\\
\hline
 Testen aller validen Paketgrößen & Wiederhole 8 und 9 mit allen validen
 Konfigurationen. &
\\
\hline
Dialog öffnen (10) & Betätigen des 'Neu' Buttons im Senderbereich & Der Dialog
 zum Hinzufügen eines Senders erscheint.
\\
\hline
TTL Test (11) & Die 'Time To Live' auf einen validen Wert setzen und den Dialog
 mit OK beenden. & Ein neuer Sender mit den Eingestellten Werten erscheint im
 Senderbereich.
\\
\hline
Testen aller validen TTLs & Wiederhole 10 und 11 mit allen validen
 Konfigurationen. &
\\
\hline
\end{tabular}
\end{center}
\end{table}

\begin{table}[h]
\caption{/TC1002/}
\label{tab:TC1002}
\begin{center}
\begin{tabular}{|p{3.5cm}|p{9cm}|}
\hline
\textbf{Testfall Id} & /TC1002/\\
\hline
\textbf{Testfall Name} & Testen Aktivierung der Sender.
\\
\hline
\textbf{Requirement Id} & /VA0100/, /VA0500/, /VA0700/\\
\hline
\textbf{Beschreibung} & Der Test stellt sicher, dass angelegte Sender aktiviert
werden können.
\\
\hline
\end{tabular}
\begin{tabular}{|p{2.5cm}|p{5cm}|p{4.55cm}|}
\multicolumn{3}{|c|}{\textbf{Einzelschritte des Testfalls}} \\
\hline
\textbf{Schritt} & \textbf{Aktion} & \textbf{Ergebnis}\\
\hline
TC1001 (1) & Anlegen von Sendern nach TC1001 (1) & Die angelegten Sender
 erscheinen.
\\
\hline
Aktivieren eines Senders (2) & Der erste Sender wird ausgewählt und mittels
 'Aktivieren'-Button gestartet. & Das rote Viereck wird mit einem grünen Dreieck
 ersetzt, was bedeutet, dass der Sender aktiv ist.
\\
\hline
Aktivieren mehrerer Sender gleichzeitig (3) & Die restlichen Sender werden
markiert und über die 'Aktivieren'-Schaltfläche gestartet. & Die Sender werden aktiviert,
was die GUI durch das Ersetzen des roten Vierecks mit einem grünen Dreieck
anzeigt.
\\
\hline
Aktivieren bei Erstellung (4) & Der Erstellen Dialog wird über die 'Neu'
 Schaltfläche geöffnet. Die Standardwerte werden beibehalten und die 'Aktivieren'
 Checkbox gesetzt. Dann wir der Dialog über OK geschlossen. & Ein bereits
 aktivierter Sender erscheint neu in der Senderliste. Dies ist an dem grünen
 Dreieck vor der Zeile des Senders zu erkennen.
\\
\hline
\end{tabular}
\end{center}
\end{table}

\begin{table}[h]
\caption{/TC1003/}
\label{tab:TC1003}
\begin{center}
\begin{tabular}{|p{3.5cm}|p{9cm}|}
\hline
\textbf{Testfall Id} & /TC1003/\\
\hline
\textbf{Testfall Name} & Testen auf Veränderbarkeit der Werte
\\
\hline
\textbf{Requirement Id} & /VA0100/, /VA0500/, /VA0700/\\
\hline
\textbf{Beschreibung} & Der Test stellt sicher, dass die Eigenschaften der
Sender verändert werden können.
\\
\hline
\end{tabular}
\begin{tabular}{|p{2.5cm}|p{5cm}|p{4.55cm}|}
\multicolumn{3}{|c|}{\textbf{Einzelschritte des Testfalls}} \\
\hline
\textbf{Schritt} & \textbf{Aktion} & \textbf{Ergebnis}\\
\hline
Wiederholen von TC1002 Schritte 1-3 & & Eine Liste von Sendern ist
 angelegt. Der erste Sender ist aktiviert, alle anderen sind deaktiviert.
\\
\hline
Öffnen des Bearbeiten-Dialogs (1) & Ein inaktiver Sender wird markiert und die
 'Bearbeiten'-Schaltfläche betätigt. & Der Dialog zum Bearbeiten erscheint.
 \\
\hline
Bearbeiten eines inaktiven Senders (2) & Das Analyseverhalten wird auf lazy 
gesetzt und OK betätigt. & Der Dialog schließt sich.
 \\
\hline
Wiederhole 1 & &.
\\
\hline
Änderung überprüfen (3) &  & Das Analyseverhalten ist 'lazy'.
\\
\hline
Öffnen des Bearbeiten-Dialogs (4) & Der aktive Sender wird markiert und die
 Bearbeiten Schaltfläche betätigt. & Der Dialog zum Bearbeiten erscheint.
\\
\hline
Bearbeiten eines aktiven Senders (5) & Das Analyseverhalten wird auf eager
gesetzt und mit 'OK' betätigt. & Der Dialog schließt sich.
\\
\hline
Wiederhole 4 & &.
\\
\hline
Änderung überprüfen (6) &  & Das Analyseverhalten ist 'eager'.
\\
\hline
\end{tabular}
\end{center}
\end{table}

\begin{table}[h]
\begin{center}
\begin{tabular}{|p{2.5cm}|p{5cm}|p{4.55cm}|}
\hline
Öffnen des Bearbeiten-Dialogs (7) & Ein inaktiver Sender wird markiert und die
 'Bearbeiten'-Schaltfläche betätigt. & Der Dialog zum Bearbeiten erscheint.
\\
\hline
Bearbeiten eines inaktiven Senders (8) & Der Packettype wird auf 'Hirschmann
Packet Format' gesetzt und mit 'OK' betätigt. & Der Dialog schließt sich.
\\
\hline
Wiederhole 7 & &.
\\
\hline
Änderung überprüfen (9) &  & Der Packettype ist 'Hirschmann
Packet Format'.
\\
\hline
\end{tabular}
\end{center}
\end{table}

\begin{table}[h]
\caption{/TC1004/}
\label{tab:TC1004}
\begin{center}
\begin{tabular}{|p{3.5cm}|p{9cm}|}
\hline
\textbf{Testfall Id} & /TC1004/\\
\hline
\textbf{Testfall Name} & Testen auf Deaktivieren von Sendern.
\\
\hline
\textbf{Requirement Id} & /VA0100/, /VA0500/, /VA0700/\\
\hline
\textbf{Beschreibung} & Der Test stellt sicher, dass aktive Sender deaktiviert
werden können.
\\
\hline
\end{tabular}
\begin{tabular}{|p{2.5cm}|p{5cm}|p{4.55cm}|}
\multicolumn{3}{|c|}{\textbf{Einzelschritte des Testfalls}} \\
\hline
\textbf{Schritt} & \textbf{Aktion} & \textbf{Ergebnis}\\
\hline
TC1001 (1) & Anlegen und Aktivieren von Sendern nach TC1002 (1) & Eine Liste
 mit aktiven Sendern liegt vor.
\\
\hline
Deaktivieren eines Senders & Der erste Sender wird ausgewählt und mittels
 'Deaktivieren'-Button gestoppt. & Das grüne Dreieck wird mit einem roten Viereck
 ersetzt, was bedeutet, dass der Sender inaktiv ist.
 \\
\hline
Deaktivieren mehrerer Sender gleichzeitig (3) & Die restlichen Sender werden
markiert und über die Deaktivieren Schaltfläche gestoppt. & Die Sender werden
deaktiviert, was die GUI durch das Ersetzen des grünen Dreiecks mit einem roten
Viereck anzeigt.
 \\
\hline
\end{tabular}
\end{center}
\end{table}

\begin{table}[h]
\caption{/TC1005/}
\label{tab:TC1005}
\begin{center}
\begin{tabular}{|p{3.5cm}|p{9cm}|}
\hline
\textbf{Testfall Id} & /TC1005/\\
\hline
\textbf{Testfall Name} & Testen auf Entfernen von Sendern
\\
\hline
\textbf{Requirement Id} & /VA0100/, /VA0500/, /VA0700/\\
\hline
\textbf{Beschreibung} & Der Test stellt sicher, dass Sender aus der Senderliste
entfernt werden können.
\\
\hline
\end{tabular}
\begin{tabular}{|p{2.5cm}|p{5cm}|p{4.55cm}|}
\multicolumn{3}{|c|}{\textbf{Einzelschritte des Testfalls}} \\
\hline
\textbf{Schritt} & \textbf{Aktion} & \textbf{Ergebnis}\\
\hline
TC1001 (1) & Anlegen von Sendern nach TC1001 (1) & Eine Liste
 von Sendern liegt vor.
\\
\hline
Entfernen eines Senders & Der erste Sender wird ausgewählt und mittels
 'Löschen'-Button entfernt. & Der entsprechende Sender ist nicht mehr in der Liste
 vorhanden.
\\
\hline
Entfernen mehrerer Sender gleichzeitig (3) & Die restlichen Sender werden
markiert und über die 'Löschen'-Schaltfläche gestoppt. & Die Sender
werden entfernt und die Senderliste ist leer.
\\
\hline
\end{tabular}
\end{center}
\end{table}

\begin{table}[h]
\caption{/TC1006/}
\label{tab:TC1006}
\begin{center}
\begin{tabular}{|p{3.5cm}|p{9cm}|}
\hline
\textbf{Testfall Id} & /TC1006/\\
\hline
\textbf{Testfall Name} & Testen auf invalide Eingaben
\\
\hline
\textbf{Requirement Id} & /VA0100/, /VA0500/, /VA0700/\\
\hline
\textbf{Beschreibung} & Der Test stellt sicher, dass das Programm invalide
Eingaben erkennt und abfängt.\\
\hline
\end{tabular}
\begin{tabular}{|p{2.5cm}|p{5cm}|p{4.55cm}|}
\multicolumn{3}{|c|}{\textbf{Einzelschritte des Testfalls}} \\
\hline
\textbf{Schritt} & \textbf{Aktion} & \textbf{Ergebnis}\\
\hline
Programm starten (0) & Das Programm starten & GUI erscheint
\\
\hline
Dialog öffnen (1) & Betätigen des 'Neu'-Buttons im Senderbereich & Der Dialog
 zum Hinzufügen eines Senders erscheint.
\\
\hline
IP Adressen Test (2) & Eine invalide Testgruppen Adresse wird eingestellt, die
 restlichen Einstellungen behalten die Standardwerte bei. Anschließend wird 'OK' betätigt.
 & Es erscheint zusätzlich ein Dialog, welcher auf eine Fehleingabe hinweist.
 Der 'Erstellen'-Dialog bleibt offen.
\\
\hline
Alle invaliden IP-Adressen testen (3) & Wiederholen von 1 und 2 mitallen unter
'Definitionen' definierten invaliden Werten. &
\\
\hline
Dialog öffnen (4) & Betätigen des 'Neu'-Buttons im Senderbereich & Der Dialog
 zum Hinzufügen eines Senders erscheint.
\\
\hline
\end{tabular}
\end{center}
\end{table}

\begin{table}[h]
\begin{center}
\begin{tabular}{|p{2.5cm}|p{5cm}|p{4.55cm}|}
\hline
Port Test (5) & Den Port auf einen invaliden Port setzen und 'OK'
 betätigen. & Es erscheint zusätzlich ein Dialog, welcher auf eine Fehleingabe
 hinweist. Der Erstellen Dialog bleibt offen.
\\
\hline
Alle invaliden Ports testen (3) & Wiederholen
von 1 und 2 mitallen unter 'Definitionen' definierten invaliden Werten. &
\\
\hline
Dialog öffnen (6) & Betätigen des 'Neu' Buttons im Senderbereich & Der Dialog
 zum Hinzufügen eines Senders erscheint.
\\
\hline
Paketraten Test (7) & Die Packetrate auf einen invaliden Wert setzen und OK
 betätigen. & Es erscheint zusätzlich ein Dialog, welcher auf eine Fehleingabe
 hinweist. Der Erstellen Dialog bleibt offen.
\\
\hline
Alle invaliden Paketraten testen (3) & Wiederholen von 1 und 2 mitallen unter
'Definitionen' definierten invaliden Werten. &
\\
\hline
Dialog öffnen (8) & Betätigen des 'Neu' Buttons im Senderbereich & Der Dialog
 zum Hinzufügen eines Senders erscheint.
\\
\hline
Paketgrößentest (9) & Die Paketgröße auf einen invaliden Wert setzen und OK
 betätigen. &  Es erscheint zusätzlich ein Dialog, welcher auf eine Fehleingabe
 hinweist. Der Erstellen Dialog bleibt offen.
\\
\hline
Alle invaliden Paketgrößen testen (3) & Wiederholen von 1 und 2 mitallen unter
'Definitionen' definierten invaliden Werten. &
\\
\hline
Dialog öffnen (10) & Betätigen des 'Neu' Buttons im Senderbereich & Der Dialog
 zum Hinzufügen eines Senders erscheint.
\\
\hline
\end{tabular}
\end{center}
\end{table}

\begin{table}[h]
\begin{center}
\begin{tabular}{|p{2.5cm}|p{5cm}|p{4.55cm}|}
\hline
TTL Test (11) & Die 'Time To Live' auf einen invaliden Wert setzen und 'OK'
 betätigen. & Es erscheint zusätzlich ein Dialog, welcher auf eine Fehleingabe
 hinweist. Der 'Erstellen'-Dialog bleibt offen.
\\
\hline
Alle invaliden TTLs testen (3) & Wiederholen von 1 und 2 mitallen unter
'Definitionen' definierten invaliden Werten.
\\
\hline
\end{tabular}
\end{center}
\end{table}

\chapter{Testsuite /TS20/}
Dieses Kapitel enthält die Testergebnisse der Testsuite /TS20/.\\
Die zugehörige Requirement ID lautet /UC10/.

\begin{table}[h]
\caption{/TC2001/}
\label{tab:TC2001}
\begin{center}
\begin{tabular}{|p{3.5cm}|p{11cm}|}
\hline
\textbf{Testfall Id} & /TC2001/\\
\hline
\textbf{Testfall Name} & Anlegen eines Empfängers\\
\hline
\textbf{Requirement ID} & /VA0700/, /VA0800/, /VA0200/\\
\hline
\textbf{Testfall Setup} & Starten des Computers unter Linux. Durchführen des Tests.
Anschließendes Starten des Computers unter Windows XP. Erneutes Durchführen des Tests. Schlägt ein Testschritt unter mindestens einem System fehl, wird er als 'FAIL' markiert.\\
\hline
\end{tabular}
\begin{tabular}{|p{4cm}|p{7.8cm}|p{2.3cm}|}
\multicolumn{3}{|c|}{\textbf{Einzelschritte des Testfalls}} \\
\hline
\textbf{Schritt} & \textbf{Erwartetes Ergebnis} & \textbf{Ergebnis}\\
\hline
Programmstart (0) & GUI erscheint. & PASS\\
\hline
Dialog öffnen (1) & Dialog zum Hinzufügen eines neuen Empfängers öffnet sich. & PASS\\
\hline
IP-Adressen-Test (2) & Neuer Empfänger auf gewählter Gruppe erscheint im
Hauptfenster im Empfängerbereich. & PASS\\
\hline
Alle validen Gruppen prüfen (3) & Die Empfänger werden korrekt angelegt. & PASS\\
\hline
Dialog öffnen (4) & Dialog zum Hinzufügen eines neuen Empfängers öffnet sich. & PASS\\
\hline
Port-Test (5) & Neuer Empfänger auf gewähltem Port erscheint im
Hauptfenster im Empfängerbereich. & PASS\\
\hline
Alle validen Ports prüfen (3) & Die Empfänger werden korrekt angelegt. & PASS\\
\hline
\end{tabular}
\begin{tabular}{|p{3.5cm}|p{11cm}|}
\textbf{Tester} & DH\\
\hline
\textbf{Datum} & 13.05.2011\\
\hline
\textbf{Ergebnis} & PASS\\
\hline
\end{tabular}
\end{center}
\end{table}

\begin{table}[h]
\caption{/TC2002/}
\label{tab:TC2002}
\begin{center}
\begin{tabular}{|p{3.5cm}|p{11cm}|}
\hline
\textbf{Testfall Id} & /TC2002/\\
\hline
\textbf{Testfall Name} & Aktivieren von Empfängern\\
\hline
\textbf{Requirement ID} & /VA0700/, /VA0800/, /VA0200/\\
\hline
\textbf{Testfall Setup} & Starten des Computers unter Linux. Durchführen des Tests.
Anschließendes Starten des Computers unter Windows XP. Erneutes Durchführen des Tests. Schlägt ein Testschritt unter mindestens einem System fehl, wird er als 'FAIL' markiert.\\
\hline
\end{tabular}
\begin{tabular}{|p{4cm}|p{7.8cm}|p{2.3cm}|}
\multicolumn{3}{|c|}{\textbf{Einzelschritte des Testfalls}} \\
\hline
\textbf{Schritt} & \textbf{Erwartetes Ergebnis} & \textbf{Ergebnis}\\
\hline
TC2001 (1)& Hauptfenster mit angelegten,
deaktivierten Empfänger befindet sich auf dem Bildschirm. & PASS\\
\hline
Einzelne Aktivierung (2) & Empfänger wird aktiv, erkennbar am
Wechsel des roten Vierecks zu einem grünen Pfeil. & PASS\\
\hline
Multiple Aktivierung (3) & Empfänger werden aktiv,
erkennbar am Wechsel des roten Vierecks zu einem grünen Pfeil. & PASS\\
\hline
\end{tabular}
\begin{tabular}{|p{3.5cm}|p{11cm}|}
\textbf{Tester} & DH\\
\hline
\textbf{Datum} & 13.05.2011\\
\hline
\textbf{Ergebnis} & FAIL\\
\hline
\end{tabular}
\end{center}
\end{table}

\begin{table}[h]
\caption{/TC2003/}
\label{tab:TC2003}
\begin{center}
\begin{tabular}{|p{3.5cm}|p{11cm}|}
\hline
\textbf{Testfall Id} & /TC2003/\\
\hline
\textbf{Testfall Name} & Bearbeiten von Empfängern\\
\hline
\textbf{Requirement ID} & /VA0700/, /VA0800/, /VA0200/\\
\hline
\textbf{Testfall Setup} & Starten des Computers unter Linux. Durchführen des Tests.
Anschließendes Starten des Computers unter Windows XP. Erneutes Durchführen des Tests. Schlägt ein Testschritt unter mindestens einem System fehl, wird er als 'FAIL' markiert.\\
\hline
\end{tabular}
\begin{tabular}{|p{4cm}|p{7.8cm}|p{2.3cm}|}
\multicolumn{3}{|c|}{\textbf{Einzelschritte des Testfalls}} \\
\hline
\textbf{Schritt} & \textbf{Erwartetes Ergebnis} & \textbf{Ergebnis}\\
\hline
TC2001 (1)& Hauptfenster mit angelegten,
deaktivierten Empfänger befindet sich auf dem Bildschirm. & PASS\\
\hline
Öffnen des Bearbeiten-Dialogs (2) & Bearbeiten Dialog erscheint. & PASS\\
\hline
Bearbeiten eines inaktiven Empfängers (3) & Fenster schließt sich. & PASS\\
\hline
Öffnen des Bearbeiten-Dialogs & Bearbeiten Dialog erscheint. & PASS\\
\hline
Änderung überprüfen (4) & Analyse Verhalten ist auf 'lazy' eingestellt. & PASS\\
\hline
Öffnen des Bearbeiten-Dialogs (5) & Bearbeiten Dialog erscheint. & PASS\\
\hline
Bearbeiten eines aktiven Empfängers (6) & Fenster schliesst sich. & PASS\\
\hline
Wiederhole (5) & Bearbeiten Dialog erscheint. & PASS\\
\hline
Überprüfen (7) & Analyse Verhalten ist 'eager'. & PASS\\
\hline
\end{tabular}
\begin{tabular}{|p{3.5cm}|p{11cm}|}
\textbf{Tester} & DH\\
\hline
\textbf{Datum} & 13.05.2011\\
\hline
\textbf{Ergebnis} & PASS\\
\hline
\end{tabular}
\end{center}
\end{table}

\begin{table}[h]
\caption{/TC2004/}
\label{tab:TC2004}
\begin{center}
\begin{tabular}{|p{3.5cm}|p{11cm}|}
\hline
\textbf{Testfall Id} & /TC2004/\\
\hline
\textbf{Testfall Name} & Deaktivieren von Empfängern\\
\hline
\textbf{Requirement ID} & /VA0700/, /VA0800/, /VA0200/\\
\hline
\textbf{Testfall Setup} & Starten des Computers unter Linux. Durchführen des Tests.
Anschließendes Starten des Computers unter Windows XP. Erneutes Durchführen des Tests. Schlägt ein Testschritt unter mindestens einem System fehl, wird er als 'FAIL' markiert.\\
\hline
\end{tabular}
\begin{tabular}{|p{4cm}|p{7.8cm}|p{2.3cm}|}
\multicolumn{3}{|c|}{\textbf{Einzelschritte des Testfalls}} \\
\hline
\textbf{Schritt} & \textbf{Erwartetes Ergebnis} & \textbf{Ergebnis}\\
\hline
TC2002 & Hauptfenster mit angelegten,
aktiven Empfängern befindet sich auf dem Bildschirm. & PASS\\
\hline
Einzelne Deaktivierung (1) & Empfänger wird inaktiv,
erkennbar am Wechsel des grünen Pfeils zu einem roten Viereck. & PASS\\
\hline
Multiple Deaktivierung (2) & Empfänger werden
inaktiv, erkennbar am Wechsel des grünen Pfeils zu einem roten Viereck. & PASS\\
\hline
\end{tabular}
\begin{tabular}{|p{3.5cm}|p{11cm}|}
\textbf{Tester} & DH\\
\hline
\textbf{Datum} & 13.05.2011\\
\hline
\textbf{Ergebnis} & PASS\\
\hline
\end{tabular}
\end{center}
\end{table}

\begin{table}[h]
\caption{/TC2005/}
\label{tab:TC2005}
\begin{center}
\begin{tabular}{|p{3.5cm}|p{11cm}|}
\hline
\textbf{Testfall Id} & /TC2005/\\
\hline
\textbf{Testfall Name} & Entfernen von Empfängern\\
\hline
\textbf{Requirement ID} & /VA0700/, /VA0800/, /VA0200/\\
\hline
\textbf{Testfall Setup} & Starten des Computers unter Linux. Durchführen des Tests.
Anschließendes Starten des Computers unter Windows XP. Erneutes Durchführen des Tests. Schlägt ein Testschritt unter mindestens einem System fehl, wird er als 'FAIL' markiert.\\
\hline
\end{tabular}
\begin{tabular}{|p{4cm}|p{7.8cm}|p{2.3cm}|}
\multicolumn{3}{|c|}{\textbf{Einzelschritte des Testfalls}} \\
\hline
\textbf{Schritt} & \textbf{Erwartetes Ergebnis} & \textbf{Ergebnis}\\
\hline
TC2001 & Hauptfenster mit angelegten,
inaktiven Empfängern befindet sich auf dem Bildschirm. & PASS\\
\hline
Einzelnes Entfernen (1) & Empfänger ist nicht mehr
in der Liste vorhanden. & PASS\\
\hline
Multiples Entfernen (2) & Empfänger sind nicht mehr
in der Liste vorhanden. & PASS\\
\hline
\end{tabular}
\begin{tabular}{|p{3.5cm}|p{11cm}|}
\textbf{Tester} & DH\\
\hline
\textbf{Datum} & 13.05.2011\\
\hline
\textbf{Ergebnis} & PASS\\
\hline
\end{tabular}
\end{center}
\end{table}

\begin{table}[h]
\caption{/TC2006/}
\label{tab:TC2006}
\begin{center}
\begin{tabular}{|p{3.5cm}|p{11cm}|}
\hline
\textbf{Testfall Id} & /TC2004/\\
\hline
\textbf{Testfall Name} & Abfangen invalider Angaben\\
\hline
\textbf{Requirement ID} & /VA0700/, /VA0800/, /VA0200/\\
\hline
\textbf{Testfall Setup} & Starten des Computers unter Linux. Durchführen des Tests.
Anschließendes Starten des Computers unter Windows XP. Erneutes Durchführen des Tests. Schlägt ein Testschritt unter mindestens einem System fehl, wird er als 'FAIL' markiert.\\
\hline
\end{tabular}
\begin{tabular}{|p{4cm}|p{7.8cm}|p{2.3cm}|}
\multicolumn{3}{|c|}{\textbf{Einzelschritte des Testfalls}} \\
\hline
\textbf{Schritt} & \textbf{Erwartetes Ergebnis} & \textbf{Ergebnis}\\
\hline
Programmstart (0) & GUI
erscheint. & PASS\\
\hline
Dialog öffnen (1) & Dialog zum Hinzufügen eines neuen Empfängers öffnet sich. & PASS\\
\hline
IP-Adressen-Test (2) & Dialog bleibt offen und Fehlermeldung über falsche
Gruppen-Adresse wird angezeigt. & PASS\\
\hline
Alle validen Gruppen prüfen (3) & Dialog bleibt offen und Fehlermeldung über falsche
Gruppen-Adresse wird angezeigt. & PASS\\
\hline
Dialog öffnen (4) & Dialog zum Hinzufügen eines neuen Empfängers öffnet sich. & PASS\\
\hline
Port-Test (5) & Dialog bleibt offen und Fehlermeldung über falschen
Port erscheint. & PASS\\
\hline
Alle invaliden Ports prüfen (3) & Dialog bleibt offen und Fehlermeldung über falschen
Port erscheint. & PASS\\
\hline
\end{tabular}
\begin{tabular}{|p{3.5cm}|p{11cm}|}
\textbf{Tester} & DH\\
\hline
\textbf{Datum} & 13.05.2011\\
\hline
\textbf{Ergebnis} & PASS\\
\hline
\end{tabular}
\end{center}
\end{table}
\chapter{Testsuite /TS30/}

Dieser Anhang enthält die Testfälle der Testsuite /TS30/.\\
Die zugehörige Requirement ID lautet /UC10/.\\
\\
Die Testfälle dieser Testsuite sollen die korrekte Konfigurationsmöglichkeit mit Hilfe von Profilen über die GUI sicherstellen.\\
Dazu zählen zum einen das Abspeichern von Profilen, das Laden von Profilen und der Zugriff auf zuletzt verwendete Profile von dem Programm aus.\\
\\
Spezifiziert im Anhang sind valide und invalide Konfigurationsdateien.

\begin{table}[h]
\caption{/TC3001/}
\label{tab:TC3001}
\begin{center}
\begin{tabular}{|p{3.5cm}|p{9cm}|}
\hline
\textbf{Testfall Id} & /TC3001/\\
\hline
\textbf{Testfall Name} & Laden valider Konfigurationsdateien \\
\hline
\textbf{Requirement ID} & /VA0500/ (/LVA0600/), /VA0600/ (/LVA0800/) \\
\hline
\textbf{Beschreibung} & Testen des Ladens und Verarbeiten valider Konfigurationsdateien. \\
\hline
\end{tabular}
\begin{tabular}{|p{2.5cm}|p{5cm}|p{4.55cm}|}
\multicolumn{3}{|c|}{\textbf{Einzelschritte des Testfalls}} \\
\hline
\textbf{Schritt} & \textbf{Aktion} & \textbf{Ergebnis}\\
\hline
Test-Setup & Das Multicast-Test-Tool installieren & Das Tool ist erfolgreich
installiert. \\
\hline
Konfigura- tionen laden & Für jede valide Konfiguration (wie oben spezifiziert) wird das Tool gestartet. & Das Tool lädt die Konfiguration ohne Probleme. \\
\hline
Überprüfen geladener Werte & Überprüfen der Anzeigen in der GUI um zu sehen ob alle Daten richtig geladen wurden. & In der GUI des Tools werden alle Daten aus der Konfiguration richtig angezeigt. \\
\hline
\end{tabular}
\end{center}
\label{default}
\end{table}

\begin{table}[h]
\caption{/TC3002/}
\label{tab:TC3002}
\begin{center}
\begin{tabular}{|p{3.5cm}|p{9cm}|}
\hline
\textbf{Testfall Id} & /TC3002/\\
\hline
\textbf{Testfall Name} & Laden invalider Konfigurationsdateien \\
\hline
\textbf{Requirement ID} & /VA0600/ (/LVA0800/), /QZ10/ (/LQZ10/) \\
\hline
\textbf{Beschreibung} & Testen des Ladens und Verarbeiten invalider Konfigurationsdateien. 
Die Software darf bei fehlerhafter Konfiguration durch den
Benutzer nicht abstürzen. Darunter fallen ungültige Multicast-
Adressen, ungültige Netzwerkgeräte oder Syntaxfehler in Konfigurationsdateien.\\
\hline
\end{tabular}
\begin{tabular}{|p{2.5cm}|p{5cm}|p{4.55cm}|}
\multicolumn{3}{|c|}{\textbf{Einzelschritte des Testfalls}} \\
\hline
\textbf{Schritt} & \textbf{Aktion} & \textbf{Ergebnis}\\
\hline
Test-Setup & Das Multicast-Test-Tool installieren & Das Tool ist erfolgreich installiert. \\
\hline
Konfigura- tionen laden & Für jede invalide Konfiguration (wie oben spezifiziert) wird das Tools gestartet. & Das Tool lädt die Konfiguration und gibt eine Fehlermeldung aus die mindestens eines der Probleme in der Konfiguration wieder spiegelt.\\
\hline
Reaktion des Programms & Überprüfen ob das Tool sinnvoll auf den Fehler reagiert. & Das Tool startet und geht sinnvoll mit Fehlern um. Das Tool kann entscheiden wie viele Informationen es aus der Konfiguration In der GUI des Tools werden entweder sinnvollen Daten aus der Konfiguration richtig angezeigt oder das Tool beendet sich.\\
\hline
\end{tabular}
\end{center}
\label{default}
\end{table}

\begin{table}[h]
        \caption{/TC3003/}
        \label{tab:TC3003}
        \begin{center}
            \begin{tabular}{|p{3.5cm}|p{12cm}|}
                \hline
                    \textbf{Testfall Id} & /TC3003/\\
                \hline
                    \textbf{Testfall Name} & Profil Lebenszyklus Test\\
                \hline
                    \textbf{Requirement ID} & /VA0500/ (/LVA0600/), /AVA0600/ (/VA0600/), /VA0700/ (/LVA0900/)\\
                \hline
                    \textbf{Beschreibung} & Dieser Test verifiziert, dass das
                    Tool in der Lage ist, ein Profil zu erstellen und dann wieder
                    zu laden. Dazu konfiguriert der User im laufenden Multicast-Tool Gruppen. Darauf folgend klickt der Benutzer auf den Speichern-Button. Danach wird das Tool neu gestartet und
die gespeicherte Datei wird geladen. Erwartetes Ergebnis:
Nach dem speichern muss eine Datei im XML-Format vorhanden sein. Nach dem laden
der Datei müssen alle Gruppen mit den Einstellungen wieder hergestellt sein.\\
                \hline
            \end{tabular}
            \begin{tabular}{|p{3.5cm}|p{5cm}|p{6.55cm}|}
                \multicolumn{3}{|c|}{\textbf{Einzelschritte des Testfalls}} \\
                \hline
                    \textbf{Schritt} & \textbf{Aktion} & \textbf{Ergebnis}\\
                \hline
                    Test-Setup &
                    Das Multicast-Test-Tool mit grafischer Oberfläche starten & 
                    Das Tool ist bereit für das Anlegen von Datenströmen \\
                \hline
                    Konfigurationen Anlegen &
                    Anlegen eines Senders und dann eines Receivers.
                    Hierbei sollte sichergestellt werden, dass alle eingegebenen
                    Werte nicht den Standardwerten entsprechen. &
                    Sender und Receiver werden angelegt \\
                \hline
                    Speichern des Profiles & 
                    Profil über das Menü 'Speicher Als ...' speichern und Angeben eines 
                    Profil Namens und Speicherorts. & Profil wird gespeichert.
                    \\
                \hline
                    Tool beenden &
                    Tool beenden &
                    Tool wird beenden, der java Prozess wird beendet\\
                \hline
                    Tool erneut starten &
                    Das Multicast-Test-Tool mit grafischer Oberfläche starten & 
                    Das Tool ist bereit für das Anlegen von Datenströmen \\
                \hline
                    Startzustand überprüfen &
                    Es ist kein Sender oder Receiver angelegt & 
                    Es ist kein Sender oder Receiver angelegt \\
                \hline
                    Profil laden &
                    Profil über das Menü 'Profil laden ...' laden und Angabe des 
                    vorherigen Speicherorts. & 
                    Das Profil wird erfolgreich geladen. Der Fenstertitel der 
                    Applikation spiegelt den Namen des Profils wieder.\\
                \hline
            \end{tabular}
        \end{center}
    \end{table}
    
    \begin{table}[h]
        \begin{center}
           \begin{tabular}{|p{3.5cm}|p{5cm}|p{6.55cm}|}
                \hline
                    Konfigurationen überprüfen &
                    Prüfen der Einträge in den Listen.
                    Durch klicken auf Edit überprüfen ob der Sender und Receiver
                    die korrekten Werte haben (die Werte die beim Anlegen vor
                    dem speichern des Profils definiert wurden) & 
                    In der Liste des Senders und Receivers ist jeweils ein Eintrag.
                    Die Werte in den Edit Dialogen sind korrekt.\\
                \hline
            \end{tabular}
        \end{center}
    \end{table}



\begin{table}[h]
\caption{/TC3004/}
\label{tab:TC3004}
\begin{center}
\begin{tabular}{|p{3.5cm}|p{9cm}|}
\hline
\textbf{Testfall Id} & /TC3004/\\
\hline
\textbf{Testfall Name} & Laden einer zuletzt verwendeten Konfigurationsdatei \\
\hline
\textbf{Requirement ID} &  \\
\hline
\textbf{Beschreibung} & Testen des Ladens einer zuletzt verwendeten Konfigurationsdatei.\\
\hline
\end{tabular}
\begin{tabular}{|p{2.5cm}|p{5cm}|p{4.55cm}|}

\multicolumn{3}{|c|}{\textbf{Einzelschritte des Testfalls}} \\
\hline
\textbf{Schritt} & \textbf{Aktion} & \textbf{Ergebnis}\\
\hline
Test-Setup & Das Multicast-Test-Tool installieren & Das Tool ist erfolgreich installiert. \\
\hline
Konfigura- tionen speichern & Einige Einstellungen werden vorgenommen und die Konfiguration gespeichert & Konfiguration wird gespeichert.\\
\hline
Applikation beenden & Applikation wird beendet & Applikation beendet sich.\\
\hline
Applikation starten & Applikation starten & Applikation startet\\
\hline
Letzte Konfiguration laden & Menü nach letzter Konfiguration durchsuchen und laden & Letzte Konfiguration wird geladen.\\
\hline
Letzte Konfiguration überprüfen & Überprüfen ob alle Werte der Konfiguration erfolgreich geladen wurden. & Die letzte Konfiguration wurde erfolgreich geladen.\\
\hline
Konfigura- tion speichern & Konfiguration unter neuem Namen speichern. & Konfiguration wird gespeichert. \\
\hline
Applikation beenden & Applikation wird beendet & Applikation beendet sich.\\
\hline
Applikation starten & Applikation starten & Applikation startet\\
\hline
Letzte Konfigurationen überprüfen & Menü nach letzten Konfiguration durchsuchen & Letzte Konfiguration und vorletzte Konfiguration stehen zur Auswahl.\\
\hline
\end{tabular}
\end{center}
\label{default}


\end{table}


\chapter{Testsuite /TS40/}
Dieses Kapitel enthält die Testergebnisse der Testsuite /TS40/.\\
Die zugehörige Requirement ID lautet /UC20/.

\begin{table}[h]
    \caption{/TC4001/}
    \label{tab:TC4001}
    \begin{center}
        \begin{tabular}{|p{3.5cm}|p{11cm}|}
            \hline
                \textbf{Testfall ID} & /TC4001/\\
            \hline
                \textbf{Testfall Name} & CLI-Konfiguration\\
            \hline
                \textbf{Requirement ID} & VA0900 \& VA0600\\
            \hline
                \textbf{Testfall Setup} & Installiere das MC-Tool.\\
            \hline
        \end{tabular}
        \begin{tabular}{|p{2cm}|p{3.9cm}|p{3.9cm}|p{3.8cm}|}
            \multicolumn{4}{|c|}{\textbf{Einzelschritte des Testfalls}} \\
            \hline
                \textbf{Schritt} & \textbf{Aktion} & \textbf{Erwartetes
                Ergebnis} & \textbf{Ergebnis}\\
            \hline
                Starten und Konfiguration laden - korrekte Pfadangabe & Eingabe
                des Befehls zum Starten und Laden einer Konfiguration mit
                korrekter Pfadangabe zu einer Konfigurationsdatei & Anlegen
                aller Sender und Empfänger im selben Zustand, wie zum Zeitpunkt
                des Speicherns \& Ausgabe auf der Commandline, die den Erfolg
                des Ladens bestätigt. & PASS \\
            \hline
                Konfiguration laden - fehlerhafte Pfadangabe & Eingabe des
                Befehls zum Starten und Laden einer Konfiguration mit
                fehlerhafter Pfadangabe zu einer Konfigurationsdatei & Ausgabe
                einer Fehlermeldung, dass keine Datei gefunden wurde &
                PASS \\
            \hline
                Konfiguration laden - fehlerhafte Datei & Eingabe des Befehls
                zum Starten und Laden einer Konfiguration mit Pfadangabe zu
                einer fehlerhaften/nicht kompatiblen Datei & Ausgabe einer
                Fehlermeldung, dass die gelesene Datei fehlerhaft ist &
                PASS \\
            \hline
        \end{tabular}
        \begin{tabular}{|p{3.5cm}|p{11cm}|}
                \textbf{Tester} & TST\\
            \hline
                \textbf{Datum} & 14.05.2011\\
            \hline
                \textbf{Ergebnis} & PASS\\
            \hline
        \end{tabular}
    \end{center}
\end{table}

\begin{table}[h]
    \caption{/TC4002/}
    \label{tab:TC4002}
    \begin{center}
        \begin{tabular}{|p{3.5cm}|p{11cm}|}
            \hline
                \textbf{Testfall ID} & /TC4002/\\
            \hline
                \textbf{Testfall Name} & Starten aller Sender und Empfänger per
                CLI\\
            \hline
                \textbf{Requirement ID} & VA0900 \& VA0600\\
            \hline
                \textbf{Testfall Setup} & Installiere das MC-Tool\\
            \hline
        \end{tabular}
        \begin{tabular}{|p{2cm}|p{3.9cm}|p{3.9cm}|p{3.8cm}|}
            \multicolumn{4}{|c|}{\textbf{Einzelschritte des Testfalls}} \\
            \hline
                \textbf{Schritt} & \textbf{Aktion} & \textbf{Erwartetes
                Ergebnis} & \textbf{Ergebnis}\\
            \hline
                Eingabe des gültigen Start-Befehls & Eingabe des Befehls zum
                Starten des Programms, sowie zum Laden und Starten aller Sender
                und Empfänger & Ausgabe auf der Commandline, die den Erfolg des
                Startens bestätigt & PASS \\
            \hline
                Eingabe eines ungültigen Start-Befehls & Eingabe eines
                ungültigen Befehls zum Starten des Programms, sowie zum Laden
                und Starten aller Empfänger & Ausgabe auf CLI, die die ungültige
                Eingabe quittiert & PASS \\
            \hline
        \end{tabular}
        \begin{tabular}{|p{3.5cm}|p{11cm}|}
                \textbf{Tester} & TST\\
            \hline
                \textbf{Datum} & 14.05.2011\\
            \hline
                \textbf{Ergebnis} & PASS\\
            \hline
        \end{tabular}
    \end{center}
\end{table}

\begin{table}[h]
    \caption{/TC4003/}
    \label{tab:TC4003}
    \begin{center}
        \begin{tabular}{|p{3.5cm}|p{11cm}|}
            \hline
                \textbf{Testfall ID} & /TC4003/\\
            \hline
                \textbf{Testfall Name} & Starten des Programms ohne Starten der
                Sender und Empfänger per CLI \\
            \hline
                \textbf{Requirement ID} & VA0900 \& VA0600\\
            \hline
                \textbf{Testfall Setup} & Installiere das MC-Tool\\
            \hline
        \end{tabular}
        \begin{tabular}{|p{2cm}|p{3.9cm}|p{3.9cm}|p{3.8cm}|}
            \multicolumn{4}{|c|}{\textbf{Einzelschritte des Testfalls}} \\
            \hline
                \textbf{Schritt} & \textbf{Aktion} & \textbf{Erwartetes
                Ergebnis} & \textbf{Ergebnis}\\
            \hline
                Eingabe des gültigen Start-Befehls & Eingabe des Befehls zum
                Starten des Programms, sowie zum Laden und Starten keiner Sender
                und Empfänger & Ausgabe auf der Commandline, die den Erfolg des
                Startens des Programms bestätigt & PASS\\
            \hline
                Eingabe eines ungültigen Start-Befehls & Eingabe eines ungültigen
                Befehls zum Starten des Programms, sowie zum Laden und Starten
                keiner Sender und Empfänger & Ausgabe auf CLI, die die ungültige
                Eingabe quittiert & PASS \\
            \hline
        \end{tabular}
        \begin{tabular}{|p{3.5cm}|p{11cm}|}
                \textbf{Tester} & TST\\
            \hline
                \textbf{Datum} & 14.05.2011\\
            \hline
                \textbf{Ergebnis} & PASS\\
            \hline
        \end{tabular}
    \end{center}
\end{table}

\begin{table}[h]
    \caption{/TC4004/}
    \label{tab:TC4004}
    \begin{center}
        \begin{tabular}{|p{3.5cm}|p{11cm}|}
            \hline
                \textbf{Testfall ID} & /TC4004/\\
            \hline
                \textbf{Testfall Name} & Ausgabe der Daten im CLI\\
            \hline
                \textbf{Requirement ID} & VA0900 \& VA0600\\
            \hline
                \textbf{Testfall Setup} & Installiere das MC-Tool\\
            \hline
        \end{tabular}
        \begin{tabular}{|p{2cm}|p{3.9cm}|p{3.9cm}|p{3.8cm}|}
            \multicolumn{4}{|c|}{\textbf{Einzelschritte des Testfalls}} \\
            \hline
                \textbf{Schritt} & \textbf{Aktion} & \textbf{Erwartetes
                Ergebnis} & \textbf{Ergebnis}\\
            \hline
                Test-Setup Teil 2 & Starten des MC-Tools mit gültigem Pfad und
                Start-Befehl & Bestätigung auf CLI & PASS\\
            \hline
                Beobachten der Ausgabe & Auf Update der Ausgabe warten & Ausgabe
                der Statistik auf dem CLI & PASS\\
            \hline
        \end{tabular}
        \begin{tabular}{|p{3.5cm}|p{11cm}|}
                \textbf{Tester} & TST\\
            \hline
                \textbf{Datum} & 14.05.2011\\
            \hline
                \textbf{Ergebnis} & PASS\\
            \hline
        \end{tabular}
    \end{center}
\end{table}

\begin{table}[h]
    \caption{/TC4005/}
    \label{tab:TC4005}
    \begin{center}
        \begin{tabular}{|p{3.5cm}|p{11cm}|}
            \hline
                \textbf{Testfall ID} & /TC4005/\\
            \hline
                \textbf{Testfall Name} & Loggen der Statistik\\
            \hline
                \textbf{Requirement ID} & VA0900 \& VA0600\\
            \hline
                \textbf{Testfall Setup} &Installiere das MC-Tool\\
            \hline
        \end{tabular}
        \begin{tabular}{|p{2cm}|p{3.9cm}|p{3.9cm}|p{3.8cm}|}
            \multicolumn{4}{|c|}{\textbf{Einzelschritte des Testfalls}} \\
            \hline
                \textbf{Schritt} & \textbf{Aktion} & \textbf{Erwartetes
                Ergebnis} & \textbf{Ergebnis}\\
            \hline
                Test-Setup Teil 2 & Starten des MC-Tool via Commandline mit
                gültigem Start-Befehl und gestarteten Sendern/Receivern &
                Bestätigung in Commandline Interface & PASS\\
            \hline
                Korrekte Log-Daten & Öffnen der Log-Datei und Vergleich mit
                Ausgabe auf dem CLI & Logger funktioniert fehlerfrei -
                geloggte Daten sind korrekt & PASS\\
            \hline
                Inkorrekte Log-Daten & Öffnen der Log-Datei und Vergleich mit
                Ausgabe auf dem CLI & Logger funktioniert fehlerhaft - geloggte
                Daten stimmen nicht mit dem CLI überein & 
                PASS\\
            \hline
                Inkorrektes Log-Format & Öffnen der Log-Datei und betrachten des
                Log-Formats & Daten werden nicht als XML gespeichert. &
                PASS\\
            \hline
        \end{tabular}
        \begin{tabular}{|p{3.5cm}|p{11cm}|}
                \textbf{Tester} & TST\\
            \hline
                \textbf{Datum} & 14.05.2011\\
            \hline
                \textbf{Ergebnis} & PASS\\
            \hline
        \end{tabular}
    \end{center}
\end{table}

\begin{table}[h]
    \caption{/TC4006/}
    \label{tab:TC4006}
    \begin{center}
        \begin{tabular}{|p{3.5cm}|p{11cm}|}
            \hline
                \textbf{Testfall ID} & /TC4006/\\
            \hline
                \textbf{Testfall Name} & Starten des MC-Tools ohne GUI\\
            \hline
                \textbf{Requirement ID} & VA0900 \& VA0600\\
            \hline
                \textbf{Testfall Setup} & Installiere das MC-Tool\\
            \hline
        \end{tabular}
        \begin{tabular}{|p{2cm}|p{3.9cm}|p{3.9cm}|p{3.8cm}|}
            \multicolumn{4}{|c|}{\textbf{Einzelschritte des Testfalls}} \\
            \hline
                \textbf{Schritt} & \textbf{Aktion} & \textbf{Erwartetes
                Ergebnis} & \textbf{Ergebnis}\\
            \hline
                Starten des MC-Tools & Starten des MC-Tool via Commandline mit
                gültigem Start-Befehl und nogui Parameter & Programm wird ohne
                GUI gestartet. & PASS\\
            \hline
        \end{tabular}
        \begin{tabular}{|p{3.5cm}|p{11cm}|}
                \textbf{Tester} & TST\\
            \hline
                \textbf{Datum} & 14.05.2011\\
            \hline
                \textbf{Ergebnis} & \cellcolor{green} PASS \\
            \hline
        \end{tabular}
    \end{center}
\end{table}
\chapter{Testsuite /TS50/}

    Dieses Kapitel enthält die Testergebnisse der Testsuite /TS50/.\\
    Die zugehörige Requirement ID lautet /UC30/.\\
    \newline
    Die Testfälle dieser Testsuite sollen die korrekte Übermittlung von
    Multicast-Paketen sicherstellen. Dazu zählen zum Einen die fehlerfreie
    Übermittlung der Pakete (ohne Störquellen) und die korrekte Übertragung der
    in den Paketen enthaltenen Nutzdaten (Payload).\\
    \newline
    Alle Tests dieser Suite arbeiten mit mindestens zwei Instanzen des
    MC-Test-Tools. Dadurch werden Verfälschungen des Ergebnisses durch eine
    innerprogrammliche Kommunikation von Sender- und Empfängerseite
    ausgeschlossen. Bei Verwendung eines physischen Netzwerks
    zwischen Sender und Empfänger ist dessen Multicastfähigkeit im Vorfeld zu
    verifizieren (z.B. durch das Hirschmann-Tool). Sie wird in den folgenden
    Testfällen als gegeben vorausgesetzt.\\
    \newline
    Die folgenden Testfälle verwenden die Formulierung "`im selben Netzwerk"'
    und "`der gewählte Netzwerkadapter"'. Damit ist gemeint:
    Alle Tesfälle, die mit mehreren Instanzen des Multicast Test Tools arbeiten,
    sind mit Option a) durchzuführen. Zusätzlich sind alle diese Testfälle
    entweder mit Option b) oder c) durchzuführen.

    \begin{table}[h]
        \caption{/TC5001/}
        \label{tab:TC5001}
        \begin{center}
           \begin{tabular}{|p{3.5cm}|p{11cm}|}
                \hline
                    \textbf{Testfall ID} & /TC5001/\\
                \hline
                    \textbf{Testfall Name} & Multicast-Pakete senden\\
                \hline
                    \textbf{Requirement ID} & /VA0100/\\
                \hline
                    \textbf{Beschreibung} & Dieser Test verifiziert, dass das
                    Tool in der Lage ist, UDP-Pakete an eine
                    IP-Multicast-Gruppe addressiert zu verschicken.\\
                \hline
            \end{tabular}
\begin{tabular}{|p{4cm}|p{7.8cm}|p{2.3cm}|}
\multicolumn{3}{|c|}{\textbf{Einzelschritte des Testfalls}} \\
                \hline
                    \textbf{Schritt} & \textbf{Erwartetes Ergebnis} & \textbf{Ergebnis}\\
                \hline
                    Test-Setup 1 & 
                    Das Tool ist bereit für das Anlegen von Datenströmen & Pass\\
                \hline
                    Test-Setup 2 &
                    Wireshark ist bereit für Benutzereingaben & Pass\\
                \hline
                    Wireshark aktivieren
                    lauschen lassen & 
                    Wireshark zeigt diversen Netzwerkverkehr am gewählten
                    Adapter an& Pass\\
                \hline
                    Multicast-Sender aktivieren &
                    Der neu erstellte Datenstrom erscheint in der Liste, wird
                    als aktiviert und mit einer Senderate größer null
                    angezeigt& Pass\\
                \hline
                    Wireshark-Output überprüfen &
                    Es werden Pakete registriert, die an die im Sender
                    angegebene Multicast-Gruppe adressiert sind. Die Pakete
                    sind anhand des 1337-/x0539-Headers im Datenblock als
                    SPAM-Pakete identifizierbar.& Pass\\
                \hline
            \end{tabular}     
                   \begin{tabular}{|p{3.5cm}|p{11cm}|}
                \textbf{Tester} & KW\\
                \hline
                \textbf{Datum} & 13.05.2011\\
                \hline
                \textbf{Ergebnis} & PASS\\
                \hline
            \end{tabular}
        \end{center}
    \end{table}

    \begin{table}[h]
        \caption{/TC5002/}
        \label{tab:TC5002}
        \begin{center}
            \begin{tabular}{|p{3.5cm}|p{11cm}|}
                \hline
                    \textbf{Testfall ID} & /TC5002/\\
                \hline
                    \textbf{Testfall Name} & Einfacher Datenstrom mit
                                             Nutzdaten und Jumbo-Paketen\\
                \hline
                    \textbf{Requirement ID} & /VA0100/, /VA0200/, /VA0500/,
                    /OA0400/\\
                \hline
                    \textbf{Beschreibung} & Dieser Test verifiziert, dass ein
                                            einzelner, von einem Sender
                                            verschickter Datenstrom, bei einem
                                            Empfänger ankommt. Außerdem wird
                                            überprüft, dass im Sender
                                            konfigurierte Nutzdaten
                                            unverfälscht beim Empfänger
                                            ankommen und dass auch
                                            Jumbo-Pakete mit einer Größe von
                                            9000 Byte versendet und empfangen
                                            werden können.\\
                \hline
            \end{tabular}
\begin{tabular}{|p{4cm}|p{7.8cm}|p{2.3cm}|}
\multicolumn{3}{|c|}{\textbf{Einzelschritte des Testfalls}} \\
                \hline
                    \textbf{Schritt} & \textbf{Erwartetes Ergebnis} & \textbf{Ergebnis}\\
                \hline
                    Test-Setup &
                    Das Tool wird zwei Mal ausgeführt und ist jeweils bereit zum
                    Anlegen von Sendern und Empfängern.& Pass\\
                \hline
                    Sender starten 
                    & Der Sender erscheint in
                    der Liste und wird als aktiviert und mit einer Senderate
                    größer null angezeigt.& Pass\\
                \hline
                    Empfänger starten &
                    Der Empfänger erscheint in der Liste und wird als
                    aktiviert angezeigt.& Pass\\
                \hline
                    Datenstrom identifizieren &
                    Ein Datenstrom, der dieselbe Sender ID, wie der vorher
                    gestartete Sender aufweist, erscheint als Unterpunkt des
                    erstellten Empfängers.& Pass\\
                \hline
                    Nutzdaten überprüfen &
                    In der empfangenden Instanz werden dieselben Nutzdaten, die
                    beim Anlegen des Senders angegeben wurden ("`Hallo Welt"')
                    beim entsprechenden Datenstrom angezeigt.& Pass\\
                \hline
                    Jumbo-Pakete versenden &
                    Der Datenstrom wird weiterhin mit den korrekten Nutzdaten
                    beim Empfänger angezeigt.& Pass\\
                \hline
            \end{tabular}
                   \begin{tabular}{|p{3.5cm}|p{11cm}|}
                \textbf{Tester} & KW\\
                \hline
                \textbf{Datum} & 13.05.2011\\
                \hline
                \textbf{Ergebnis} & PASS\\
                \hline
            \end{tabular}
        \end{center}
    \end{table}

    \begin{table}[h]
        \caption{/TC5003/}
        \label{tab:TC5003}
        \begin{center}
            \begin{tabular}{|p{3.5cm}|p{11cm}|}
                \hline
                    \textbf{Testfall ID} & /TC5003/\\
                \hline
                    \textbf{Testfall Name} & Datenstrom an mehrere Empfänger\\
                \hline
                    \textbf{Requirement ID} & /VA0100/, /VA0200/\\
                \hline
                    \textbf{Beschreibung} & Dieser Test verifiziert, dass das
                    Tool echte Multicasts verschickt, in dem Sinne, dass sie
                    von mehreren Empfängern empfangen werden können.\\
                \hline
            \end{tabular}
            \begin{tabular}{|p{4cm}|p{7.8cm}|p{2.3cm}|}
\multicolumn{3}{|c|}{\textbf{Einzelschritte des Testfalls}} \\
                \hline
                    \textbf{Schritt} & \textbf{Erwartetes Ergebnis} & \textbf{Ergebnis}\\
                \hline
                    Test-Setup 1 &
                    Die sendende Instanz ist bereit für Benutzereingaben. & Pass\\
                \hline
                    Test-Setup 2 & 
                    Die empfangenden Instanzen sind bereit für
                    Benutzereingaben.& Pass\\
                \hline
                    Sender anlegen &
                    Ein neuer Sender erscheint in der Liste.& Pass\\
                \hline
                    Empfänger anlegen &
                    In jeder empfangenden Instanz erscheint ein Empfänger.& Pass\\
                \hline
                    Empfänger starten &
                    Die Empfänger werden als aktiv in der Liste angezeigt.& Pass\\
                \hline
                    Sender starten &
                    Der Sender wird als aktiv und mit einer Senderate größer
                    null angezeigt.& Pass\\
                \hline
                    Empfänger-Output prüfen &
                    Jede der empfangenden Instanzen zeigt als Unterpunkt der
                    zuvor angelegten Empfänger einen Datenstrom mit der Sender ID
                    des zuvor angelegten Senders an.& Pass\\
                \hline
            \end{tabular}
                   \begin{tabular}{|p{3.5cm}|p{11cm}|}
                \textbf{Tester} & KW\\
                \hline
                \textbf{Datum} & 13.05.2011\\
                \hline
                \textbf{Ergebnis} & PASS\\
                \hline
            \end{tabular}
        \end{center}
    \end{table}

    \begin{table}[h]
        \caption{/TC5004/}
        \label{tab:TC5004}
        \begin{center}
            \begin{tabular}{|p{3.5cm}|p{11cm}|}
                \hline
                    \textbf{Testfall ID} & /TC5004/\\
                \hline
                    \textbf{Testfall Name} & Lasttest\\
                \hline
                    \textbf{Requirement ID} & /VA0100/, /VA0200/, /VA0300/\\
                \hline
                    \textbf{Beschreibung} & Dieser Test verifiziert, dass das
                    Tool mindestens 30 Sendeströme gleichzeitig senden, bzw.
                    empfangen kann. Außerdem wird geprüft, dass das Tool auch
                    unter diesen Bedingungen noch ohne Einschränkungen
                    bedienbar ist.\\
                \hline
            \end{tabular}
            \begin{tabular}{|p{4cm}|p{7.8cm}|p{2.3cm}|}
\multicolumn{3}{|c|}{\textbf{Einzelschritte des Testfalls}} \\
                \hline
                    \textbf{Schritt} & \textbf{Erwartetes Ergebnis} & \textbf{Ergebnis}\\
                \hline
                    Test-Setup & 
                    Beide Instanzen sind bereit für Benutzereingaben.& Pass\\
                \hline
                    Sender starten &
                    Alle 30 Ströme werden als aktiv und mit einer
                    eingestellten und gemessenen Senderate von jeweils 10
                    Pakten pro Sekunde angezeigt.& Pass\\
                \hline
                    Empfänger starten &
                    Der erstellte Empfänger erscheint als aktiv in der Liste.
                    Ihm untergeordnet erscheinen 30 Sendeströme mit einer
                    gemessenen Paketrate von jeweils 10 Paketen pro Sekunde.& Pass\\
                \hline
                    Senderperformance prüfen &
                    Die grafische Oberfläche reagiert weiterhin ohne merkliche
                    Verzögerung auf Benutzereingaben.& Pass\\
                \hline
                    Empfängerperformance prüfen &
                    Die grafische Oberfläche reagiert weiterhin ohne merkliche
                    Verzögerung auf Benutzereingaben.& Pass\\
                \hline
            \end{tabular}
                   \begin{tabular}{|p{3.5cm}|p{11cm}|}
                \textbf{Tester} & KW\\
                \hline
                \textbf{Datum} & 13.05.2011\\
                \hline
                \textbf{Ergebnis} & PASS\\
                \hline
            \end{tabular}
        \end{center}
    \end{table}

    \begin{table}[h]
        \caption{/TC5005/}
        \label{tab:TC5005}
        \begin{center}
            \begin{tabular}{|p{3.5cm}|p{11cm}|}
                \hline
                    \textbf{Testfall ID} & /TC5005/\\
                \hline
                    \textbf{Testfall Name} & Empfang von Hirschmann-Paketen\\
                \hline
                    \textbf{Requirement ID} & /VA0200/, /VA0400/\\
                \hline
                    \textbf{Beschreibung} & Dieser Test verifiziert die
                    Fähigkeit des Tools, Pakete des Hirschmann Tools zu
                    empfangen und zu verarbeiten.\\
                \hline
            \end{tabular}
            \begin{tabular}{|p{4cm}|p{7.8cm}|p{2.3cm}|}
\multicolumn{3}{|c|}{\textbf{Einzelschritte des Testfalls}} \\
                \hline
                    \textbf{Schritt} & \textbf{Erwartetes Ergebnis} & \textbf{Ergebnis}\\
                \hline
                    Test-Setup 1 &
                    Das Tool ist bereit für Benutzereingaben& Pass\\
                \hline
                    Test-Setup 2 &
                    Das Hirschmann Tool ist bereit für Benutzereingaben. & Pass\\
                \hline
                    Empfänger starten &
                    Der angelegte Empfänger wird in der Liste als aktiv
                    angezeigt.& Pass\\
                \hline
                    Sender starten &
                    Der Sender wird als aktiv in der Liste angezeigt.& Pass\\
                \hline
                    Empfang verifizieren &
                    Der Datenstrom des Hirschmann Tools erscheint als
                    Unterpunkt des zuvor angelegten Empfängers. Die
                    gemessene Datenrate entspricht weitgehend der im Hirschmann
                    Tool angegebenen.& Pass\\
                \hline
            \end{tabular}
                   \begin{tabular}{|p{3.5cm}|p{11cm}|}
                \textbf{Tester} & KW\\
                \hline
                \textbf{Datum} & 13.05.2011\\
                \hline
                \textbf{Ergebnis} & PASS\\
                \hline
            \end{tabular}
        \end{center}
    \end{table}

    \begin{table}[h]
        \caption{/TC5006/}
        \label{tab:TC5006}
        \begin{center}
            \begin{tabular}{|p{3.5cm}|p{11cm}|}
                \hline
                    \textbf{Testfall ID} & /TC5006/\\
                \hline
                    \textbf{Testfall Name} & Abonnierte Gruppen empfangen
                    \\
                \hline
                    \textbf{Requirement ID} & /VA0100/, /VA0200/\\
                \hline
                    \textbf{Beschreibung} &  Dieser Test verifiziert die
                    korrekte Umsetzung des Multicast-Protokolls. Er überprüft
                    den Gruppen-Abonnement-Mechanismus und stellt sicher, dass
                    tatsächlich nur die Abonnierten Gruppen empfangen und
                    ausgewertet werden.\\
                \hline
            \end{tabular}
            \begin{tabular}{|p{4cm}|p{7.8cm}|p{2.3cm}|}
\multicolumn{3}{|c|}{\textbf{Einzelschritte des Testfalls}} \\
                \hline
                    \textbf{Schritt} & \textbf{Erwartetes Ergebnis} & \textbf{Ergebnis}\\
                \hline
                    Test-Setup 1 &
                    Die Tools sind bereit für Benutzereingaben.& Pass\\
                \hline
                    Sender erzeugen &
                    Die drei Datenströme werden als aktiv und mit einer
                    Senderate größer null in der Liste angezeigt.& Pass\\
                \hline
                    Empfänger anlegen &
                    Beide Empfänger werden als aktiv in der Liste angezeigt.& Pass\\
                \hline
                    Empfangende Datenströme verifizieren &
                    Es werden nur die Datenströme in Gruppe 225.1.1.1 und
                    225.1.1.2 empfangen und angezeigt. Gruppe 225.1.1.3 wird
                    nicht empfangen.& Pass\\
                \hline
            \end{tabular}
                   \begin{tabular}{|p{3.5cm}|p{11cm}|}
                \textbf{Tester} & KW\\
                \hline
                \textbf{Datum} & 13.05.2011\\
                \hline
                \textbf{Ergebnis} & PASS\\
                \hline
            \end{tabular}
        \end{center}
    \end{table}

\chapter{Testsuite /TS60/}
Dieses Kapitel enthält die Testergebnisse der Testsuite /TS60/.\\
Die zugehörige Requirement Id lautet /UC40/.

\begin{table}[h]
        \caption{/TC6001/}
        \label{tab:TC6001}
        \begin{center}
            \begin{tabular}{|p{3.5cm}|p{12cm}|}
                \hline
                    \textbf{Testfall Id} & /TC6001/\\
                \hline
                    \textbf{Testfall Name} & Sendestatistik eines Sender\\
                \hline
                    \textbf{Requirement ID} & /VA1000/, /VA1200/, /OA0300/\\
                \hline
                    \textbf{Beschreibung} & Dieser Testfall verifiziert die
                    korrekte Erstellung von Statistiken für einen einzelnen
                    Sender.\\
                \hline
            \end{tabular}
            \begin{tabular}{|p{2.5cm}|p{7.55cm}|p{5cm}|}
                \multicolumn{3}{|c|}{\textbf{Einzelschritte des Testfalls}} \\
                \hline
                    \textbf{Schritt} & \textbf{Erwartetes
                    Ergebnis} & \textbf{Ergebnis}\\
                \hline
                    Test-Setup & Das Tool
                    wird ausgeführt und ist bereit zum Starten von Sendern. &
                    Pass
                    \\
                \hline
                    Sender starten & Der Sender
                    sendet und wird angezeigt. Die maximale, minimale und
                    durchschnittliche Senderate liegt bei 10 PPS. &
                    Pass\\
                \hline
                    Erhöhen der Senderate & Die minimale Senderate
                    liegt bei 10 PPS, die maximale bei 1000 PPS, die
                    durchschnittliche bei 1000 PPS. &
                    Pass\\
                \hline
                    Pausieren des Senders & Die Statistiken bleiben erhalten.
                    Die minimale Senderate wird nicht verändert. &
                    Pass\\
                \hline
                    Aktivieren des Senders & Die
                    Statistiken bleiben erhalten. &
                    Pass\\
                \hline
                    Hinzufügen eines neuen Senders & Die durchschnittliche, maximale und minimale
                    Senderate liegen bei 1000 PPS. &
                    Pass\\
                \hline
                    Überlasten von Sendern & Die minimale Senderate wird jeweils bei niedrigerer
                    durchschnittlicher Senderate angepasst. &
                    Pass\\
                \hline
                    Detailansicht & Die angezeigten Statistiken
                    stimmen mit den Statistiken der Tabelle überein. Der Graph
                    wird um die jeweilige durchschnittliche Senderate ergänzt.
                    & Pass\\
                \hline
            \end{tabular}
            \begin{tabular}{|p{3.5cm}|p{11cm}|}
                \textbf{Tester} & RS\\
                \hline
                \textbf{Datum} & 13.05.2011\\
                \hline
                \textbf{Ergebnis} & PASS\\
                \hline
            \end{tabular}
        \end{center}
    \end{table}
    
    \begin{table}[h]
        \caption{/TC6002/}
        \label{tab:TC6002}
        \begin{center}
            \begin{tabular}{|p{3.5cm}|p{12cm}|}
                \hline
                    \textbf{Testfall Id} & /TC6002/\\
                \hline
                    \textbf{Testfall Name} & Globale Sender-Statistiken\\
                \hline
                    \textbf{Requirement ID} & /VA1000/\\
                \hline
                    \textbf{Beschreibung} & Dieser Testfall verifiziert die
                    korrekte Erstellung von globalen Sender-Statistiken.\\
                \hline
            \end{tabular}
            \begin{tabular}{|p{2.5cm}|p{5cm}|p{7.55cm}|}
                \multicolumn{3}{|c|}{\textbf{Einzelschritte des Testfalls}} \\
                \hline
                    \textbf{Schritt} &  & \textbf{Ergebnis}\\
                \hline
                    Test-Setup & Das Tool
                    wird ausgeführt und ist bereit zum Starten von Sendern. \\
                \hline
                    Sender starten & Der Sender
                    sendet und wird angezeigt. Die Senderate liegt bei 10 PPS.
                    Die globale Senderate ist identisch. Der Zähler für die
                    Anzahl insgesamt gesendeter Pakete erhöht sich pro Sekunde
                    im Schnitt um 10 Pakete. & Pass\\
                \hline
                    Sender hinzufügen & Die globale Senderate
                    liegt bei 100 PPS. Der Zähler für die Anzahl insgesamt
                    gesendeter Pakete erhöht sich pro Sekunde im Schnitt um 100
                    Pakete. & Pass\\
                \hline
                    Sender hinzufügen & Die globale Senderate
                    liegt bei 1000 PPS. Der Zähler für die Anzahl insgesamt
                    gesendeter Pakete erhöht sich pro Sekunde im Schnitt um 1000
                    Pakete. & Pass\\
                \hline
                    Sender pausieren & Die globale Senderate liegt bei 100 PPS. Der
                    Zähler für die Anzahl insgesamt gesendeter Pakete erhöht
                    sich pro Sekunde im Schnitt um 100 Pakete. & Pass\\
                \hline
                    Sender löschen & Die globale Senderate liegt bei 10 PPS. Der Zähler für die
                    Anzahl insgesamt gesendeter Pakete erhöht sich pro Sekunde
                    im Schnitt um 100 Pakete.  & Pass\\
                \hline
                    Sender pausieren & Die globale Senderate liegt bei 0 PPS. Der
                    Zähler für die Anzahl insgesamt gesendeter Pakete erhöht
                    sich nicht. & Pass\\
                \hline
                    Sender hinzufügen & Die globale Senderate
                    liegt bei 500 PPS. Der Zähler für die Anzahl insgesamt
                    gesendeter Pakete erhöht sich pro Sekunde im Schnitt um 500
                    Pakete. & Pass\\
                \hline
                    Sender überlasten & Die
                    globale Senderate schwankt um einen Wert. & Pass\\
                \hline
            \end{tabular}
             \begin{tabular}{|p{3.5cm}|p{11cm}|}
                \textbf{Tester} & RS\\
                \hline
                \textbf{Datum} & 13.05.2011\\
                \hline
                \textbf{Ergebnis} & PASS\\
                \hline
            \end{tabular}
        \end{center}
    \end{table}
    
    \begin{table}[h]
        \caption{/TC6003/}
        \label{tab:TC6003}
        \begin{center}
            \begin{tabular}{|p{3.5cm}|p{12cm}|}
                \hline
                    \textbf{Testfall Id} & /TC6003/\\
                \hline
                    \textbf{Testfall Name} & Paketraten-Statistik auf
                    Empfängerseite\\
                \hline
                    \textbf{Requirement ID} & /VA1000/, /VA1200/, /VA1300/\\
                \hline
                    \textbf{Beschreibung} & Dieser Testfall verifiziert die
                    korrekte Erstellung von Statistiken über die
                    gemessene Paketrate für einen empfangenen Datenstrom.\\
                \hline
            \end{tabular}
            \begin{tabular}{|p{2.5cm}|p{7.55cm}|p{5cm}|}
                \multicolumn{3}{|c|}{\textbf{Einzelschritte des Testfalls}} \\
                \hline
                    \textbf{Schritt} & \textbf{Erwartetes Ergebnis} &
                    \textbf{Ergebnis}\\
                \hline
                    Test-Setup & Das Tool
                    wird ausgeführt und ist bereit zum Starten von Sendern. &
                    Pass\\
                \hline
                    Sender starten & Der
                    Sender sendet und wird angezeigt. & Pass\\
                \hline
                    Empfänger hinzufügen & Der Empfänger empfängt einen Datenstrom mit 100 PPS.\\
                \hline
                    Sender hinzufügen & Der Empfänger empfängt die Datenströme zweier Sender mit 100 bzw. 1000
                    PPS (jeweils konfigurierte und gemessene Paketrate). &
                    Pass\\
                \hline
                    Verändern der Senderate & Der Empfänger empfängt die Datenströme
                    zweier Sender mit je 100 PPS als konfigurierte und
                    gemessene Paketrate. & Pass\\
                \hline
                    Sender pausieren & Der Empfänger
                    zeigt den Datenstrom als pausiert an. Die Statistiken
                    bleiben erhalten. & Pass\\
                \hline
                    Sender aktivieren & Der Empfänger zeigt den Sender als aktiv an. Die Statistiken
                    werden fortgesetzt. & Pass\\
                \hline
                    Überlasten der Sender & Pakete gehen verloren. Die Empfangsrate der Datenströme schwankt analog
                    zu den Sendern. & Pass\\
                \hline
            \end{tabular}
        \begin{tabular}{|p{3.5cm}|p{11cm}|}
                \textbf{Tester} & RS\\
                \hline
                \textbf{Datum} & 13.05.2011\\
                \hline
                \textbf{Ergebnis} & PASS\\
                \hline
            \end{tabular}
        \end{center}
    \end{table}
    
    \begin{table}[h]
        \caption{/TC6004/}
        \label{tab:TC6004}
        \begin{center}
            \begin{tabular}{|p{3.5cm}|p{12cm}|}
                \hline
                    \textbf{Testfall Id} & /TC6004/\\
                \hline
                    \textbf{Testfall Name} & Globale Empfänger-Statistiken\\
                \hline
                    \textbf{Requirement ID} & /VA1000/, /VA1300/\\
                \hline
                    \textbf{Beschreibung} & Dieser Testfall verifiziert die
                    korrekte Erstellung globaler Empfänger-Statistiken.\\
                \hline
            \end{tabular}
            \begin{tabular}{|p{2.5cm}|p{7.55cm}|p{5cm}|}
                \multicolumn{3}{|c|}{\textbf{Einzelschritte des Testfalls}} \\
                \hline
                    \textbf{Schritt} &  & \textbf{Ergebnis}\\
                \hline
                    Test-Setup & Das Tool
                    wird ausgeführt und ist bereit zum Starten von Sendern. &
                    Pass\\
                \hline
                    Sender Starten & Der
                    Sender sendet und wird angezeigt. & Pass\\
                \hline
                    Empfänger hinzufügen &
                    Der Empfänger empfängt einen Datenstrom mit 100 PPS. Die
                    globale Empfangsrate liegt bei 100 PPS. Der Zähler für
                    insgesamt empfangene Pakete wird pro Sekunde um 100
                    Pakete erhöht. & Pass\\
                \hline
                    Sender hinzufügen & Der Empfänger empfängt zwei Datenströme
                    mit 100 bzw. 900 PPS. Die globale Empfangsrate liegt bei 1000 PPS. Der Zähler für
                    insgesamt empfangene Pakete wird pro Sekunde im Schnitt um
                    1000 Pakete erhöht. & Pass\\
                \hline
                    Sender pausieren & Die globale Empfangsrate liegt bei 100 PPS. Der Zähler für
                    insgesamt empfangene Pakete wird pro Sekunde im Schnitt um
                    1000 Pakete erhöht. & Pass\\
                \hline
                    Sender aktivieren & Die globale Empfangsrate liegt bei 1000 PPS. Der Zähler für
                    insgesamt empfangene Pakete wird pro Sekunde im Schnitt um
                    1000 Pakete erhöht. & Pass\\
                \hline
                    Überlasten der Sender Pakete gehen verloren. Die Summe der verlorenen Pakete der einzelnen
                    Datenströme stimmt mit den global verlorenen Paketen
                    überein. & Pass\\
                \hline
            \end{tabular}
            \begin{tabular}{|p{3.5cm}|p{11cm}|}
                \textbf{Tester} & RS\\
                \hline
                \textbf{Datum} & 13.05.2011\\
                \hline
                \textbf{Ergebnis} & PASS\\
                \hline
            \end{tabular}
        \end{center}
    \end{table}
    
    \begin{table}[h]
        \caption{/TC6005/}
        \label{tab:TC6005}
        \begin{center}
            \begin{tabular}{|p{3.5cm}|p{12cm}|}
                \hline
                    \textbf{Testfall Id} & /TC6005/\\
                \hline
                    \textbf{Testfall Name} & Detaillierte Statistik eines
                    empfangenen Datenstromes\\
                \hline
                    \textbf{Requirement ID} & /VA1000/, /VA1200/, /VA1300/,
                    /OA0300/\\
                \hline
                    \textbf{Beschreibung} & Dieser Testfall verifiziert die
                    korrekte Erstellung von Statistiken für einen einzelnen
                    empfangenen Datenstrom sowie für den einzelnen Empfänger,
                    der diesen Datenstrom empfängt.\\
                \hline
            \end{tabular}
            \begin{tabular}{|p{2.5cm}|p{5cm}|p{7.55cm}|}
                \multicolumn{3}{|c|}{\textbf{Einzelschritte des Testfalls}} \\
                \hline
                    \textbf{Schritt} & \textbf{Ergebnis}\\
                \hline
                    Test-Setup & Das Tool wird auf beiden Rechnern
                    ausgeführt und ist bereit zum Starten von Sendern. & Pass\\
                \hline
                    Zeit Synchronisierung & Die Zeiten
                    beider Computer sind hinreichend synchron. & Pass\\
                \hline
                    Sender Starten & Der Sender sendet und wird
                    angezeigt. & PAss\\
                \hline
                    Empfänger anlegen & Der Empfänger empfängt den Sender aus
                    dem Netzwerk. & Pass\\
                \hline
                    Detailansicht Paketraten & Die konfigurierte und gemessene Senderate
                    stimmen mit den Daten des Senders überein. Die gemessene
                    Empfangsrate stimmt mit der Tabellenangabe überein. & Pass\\
                \hline
                    Detailansicht Übertragungszeit & Die angezeigte Übetragungszeit stimmt
                    mit der Hälfte der Ping-Zeit überein. & Pass\\
                \hline
                    Detailansicht maximaler Versatz & Maximaler Versatz wird
                    angezeigt. & Pass\\
                \hline
            \end{tabular}
            \begin{tabular}{|p{3.5cm}|p{11cm}|}
                \textbf{Tester} & RS\\
                \hline
                \textbf{Datum} & 13.05.2011\\
                \hline
                \textbf{Ergebnis} & PASS\\
                \hline
            \end{tabular}
        \end{center}
    \end{table}
\chapter{Testsuite /TS70/}
Dieser Anhang enthält die Testfälle der Testsuite /TS70/.\\
Die zugehörige Requirement ID lautet /QP30/.\\
\newline
Die Testfälle dieser Testsuite sollen die korrekte und einfache Installation der Applikation sowohl unter Windows als auch unter Linux sicherstellen.\\

\begin{table}[h]
\caption{/TC7001/}
\label{tab:TC7001}
\begin{center}
\begin{tabular}{|p{3.5cm}|p{12cm}|}
\hline
\textbf{Testfall Id} & /TC7001/\\
\hline
\textbf{Testfall Name} & Installation unter Windows\\
\hline
\textbf{Requirement ID} & /QP30/\\
\hline
\textbf{Beschreibung} & Dieser Testfall verifiziert die, für einen erfahrenen Benutzer, einfache Installation der Applikation.\\
\hline
\end{tabular}
\begin{tabular}{|p{2.5cm}|p{5cm}|p{7.55cm}|}
\multicolumn{3}{|c|}{\textbf{Einzelschritte des Testfalls}} \\
\hline
\textbf{Schritt} & \textbf{Aktion} & \textbf{Ergebnis}\\
\hline
Installer herunterladen & Herunterladen des Installers & Die URL ist valide und der Download startet. Der Download muss auch ohne eine Anmeldung auf der Seite funktionieren.\\
\hline
Installation starten & Ausführen des heruntergeladenen Installers. & Ein Dialog öffnet sich. \\
\hline
Konfiguration der Installation & Der Installationswizzard wird verwendet. & Wenn Konfigurationsmöglichkeiten vorhanden sind, so beschränken sie sich nur auf allgemein bekannte Einstellungen, wie das Angeben eines Installationspfads.\\
\hline
Installation & Die Applikation wird installiert. & Der Fortschritt der Installation wird angezeigt. \\
\hline
Starten der Applikation & Die nun installierte Applikation wird über das Start Menü gestartet. & Die Applikation startet. \\
\hline
\end{tabular}
\end{center}
\label{default}
\end{table}

\begin{table}[h]
\caption{/TC7002/}
\label{tab:TC7002}
\begin{center}
\begin{tabular}{|p{3.5cm}|p{12cm}|}
\hline
\textbf{Testfall Id} & /TC7002/\\
\hline
\textbf{Testfall Name} & Installation unter Linux (Ubuntu)\\
\hline
\textbf{Requirement ID} & /QP30/\\
\hline
\textbf{Beschreibung} & Dieser Testfall verifiziert die korrekte Installation unter Linux (Ubuntu).\\
\hline
\end{tabular}
\begin{tabular}{|p{2.5cm}|p{5cm}|p{7.55cm}|}
\multicolumn{3}{|c|}{\textbf{Einzelschritte des Testfalls}} \\
\hline
\textbf{Schritt} & \textbf{Aktion} & \textbf{Ergebnis}\\
\hline
Debian Paket herunterladen & Herunterladen des Debian Pakets. & Die URL ist valide und der Download startet. Der Download muss auch ohne eine Anmeldung auf der Seite funktionieren.\\
\hline
Debian Paket installieren (GUI) & Ein Doppelklick auf das Icon der heruntergeladenen Datei. & Start eines Installationsdialogs. \\
\hline
Debian Paket installieren (CLI) & Ausführen des Befehls sudo dpkg -i [file] & Installation der Applikation.\\
\hline
Installation & Die Applikation wird installiert. & Der Fortschritt der Installation wird angezeigt. \\
\hline
Starten der Applikation & Die nun installierte Applikation wird über das Start Menü gestartet. & Die Applikation startet. \\
\hline
\end{tabular}
\end{center}
\label{default}
\end{table}

\chapter{Test Eingabewerte}

Dieser Anhang enthält Eingabewerte für Testfälle, die zu umfangreich sind, um sie direkt mit den Testfällen zu speichern.\\

\paragraph{Valide XML Konfiguration A}
~\\Konfiguration ohne Sender und Receiver\\
\lstinputlisting[language=XML]{./listings/valid-profile-A.xml}
\newpage 

\paragraph{Valide XML Konfiguration B}
~\\Konfiguration mit Newline in der Payload\\
\lstinputlisting[language=XML]{./listings/valid-profile-B.xml}
\newpage 

\paragraph{Valide XML Konfiguration C}
~\\Konfiguration mit mehreren Sendern und Receivern und vielen Werten\\
\lstinputlisting[language=XML]{./listings/valid-profile-C.xml}
\newpage 


\paragraph{Invalide XML Konfiguration A}
~\\Konfiguration mit invalidem XML\\
\lstinputlisting[language=XML]{./listings/invalid-profile-A.xml}
\newpage 

\paragraph{Invalide XML Konfiguration B}
~\\Konfiguration mit invalider Gruppe\\
\lstinputlisting[language=XML]{./listings/invalid-profile-B.xml}
\newpage 

\paragraph{Invalide XML Konfiguration C}
~\\Konfiguration mit doppeltem <senders> Element \\
\lstinputlisting[language=XML]{./listings/invalid-profile-C.xml}
\newpage 

\paragraph{Invalide XML Konfiguration D}
~\\Konfiguration mit invalidem Packet Typ. Es werden nur so wenige 
Werte für Elemente getestet, da man davon ausgehen kann, dass Fehler die hier auftreten schon bei GUI Tests
gefunden wurden.\\
\lstinputlisting[language=XML]{./listings/invalid-profile-D.xml}
\newpage 



\end{appendix}
