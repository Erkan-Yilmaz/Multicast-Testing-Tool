\chapter{Testsuite /TS60/}
    
    Dieser Anhang enthält die Testfälle der Testsuite /TS60/.\\
    Die zugehörige Requirement ID lautet /UC40/.\\
    \newline
    Die Testfälle dieser Testsuite sollen die korrekte Analyse der
    übertragenen/empfangenen Multicast-Pakete und deren Anzeige in der GUI
    sicherstellen.\\
    Dazu zählt die Auswertung der Messdaten pro Sender-Stream und pro
    Empfänger-Stream, für alle Sender und für alle Empfänger.

    \begin{table}[h]
        \caption{/TC6001/}
        \label{tab:TC6001}
        \begin{center}
            \begin{tabular}{|p{3.5cm}|p{12cm}|}
                \hline
                    \textbf{Testfall Id} & /TC6001/\\
                \hline
                    \textbf{Testfall Name} & Sendestatistik eines Sender\\
                \hline
                    \textbf{Requirement ID} & /VA1000/, /VA1200/, /OA0300/\\
                \hline
                    \textbf{Beschreibung} & Dieser Testfall verifiziert die
                    korrekte Erstellung von Statistiken für einen einzelnen
                    Sender.\\
                \hline
            \end{tabular}
            \begin{tabular}{|p{2.5cm}|p{5cm}|p{7.55cm}|}
                \multicolumn{3}{|c|}{\textbf{Einzelschritte des Testfalls}} \\
                \hline
                    \textbf{Schritt} & \textbf{Aktion} & \textbf{Ergebnis}\\
                \hline
                    Test-Setup & Installieren und Starten des MC-TOOL & Das Tool
                    wird ausgeführt und ist bereit zum Starten von Sendern. \\
                \hline
                    Sender starten & Anlegen eines Senders mit niedriger
                    Senderate, 10 PPS, und Aktivieren des Senders & Der Sender
                    sendet und wird angezeigt. Die maximale, minimale und
                    durchschnittliche Senderate liegt bei 10 PPS. \\
                \hline
                    Erhöhen der Senderate & Die Senderate wird über
                    "`Editieren"' auf 1000 PPS erhöht. &  Die minimale Senderate
                    liegt bei 10 PPS, die maximale bei 1000 PPS, die
                    durchschnittliche bei 1000 PPS.\\
                \hline
                    Pausieren des Senders & Der Sender wird deaktiviert. & Die
                    Statistiken bleiben erhalten. Die minimale Senderate wird
                    nicht verändert. \\
                \hline
                    Aktivieren des Senders & Der Sender wird aktiviert. & Die
                    Statistiken bleiben erhalten. \\
                \hline
                    Hinzufügen eines neuen Senders & Hinzufügen eines Senders
                    mit 1000 PPS. & Die durchschnittliche, maximale und minimale
                    Senderate liegen bei 1000 PPS. \\
                \hline
                    Überlasten von Sendern & Schrittweises Hinzufügen von
                    Sendern mit 1000 PPS, bis die durchschnittliche Senderate
                    einzelner Sender unter die maximale gemessene Senderate
                    fällt. & Die minimale Senderate wird jeweils bei niedrigerer
                    durchschnittlicher Senderate angepasst.\\
                \hline
                    Detailansicht & Öffnen der Detailansicht eines Senders mit
                    schwankender Senderate. & Die angezeigten Statistiken
                    stimmen mit den Statistiken der Tabelle überein. Der Graph
                    wird um die jeweilige durchschnittliche Senderate ergänzt.\\
                \hline
            \end{tabular}
        \end{center}
    \end{table}
    
    \begin{table}[h]
        \caption{/TC6002/}
        \label{tab:TC6002}
        \begin{center}
            \begin{tabular}{|p{3.5cm}|p{12cm}|}
                \hline
                    \textbf{Testfall Id} & /TC6002/\\
                \hline
                    \textbf{Testfall Name} & Globale Sender-Statistiken\\
                \hline
                    \textbf{Requirement ID} & /VA1000/\\
                \hline
                    \textbf{Beschreibung} & Dieser Testfall verifiziert die
                    korrekte Erstellung von globalen Sender-Statistiken.\\
                \hline
            \end{tabular}
            \begin{tabular}{|p{2.5cm}|p{5cm}|p{7.55cm}|}
                \multicolumn{3}{|c|}{\textbf{Einzelschritte des Testfalls}} \\
                \hline
                    \textbf{Schritt} & \textbf{Aktion} & \textbf{Ergebnis}\\
                \hline
                    Test-Setup & Installieren und Starten des MC-TOOL & Das Tool
                    wird ausgeführt und ist bereit zum Starten von Sendern. \\
                \hline
                    Sender starten & Anlegen eines Senders mit niedriger
                    Senderate, 10 PPS und Aktivieren des Senders & Der Sender
                    sendet und wird angezeigt. Die Senderate liegt bei 10 PPS.
                    Die globale Senderate ist identisch. Der Zähler für die
                    Anzahl insgesamt gesendeter Pakete erhöht sich pro Sekunde
                    im Schnitt um 10 Pakete.\\
                \hline
                    Sender hinzufügen & Anlegen eines Senders mit 90 PPS
                    Senderate und Aktivieren des Senders & Die globale Senderate
                    liegt bei 100 PPS. Der Zähler für die Anzahl insgesamt
                    gesendeter Pakete erhöht sich pro Sekunde im Schnitt um 100
                    Pakete.\\
                \hline
                    Sender hinzufügen & Anlegen eines Senders mit 900 PPS
                    Senderate und Aktivieren des Senders & Die globale Senderate
                    liegt bei 1000 PPS. Der Zähler für die Anzahl insgesamt
                    gesendeter Pakete erhöht sich pro Sekunde im Schnitt um 1000
                    Pakete.\\
                \hline
                    Sender pausieren & Deaktivieren des Senders mit 900 PPS
                    Senderate. & Die globale Senderate liegt bei 100 PPS. Der
                    Zähler für die Anzahl insgesamt gesendeter Pakete erhöht
                    sich pro Sekunde im Schnitt um 100 Pakete.\\
                \hline
                    Sender löschen & Löschen des Senders mit 90 PPS Senderate. &
                    Die globale Senderate liegt bei 10 PPS. Der Zähler für die
                    Anzahl insgesamt gesendeter Pakete erhöht sich pro Sekunde
                    im Schnitt um 100 Pakete.\\
                \hline
                    Sender pausieren & Deaktivieren des Senders mit 10 PPS
                    Senderate. & Die globale Senderate liegt bei 0 PPS. Der
                    Zähler für die Anzahl insgesamt gesendeter Pakete erhöht
                    sich nicht.\\
                \hline
                    Sender hinzufügen & Anlegen eines Senders mit 500 PPS
                    Senderate und Aktivieren des Senders & Die globale Senderate
                    liegt bei 500 PPS. Der Zähler für die Anzahl insgesamt
                    gesendeter Pakete erhöht sich pro Sekunde im Schnitt um 500
                    Pakete.\\
                \hline
                    Sender überlasten & Hinzufügen von Sendern mit 1000 PPS
                    Senderate und Aktivieren der Senders, bis die
                    durchschnittliche Senderate einzelner Sender schwankt. & Die
                    globale Senderate schwankt um einen Wert.\\
                \hline
            \end{tabular}
        \end{center}
    \end{table}
    
    \begin{table}[h]
        \caption{/TC6003/}
        \label{tab:TC6003}
        \begin{center}
            \begin{tabular}{|p{3.5cm}|p{12cm}|}
                \hline
                    \textbf{Testfall Id} & /TC6003/\\
                \hline
                    \textbf{Testfall Name} & Paketraten-Statistik auf
                    Empfängerseite\\
                \hline
                    \textbf{Requirement ID} & /VA1000/, /VA1200/, /VA1300/\\
                \hline
                    \textbf{Beschreibung} & Dieser Testfall verifiziert die
                    korrekte Erstellung von Statistiken über die
                    gemessene Paketrate für einen empfangenen Datenstrom.\\
                \hline
            \end{tabular}
            \begin{tabular}{|p{2.5cm}|p{5cm}|p{7.55cm}|}
                \multicolumn{3}{|c|}{\textbf{Einzelschritte des Testfalls}} \\
                \hline
                    \textbf{Schritt} & \textbf{Aktion} & \textbf{Ergebnis}\\
                \hline
                    Test-Setup & Installieren und Starten des MC-TOOL & Das Tool
                    wird ausgeführt und ist bereit zum Starten von Sendern. \\
                \hline
                    Sender starten & Anlegen eines Senders mit Senderate 100
                    PPS auf Loopback-Interface und Aktivieren des Senders & Der
                    Sender sendet und wird angezeigt.\\
                \hline
                    Empfänger hinzufügen & Anlegen eines Empfängers
                    auf Loopback-Interface mit selber Gruppen-Adresse und Port &
                    Der Empfänger empfängt einen Datenstrom mit 100 PPS.\\
                \hline
                    Sender hinzufügen & Anlegen eines Senders mit Senderate
                    1000 PPS auf Loopback-Interface mit selber Gruppen-Adresse
                    und Port und Aktivieren des Senders & Der Empfänger
                    empfängt die Datenströme zweier Sender mit 100 bzw. 1000
                    PPS (jeweils konfigurierte und gemessene Paketrate).\\
                \hline
                    Verändern der Senderate & Verändern der Senderate von 1000
                    PPS auf 100 PPS. & Der Empfänger empfängt die Datenströme
                    zweier Sender mit je 100 PPS als konfigurierte und
                    gemessene Paketrate.\\
                \hline
                    Sender pausieren & Pausieren eines Senders & Der Empfänger
                    zeigt den Datenstrom als pausiert an. Die Statistiken
                    bleiben erhalten.\\
                \hline
                    Sender aktivieren & Aktivieren des deaktivierten Senders &
                    Der Empfänger zeigt den Sender als aktiv an. Die Statistiken
                    werden fortgesetzt.\\
                \hline
                    Überlasten der Sender & Hinzufügen und aktivieren von
                    Sendern auf selber Multicast-Adresse und Port, bis die
                    Senderate der Sender beginnt zu schwanken. & Pakete gehen
                    verloren. Die Empfangsrate der Datenströme schwankt analog
                    zu den Sendern.\\
                \hline
            \end{tabular}
        \end{center}
    \end{table}
    
    \begin{table}[h]
        \caption{/TC6004/}
        \label{tab:TC6004}
        \begin{center}
            \begin{tabular}{|p{3.5cm}|p{12cm}|}
                \hline
                    \textbf{Testfall Id} & /TC6004/\\
                \hline
                    \textbf{Testfall Name} & Globale Empfänger-Statistiken\\
                \hline
                    \textbf{Requirement ID} & /VA1000/, /VA1300/\\
                \hline
                    \textbf{Beschreibung} & Dieser Testfall verifiziert die
                    korrekte Erstellung globaler Empfänger-Statistiken.\\
                \hline
            \end{tabular}
            \begin{tabular}{|p{2.5cm}|p{5cm}|p{7.55cm}|}
                \multicolumn{3}{|c|}{\textbf{Einzelschritte des Testfalls}} \\
                \hline
                    \textbf{Schritt} & \textbf{Aktion} & \textbf{Ergebnis}\\
                \hline
                    Test-Setup & Installieren und Starten des MC-TOOL & Das Tool
                    wird ausgeführt und ist bereit zum Starten von Sendern. \\
                \hline
                    Sender Starten & Anlegen eines Senders mit Senderate von 100
                    PPS auf Loopback-Interface und Aktivieren des Senders & Der
                    Sender sendet und wird angezeigt.\\
                \hline
                    Empfänger hinzufügen & Anlegen eines Empfängers
                    auf Loopback-Interface mit selber Gruppen-Adresse und Port &
                    Der Empfänger empfängt einen Datenstrom mit 100 PPS. Die
                    globale Empfangsrate liegt bei 100 PPS. Der Zähler für
                    insgesamt empfangene Pakete wird pro Sekunde um 100
                    Pakete erhöht.\\
                \hline
                    Sender hinzufügen & Anlegen eines Senders mit Senderate von
                    900 PPS auf Loopback-Interface mit selber Gruppen-Adresse
                    und Port und Aktivieren des Senders & Der Empfänger
                    empfängt zwei Datenströme mit 100 bzw. 900 PPS. Die
                    globale Empfangsrate liegt bei 1000 PPS. Der Zähler für
                    insgesamt empfangene Pakete wird pro Sekunde im Schnitt um
                    1000 Pakete erhöht.\\
                \hline
                    Sender pausieren & Pausieren des Senders mit 900 PPS. & Die
                    globale Empfangsrate liegt bei 100 PPS. Der Zähler für
                    insgesamt empfangene Pakete wird pro Sekunde im Schnitt um
                    1000 Pakete erhöht.\\
                \hline
                    Sender aktivieren & Aktivieren des deaktivierten Senders & 
                    Die globale Empfangsrate liegt bei 1000 PPS. Der Zähler für
                    insgesamt empfangene Pakete wird pro Sekunde im Schnitt um
                    1000 Pakete erhöht.\\
                \hline
                    Überlasten der Sender & Hinzufügen und Aktivieren von
                    Sendern auf selber Gruppen-Adresse und Port, bis die
                    Senderate der Sender beginnt zu schwanken. & Pakete gehen
                    verloren. Die Summe der verlorenen Pakete der einzelnen
                    Datenströme stimmt mit den global verlorenen Paketen
                    überein.\\
                \hline
            \end{tabular}
        \end{center}
    \end{table}
    
    \begin{table}[h]
        \caption{/TC6005/}
        \label{tab:TC6005}
        \begin{center}
            \begin{tabular}{|p{3.5cm}|p{12cm}|}
                \hline
                    \textbf{Testfall Id} & /TC6005/\\
                \hline
                    \textbf{Testfall Name} & Detaillierte Statistik eines
                    empfangenen Datenstromes\\
                \hline
                    \textbf{Requirement ID} & /VA1000/, /VA1200/, /VA1300/,
                    /OA0300/\\
                \hline
                    \textbf{Beschreibung} & Dieser Testfall verifiziert die
                    korrekte Erstellung von Statistiken für einen einzelnen
                    empfangenen Datenstrom sowie für den einzelnen Empfänger,
                    der diesen Datenstrom empfängt.\\
                \hline
            \end{tabular}
            \begin{tabular}{|p{2.5cm}|p{5cm}|p{7.55cm}|}
                \multicolumn{3}{|c|}{\textbf{Einzelschritte des Testfalls}} \\
                \hline
                    \textbf{Schritt} & \textbf{Aktion} & \textbf{Ergebnis}\\
                \hline
                    Test-Setup & Installieren und Starten des MC-TOOL auf zwei
                    Rechnern im Netzwerk & Das Tool wird auf beiden Rechnern
                    ausgeführt und ist bereit zum Starten von Sendern. \\
                \hline
                    Zeit Synchronisierung & Synchronisieren der Systemzeit auf
                    beiden Computern mit dem selben NTP-Server & Die Zeiten
                    beider Computer sind hinreichend synchron.\\
                \hline
                    Sender Starten & Anlegen eines Senders mit Senderate von 100
                    PPS und Aktivieren des Senders & Der Sender sendet und wird
                    angezeigt.\\
                \hline
                    Empfänger anlegen & Anlegen des Empfängers auf dem
                    anderen Computer mit entsprechender Gruppen-Adresse und
                    Port. & Der Empfänger empfängt den Sender aus dem Netzwerk.\\
                \hline
                    Detailansicht Paketraten & Die Detailansicht des Empfängers
                    wird geöffnet.& Die konfigurierte und gemessene Senderate
                    stimmen mit den Daten des Senders überein. Die gemessene
                    Empfangsrate stimmt mit der Tabellenangabe überein.\\
                \hline
                    Detailansicht Übertragungszeit & Messen der Ping-Zeit zum
                    anderen Computer. & Die angezeigte Übetragungszeit stimmt
                    mit der Hälfte der Ping-Zeit überein.\\
                \hline
                    Detailansicht maximaler Versatz & Korrektheit des Wertes
                    kann nicht verifiziert werden. & Maximaler Versatz wird
                    angezeigt.\\
                \hline
            \end{tabular}
        \end{center}
    \end{table}