\chapter{Testsuite /TS20/}
Dieser Anhang enthält die Testfälle der Testsuite /TS20/.\\
Die zugehörige Requirement ID lautet /UC10/.\\
\newline
Die Testfälle dieser Testsuite sollen die korrekte Konfigurationsmöglichkeit der Empfänger über die GUI sicherstellen.\\
Dazu zählen zum Einen die korrekte Übernahme der eingegebenen Werte, die Fehlerbehandlung, sowie die Zusammenarbeit mit dem Rest des Programms während dem Betrieb.
Zur Konfiguration gehört das Anlegen und Löschen bestehender Sender, das Aktivieren und Deaktivieren von Streams, sowie das Bearbeiten von bereits angelegten Streams.

\begin{table}[h]
\caption{/TC2001/}
\label{tab:TC2001}
\begin{center}
\begin{tabular}{|p{3.5cm}|p{12cm}|}
\hline
\textbf{Testfall Id} & /TC2001/\\
\hline
\textbf{Testfall Name} & Anlegen eines Empfängers\\
\hline
\textbf{Requirement ID} & /VA0700/, /VA0800/, /VA0200/\\
\hline
\textbf{Beschreibung} & Dieser Testfall stellt die Funktionalität sicher, mit
der GUI einen korrekt konfigurierten Empfänger erstellen zu können.\\
\hline
\end{tabular}
\begin{tabular}{|p{2.5cm}|p{5cm}|p{7.55cm}|}
\multicolumn{3}{|c|}{\textbf{Einzelschritte des Testfalls}} \\
\hline
\textbf{Schritt} & \textbf{Aktion} & \textbf{Ergebnis}\\
\hline
Programmstart (0)& Das Multicast-Test-Tool auf üblichem Wege starten. & GUI
erscheint.\\
\hline
Dialog öffnen (1) & Den 'Hinzufügen' Knopf im Empfängerbereich des Hauptfensters
betätigen. & Dialog zum Hinzufügen eines neuen Empfängers öffnet sich.\\
\hline
IP-Adfressen-Test (2) & Eine valide Test-Gruppen-Addresse (wie in 'Definitionen'
beschrieben) wählen und eintragen, alle anderen Werte auf Standardzuweisungen
belassen, mit 'OK' betätigen. & Neuer Empfänger auf gewählter Gruppe erscheint im
Hauptfenster im Empfängerbereich.\\
\hline
Alle validen Gruppen prüfen (3) & Wiederhole 1-2 für alle validen Testwerte. & Die Empfänger werden korrekt angelegt.\\
\hline
Dialog öffnen (4) & Den 'Hinzufügen' Knopf im Empfängerbereich des Hauptfensters
betätigen & Dialog zum Hinzufügen eines neuen Empfängers öffnet sich.\\
\hline
Port-Test (5) & Einen validen Test-Port (wie in 'Definitionen'
beschrieben) wählen und eintragen, alle anderen Werte auf Standardzuweisungen
belassen, mit 'OK' betätigen. & Neuer Empfänger auf gewähltem Port erscheint im
Hauptfenster im Empfängerbereich.\\
\hline
Alle validen Ports prüfen (3) & Wiederhole 4-5 für alle validen Testwerte. & Die Empfänger werden korrekt angelegt.\\
\hline
\end{tabular}
\end{center}
\label{default}
\end{table}

\begin{table}[h]
\caption{/TC2002/}
\label{tab:TC2002}
\begin{center}
\begin{tabular}{|p{3.5cm}|p{12cm}|}
\hline
\textbf{Testfall Id} & /TC2002/\\
\hline
\textbf{Testfall Name} & Aktivieren von Empfängern\\
\hline
\textbf{Requirement ID} & /VA0700/, /VA0800/, /VA0200/\\
\hline
\textbf{Beschreibung} & Dieser Testfall stellt die Funktionalität sicher, mit
der GUI bestehende Empfänger aktivieren zu können.\\
\hline
\end{tabular}
\begin{tabular}{|p{2.5cm}|p{5cm}|p{7.55cm}|}
\multicolumn{3}{|c|}{\textbf{Einzelschritte des Testfalls}} \\
\hline
\textbf{Schritt} & \textbf{Aktion} & \textbf{Ergebnis}\\
\hline
TC2001 (1)& Alle Schritte aus TC2001 wiederholen. & Hauptfenster mit angelegten,
deaktivierten Empfänger befindet sich auf dem Bildschirm.\\
\hline
Einzelne Aktivierung (2) & Den obersten Empfänger markieren, 'Aktivieren'-Knopf
des Empfängerbereichs betätigen. & Empfänger wird aktiv, erkennbar am
Wechsel des roten Vierecks zu einem grünen Pfeil.\\
\hline
Multiple Aktivierung (3) & Die restlichen Empfänger markieren,
'Aktivieren'-Knopf des Empfängerbereichs betätigen. & Empfänger werden aktiv,
erkennbar am Wechsel des roten Vierecks zu einem grünen Pfeil.\\
\hline
\end{tabular}
\end{center}
\label{default}
\end{table}

\begin{table}[h]
\caption{/TC2003/}
\label{tab:TC2003}
\begin{center}
\begin{tabular}{|p{3.5cm}|p{12cm}|}
\hline
\textbf{Testfall Id} & /TC2003/\\
\hline
\textbf{Testfall Name} & Bearbeiten von Empfängern\\
\hline
\textbf{Requirement ID} & /VA0700/, /VA0800/, /VA0200/\\
\hline
\textbf{Beschreibung} & Dieser Testfall stellt die Funktionalität sicher, mit
der GUI bestehende Empfänger bearbeiten zu können.\\
\hline
\end{tabular}
\begin{tabular}{|p{2.5cm}|p{5cm}|p{7.55cm}|}
\multicolumn{3}{|c|}{\textbf{Einzelschritte des Testfalls}} \\
\hline
\textbf{Schritt} & \textbf{Aktion} & \textbf{Ergebnis}\\
\hline
TC2001 (1)& Alle Schritte aus TC2001 wiederholen. & Hauptfenster mit angelegten,
deaktivierten Empfänger befindet sich auf dem Bildschirm.\\
\hline
Öffnen des Bearbeiten-Dialogs (2) & Einen der inaktiven Empfänger markieren,
'Bearbeiten'-Schaltfläche betätigen. & Bearbeiten Dialog erscheint.\\
\hline
Bearbeiten eines inaktiven Empfängers (3) & Analyse-Verhalten auf 'lazy' ändern,
'OK' betätigen. & Fenster schließt sich.\\
\hline
Öffnen des Bearbeiten-Dialogs & Einen der inaktiven Empfänger markieren,
'Bearbeiten'-Schaltfläche betätigen. & Bearbeiten Dialog erscheint.\\
\hline
Änderung überprüfen (4) & & Analyse Verhalten ist auf 'lazy' eingestellt.\\
\hline
Öffnen des Bearbeiten-Dialogs (5) & Nun einen der aktiven Empfänger markieren,
'Bearbeiten'-Schaltfläche betätigen. & Bearbeiten Dialog erscheint.\\
\hline
Bearbeiten eines aktiven Empfängers (6) & Analyse-Verhalten auf 'eager' ändern,
'OK' betätigen. & Fenster schliesst sich.\\
\hline
Wiederhole (5) & Bearbeiten Fenster erneut öffnen. & Bearbeiten Dialog erscheint.\\
\hline
Überprüfen (7) & & Analyse Verhalten ist 'eager'.\\
\hline
\end{tabular}
\end{center}
\label{default}
\end{table}

\begin{table}[h]
\caption{/TC2004/}
\label{tab:TC2004}
\begin{center}
\begin{tabular}{|p{3.5cm}|p{12cm}|}
\hline
\textbf{Testfall Id} & /TC2004/\\
\hline
\textbf{Testfall Name} & Deaktivieren von Empfängern\\
\hline
\textbf{Requirement ID} & /VA0700/, /VA0800/, /VA0200/\\
\hline
\textbf{Beschreibung} & Dieser Testfall stellt die Funktionalität sicher, mit
der GUI bestehende Empfänger deaktivieren zu können.\\
\hline
\end{tabular}
\begin{tabular}{|p{2.5cm}|p{5cm}|p{7.55cm}|}
\multicolumn{3}{|c|}{\textbf{Einzelschritte des Testfalls}} \\
\hline
\textbf{Schritt} & \textbf{Aktion} & \textbf{Ergebnis}\\
\hline
TC2002& Alle Schritte aus TC2002 wiederholen. & Hauptfenster mit angelegten,
aktiven Empfängern befindet sich auf dem Bildschirm.\\
\hline
Einzelne Deaktivierung (1) & Den obersten Empfänger markieren,
'Deaktivieren'-Knopf des Empfängerbereichs betätigen. & Empfänger wird inaktiv,
erkennbar am Wechsel des grünen Pfeils zu einem roten Viereck.\\
\hline
Multiple Deaktivierung (2) & Die restlichen Empfänger markieren,
'Deaktivieren'-Knopf des Empfängerbereichs betätigen. & Empfänger werden
inaktiv, erkennbar am Wechsel des grünen Pfeils zu einem roten Viereck.\\
\hline
\end{tabular}
\end{center}
\label{default}
\end{table}

\begin{table}[h]
\caption{/TC2005/}
\label{tab:TC2005}
\begin{center}
\begin{tabular}{|p{3.5cm}|p{12cm}|}
\hline
\textbf{Testfall Id} & /TC2005/\\
\hline
\textbf{Testfall Name} & Entfernen von Empfängern\\
\hline
\textbf{Requirement ID} & /VA0700/, /VA0800/,/VA0200/\\
\hline
\textbf{Beschreibung} & Dieser Testfall stellt die Funktionalität sicher, mit
der GUI bestehende Empfänger entfernen zu können.\\
\hline
\end{tabular}
\begin{tabular}{|p{2.5cm}|p{5cm}|p{7.55cm}|}
\multicolumn{3}{|c|}{\textbf{Einzelschritte des Testfalls}} \\
\hline
\textbf{Schritt} & \textbf{Aktion} & \textbf{Ergebnis}\\
\hline
TC2001& Alle Schritte aus TC2001 wiederholen. & Hauptfenster mit angelegten,
inaktiven Empfängern befindet sich auf dem Bildschirm.\\
\hline
Einzelnes Entfernen (1) & Den obersten Empfänger markieren,
'Entfernen'-Knopf des Empfängerbereichs betätigen. & Empfänger ist nicht mehr
in der Liste vorhanden.\\
\hline
Multiples Entfernen (2) & Die restlichen Empfänger markieren,
'Entfernen'-Knopf des Empfängerbereichs betätigen. & Empfänger sind nicht mehr
in der Liste vorhanden.\\
\hline
\end{tabular}
\end{center}
\label{default}
\end{table}

\begin{table}[h]
\caption{/TC2006/}
\label{tab:TC2006}
\begin{center}
\begin{tabular}{|p{3.5cm}|p{12cm}|}
\hline
\textbf{Testfall Id} & /TC2006/\\
\hline
\textbf{Testfall Name} & Abfangen invalider Angaben\\
\hline
\textbf{Requirement ID} & /VA0700/, /VA0800/,/VA0200/\\
\hline
\textbf{Beschreibung} & Dieser Testfall stellt die Funktionalität sicher, mit
der GUI falsche Wertangaben abfangen zu können.\\
\hline
\end{tabular}
\begin{tabular}{|p{2.5cm}|p{5cm}|p{7.55cm}|}
\multicolumn{3}{|c|}{\textbf{Einzelschritte des Testfalls}} \\
\hline
\textbf{Schritt} & \textbf{Aktion} & \textbf{Ergebnis}\\
\hline
Programmstart (0)& Das Multicast-Test-Tool auf üblichem Wege starten. & GUI
erscheint.\\
\hline
Dialog öffnen (1) & Den 'Hinzufügen' Knopf im Empfängerbereich des Hauptfensters
betätigen. & Dialog zum Hinzufügen eines neuen Empfängers öffnet sich.\\
\hline
IP-Adressen Test (2) & Eine invalide Test-Gruppen-Addresse (wie in
'Definitionen' beschrieben) wählen und eintragen, alle anderen Werte auf Standardzuweisungen
belassen, mit 'OK' betätigen. & Dialog bleibt offen und Fehlermeldung über falsche
Gruppen-Adresse wird angezeigt.\\
\hline
Alle validen Gruppen prüfen (3) & Wiederhole 1-2 für alle invaliden Testwerte. & Dialog bleibt offen und Fehlermeldung über falsche
Gruppen-Adresse wird angezeigt.\\
\hline
Dialog öffnen (4) & Den 'Hinzufügen' Knopf im Empfängerbereich des Hauptfensters
betätigen. & Dialog zum Hinzufügen eines neuen Empfängers öffnet sich.\\
\hline
Port-Test (5) & Einen invaliden Test-Port (wie in 'Definitionen'
beschrieben) wählen und eintragen, alle anderen Werte auf Standardzuweisungen
belassen, mit 'OK' betätigen. & Dialog bleibt offen und Fehlermeldung über falschen
Port erscheint.\\
\hline
Alle invaliden Ports prüfen (3) & Wiederhole 4-5 für alle invaliden Testwerte. & Dialog bleibt offen und Fehlermeldung über falschen
Port erscheint.\\
\hline
\end{tabular}
\end{center}
\label{default}
\end{table}