\chapter{Testsuite /TS50/}

    Dieser Anhang enthält die Testfälle der Testsuite /TS50/.\\
    Die zugehörige Requirement ID lautet /UC30/.\\
    \newline
    Die Testfälle dieser Testsuite sollen die korrekte Übermittlung von
    Multicast-Paketen sicherstellen. Dazu zählen zum Einen die fehlerfreie
    Übermittlung der Pakete (ohne Störquellen) und die korrekte Übertragung der
    in den Paketen enthaltenen Nutzdaten (Payload).\\
    \newline
    Alle Tests dieser Suite arbeiten mit mindestens zwei Instanzen des
    MC-Test-Tools. Dadurch werden Verfälschungen des Ergebnisses durch eine
    innerprogrammliche Kommunikation von Sender- und Empfängerseite
    ausgeschlossen. Bei Verwendung eines physischen Netzwerks
    zwischen Sender und Empfänger ist dessen Multicastfähigkeit im Vorfeld zu
    verifizieren (z.B. durch das Hirschmann-Tool). Sie wird in den folgenden
    Testfällen als gegeben vorausgesetzt.\\
    \newline
    Die folgenden Testfälle verwenden die Formulierung "`im selben Netzwerk"'
    und "`der gewählte Netzwerkadapter"'. Damit ist gemeint:
    \begin{enumerate}[a)]
        \item in der selben Instanz eines Betriebssystems. Der gewählte
              Netzwerkadapter ist das Loopback-Interface.
        \item in verschiedenen Betriebssysteminstanzen, die aber per
              Virtualisierung auf der selben physischen Maschine laufen. Die
              Verbindung wird über ein ebenfalls virtualisiertes Netzwerk
              hergestellt. Bei Einsatz von Virtualbox empfiehlt sich z.B. der
              Einsatz von entweder "`internal networking"' oder "`Host-only
              networking"'. Der gewählte Netzwerkadapter ist der
              Host-only-Adapter (Host) bzw. der Ethernet-Adapter (Gast).
        \item in verschiedenen, physisch getrennten Betriebssysteminstanzen. Die
              Verbindung wird per Ethernet hergestellt. Die Verbindung darf
              keine Router-Hops voraussetzen. Der gewählte Netzwerkadapter ist
              jeweils der Ethernet-Adapter, der mit dem genutzten Netzwerk
              verbunden ist.
    \end{enumerate}
    Alle Tesfälle, die mit mehreren Instanzen des Multicast Test Tools arbeiten,
    sind mit Option a) durchzuführen. Zusätzlich sind alle diese Testfälle
    entweder mit Option b) oder c) durchzuführen.

    \begin{table}[h]
        \caption{/TC5001/}
        \label{tab:TC5001}
        \begin{center}
            \begin{tabular}{|p{3.5cm}|p{12cm}|}
                \hline
                    \textbf{Testfall Id} & /TC5001/\\
                \hline
                    \textbf{Testfall Name} & Multicast-Pakete senden\\
                \hline
                    \textbf{Requirement ID} & /VA0100/\\
                \hline
                    \textbf{Beschreibung} & Dieser Test verifiziert, dass das
                    Tool in der Lage ist, UDP-Pakete an eine
                    IP-Multicast-Gruppe adressiert zu verschicken.\\
                \hline
            \end{tabular}
            \begin{tabular}{|p{3.5cm}|p{5cm}|p{6.55cm}|}
                \multicolumn{3}{|c|}{\textbf{Einzelschritte des Testfalls}} \\
                \hline
                    \textbf{Schritt} & \textbf{Aktion} & \textbf{Ergebnis}\\
                \hline
                    Test-Setup 1 &
                    Das Multicast-Test-Tool mit grafischer Oberfläche starten & 
                    Das Tool ist bereit für das Anlegen von Datenströmen \\
                \hline
                    Test-Setup 2 &
                    Wireshark in derselben Betriebssystemsinstanz starten &
                    Wireshark ist bereit für Benutzereingaben \\
                \hline
                    Wireshark aktivieren & Wireshark an einem Ethernet-Adapter
                    lauschen lassen & 
                    Wireshark zeigt diversen Netzwerkverkehr am gewählten
                    Adapter an\\
                \hline
                    Multicast-Sender aktivieren &
                    Im Multicast Test Tool einen Sender am für Wireshark
                    gewählten Netzwerk-Adapter, aber ansonsten mit
                    Standardwerten, anlegen und aktivieren &
                    Der neu erstellte Datenstrom erscheint in der Liste, wird
                    als aktiviert und mit einer Senderate größer null
                    angezeigt\\
                \hline
                    Wireshark-Output überprüfen &
                    Den von Wireshark aufgezeichneten Netzwerkverkehr auf
                    korrekt adressierte Multicastpakete überprüfen &
                    Es werden Pakete registriert, die an die im Sender
                    angegebene Multicast-Gruppe adressiert sind. Die Pakete
                    sind anhand des 1337-/x0539-Headers im Datenblock als
                    SPAM-Pakete identifizierbar.\\
                \hline
            \end{tabular}
        \end{center}
    \end{table}

    \begin{table}[h]
        \caption{/TC5002/}
        \label{tab:TC5002}
        \begin{center}
            \begin{tabular}{|p{3.5cm}|p{12cm}|}
                \hline
                    \textbf{Testfall Id} & /TC5002/\\
                \hline
                    \textbf{Testfall Name} & Einfacher Datenstrom mit
                                             Nutzdaten und Jumbo-Paketen\\
                \hline
                    \textbf{Requirement ID} & /VA0100/, /VA0200/, /VA0500/,
                    /OA0400/\\
                \hline
                    \textbf{Beschreibung} & Dieser Test verifiziert, dass ein
                                            einzelner, von einem Sender
                                            verschickter Datenstrom, bei einem
                                            Empfänger ankommt. Außerdem wird
                                            überprüft, dass im Sender
                                            konfigurierte Nutzdaten
                                            unverfälscht beim Empfänger
                                            ankommen und dass auch
                                            Jumbo-Pakete mit einer Größe von
                                            9000 Byte versendet und empfangen
                                            werden können.\\
                \hline
            \end{tabular}
            \begin{tabular}{|p{3.5cm}|p{5cm}|p{6.55cm}|}
                \multicolumn{3}{|c|}{\textbf{Einzelschritte des Testfalls}} \\
                \hline
                    \textbf{Schritt} & \textbf{Aktion} & \textbf{Ergebnis}\\
                \hline
                    Test-Setup &
                    Zwei Instanzen des Multicast Test Tools im selben
                    Netzwerk starten &
                    Das Tool wird zwei Mal ausgeführt und ist jeweils bereit zum
                    Anlegen von Sendern und Empfängern.\\
                \hline
                    Sender starten &
                    In der sendenden Instanz einen Sender mit Standardwerten am
                    gewählten Netzwerkadapter und mit den Nutzdaten
                    "`Hallo Welt" anlegen und starten & Der Sender erscheint in
                    der Liste und wird als aktiviert und mit einer Senderate
                    größer null angezeigt.\\
                \hline
                    Empfänger starten &
                    In der empfangenden Instanz einen Empfänger mit
                    Standardwerten am gewählten Netzwerkadapter anlegen und
                    starten &
                    Der Empfänger erscheint in der Liste und wird als
                    aktiviert angezeigt.\\
                \hline
                    Datenstrom identifizieren &
                    In der Empfängerliste den vom Sender gesendeten Datenstrom
                    identifizieren & 
                    Ein Datenstrom, der dieselbe Sender ID, wie der vorher
                    gestartete Sender aufweist, erscheint als Unterpunkt des
                    erstellten Empfängers.\\
                \hline
                    Nutzdaten überprüfen &
                    Die in der Empfängerliste angezeigten Nutzdaten mit den im
                    Sender eingestellten vergleichen &
                    In der empfangenden Instanz werden dieselben Nutzdaten, die
                    beim Anlegen des Senders angegeben wurden ("`Hallo Welt"')
                    beim entsprechenden Datenstrom angezeigt.\\
                \hline
                    Jumbo-Pakete versenden &
                    In der senden Instanz die Paketgröße des Senders auf den
                    Maximalwert erhöhen &
                    Der Datenstrom wird weiterhin mit den korrekten Nutzdaten
                    beim Empfänger angezeigt.\\
                \hline
            \end{tabular}
        \end{center}
    \end{table}

    \begin{table}[h]
        \caption{/TC5003/}
        \label{tab:TC5003}
        \begin{center}
            \begin{tabular}{|p{3.5cm}|p{12cm}|}
                \hline
                    \textbf{Testfall Id} & /TC5003/\\
                \hline
                    \textbf{Testfall Name} & Datenstrom an mehrere Empfänger\\
                \hline
                    \textbf{Requirement ID} & /VA0100/, /VA0200/\\
                \hline
                    \textbf{Beschreibung} & Dieser Test verifiziert, dass das
                    Tool echte Multicasts verschickt, in dem Sinne, dass sie
                    von mehreren Empfängern empfangen werden können.\\
                \hline
            \end{tabular}
            \begin{tabular}{|p{3.5cm}|p{5cm}|p{6.55cm}|}
                \multicolumn{3}{|c|}{\textbf{Einzelschritte des Testfalls}} \\
                \hline
                    \textbf{Schritt} & \textbf{Aktion} & \textbf{Ergebnis}\\
                \hline
                    Test-Setup 1 &
                    Eine Instanz des Mutlicast Test Tools
                    starten, die als Sender fungieren soll &
                    Die sendende Instanz ist bereit für Benutzereingaben. \\
                \hline
                    Test-Setup 2 & 
                    Mindestens zwei weitere Instanzen des Multicast Test Tools
                    im selben Netzwerk starten, die als Empfänger fungieren
                    sollen &
                    Die empfangenden Instanzen sind bereit für
                    Benutzereingaben.\\
                \hline
                    Sender anlegen &
                    In der sendenden Instanz einen Sender mit Standardwerten am
                    gewählten Netzwerk-Adapter anlegen &
                    Ein neuer Sender erscheint in der Liste.\\
                \hline
                    Empfänger anlegen &
                    In jeder der empfangenden Instanzen jeweils einen Empfänger
                    mit Standardwerten am gewählten Netzwerkadapter anlegen &
                    In jeder empfangenden Instanz erscheint ein Empfänger.\\
                \hline
                    Empfänger starten &
                    In jeder der empfangenden Instanzen den zuvor angelegten
                    Empfänger aktivieren &
                    Die Empfänger werden als aktiv in der Liste angezeigt.\\
                \hline
                    Sender starten &
                    In der sendenden Instanz den zuvor angelegen Sender
                    aktivieren &
                    Der Sender wird als aktiv und mit einer Senderate größer
                    null angezeigt.\\
                \hline
                    Empfänger-Output prüfen &
                    Die Listen der empfangenden Instanzen auf eingehende
                    Datenströme überprüfen &
                    Jede der empfangenden Instanzen zeigt als Unterpunkt der
                    zuvor angelegten Empfänger einen Datenstrom mit der Sender ID
                    des zuvor angelegten Senders an.\\
                \hline
            \end{tabular}
        \end{center}
    \end{table}

    \begin{table}[h]
        \caption{/TC5004/}
        \label{tab:TC5004}
        \begin{center}
            \begin{tabular}{|p{3.5cm}|p{12cm}|}
                \hline
                    \textbf{Testfall Id} & /TC5004/\\
                \hline
                    \textbf{Testfall Name} & Lasttest\\
                \hline
                    \textbf{Requirement ID} & /VA0100/, /VA0200/, /VA0300/\\
                \hline
                    \textbf{Beschreibung} & Dieser Test verifiziert, dass das
                    Tool mindestens 30 Sendeströme gleichzeitig senden, bzw.
                    empfangen kann. Außerdem wird geprüft, dass das Tool auch
                    unter diesen Bedingungen noch ohne Einschränkungen
                    bedienbar ist.\\
                \hline
            \end{tabular}
            \begin{tabular}{|p{3.5cm}|p{5cm}|p{6.55cm}|}
                \multicolumn{3}{|c|}{\textbf{Einzelschritte des Testfalls}} \\
                \hline
                    \textbf{Schritt} & \textbf{Aktion} & \textbf{Ergebnis}\\
                \hline
                    Test-Setup & Zwei Instanzen des Multicast Testtools im
                    selben Netzwerk starten &
                    Beide Instanzen sind bereit für Benutzereingaben.\\
                \hline
                    Sender starten &
                    In der sendenden Instanz 30 Sender mit Standardwerten am
                    gewählten Netzwerkadapter anlegen und aktivieren &
                    Alle 30 Ströme werden als aktiv und mit einer
                    eingestellten und gemessenen Senderate von jeweils 10
                    Pakten pro Sekunde angezeigt.\\
                \hline
                    Empfänger starten &
                    In der empfangenden Instanz 30 Empfänger mit Standardwerten
                    am gewählten Netzwerkadatper anlegen und aktivieren &
                    Der erstellte Empfänger erscheint als aktiv in der Liste.
                    Ihm untergeordnet erscheinen 30 Sendeströme mit einer
                    gemessenen Paketrate von jeweils 10 Paketen pro Sekunde.\\
                \hline
                    Senderperformance prüfen &
                    Responsiveness der grafischen Oberfläche der sendenden
                    Instanz überprüfen &
                    Die grafische Oberfläche reagiert weiterhin ohne merkliche
                    Verzögerung auf Benutzereingaben.\\
                \hline
                    Empfängerperformance prüfen &
                    Responsiveness der grafischen Oberfläche der empfangenden
                    Instanz überprüfen &
                    Die grafische Oberfläche reagiert weiterhin ohne merkliche
                    Verzögerung auf Benutzereingaben.\\
                \hline
            \end{tabular}
        \end{center}
    \end{table}

    \begin{table}[h]
        \caption{/TC5005/}
        \label{tab:TC5005}
        \begin{center}
            \begin{tabular}{|p{3.5cm}|p{12cm}|}
                \hline
                    \textbf{Testfall Id} & /TC5005/\\
                \hline
                    \textbf{Testfall Name} & Empfang von Hirschmann-Paketen\\
                \hline
                    \textbf{Requirement ID} & /VA0200/, /VA0400/\\
                \hline
                    \textbf{Beschreibung} & Dieser Test verifiziert die
                    Fähigkeit des Tools, Pakete des Hirschmann Tools zu
                    empfangen und zu verarbeiten.\\
                \hline
            \end{tabular}
            \begin{tabular}{|p{3.5cm}|p{5cm}|p{6.55cm}|}
                \multicolumn{3}{|c|}{\textbf{Einzelschritte des Testfalls}} \\
                \hline
                    \textbf{Schritt} & \textbf{Aktion} & \textbf{Ergebnis}\\
                \hline
                    Test-Setup 1 &
                    Das Multicast Test Tool mit grafischer Oberfläche starten & 
                    Das Tool ist bereit für Benutzereingaben\\
                \hline
                    Test-Setup 2 &
                    Das Hirschmann Tool im selben Netzwerk starten &
                    Das Hirschmann Tool ist bereit für Benutzereingaben. \\
                \hline
                    Empfänger starten &
                    Im Multicast Test Tool einen Empfänger mit Standarddaten am
                    gewählten Netzwerk-Adapter anlegen und aktivieren &
                    Der angelegte Empfänger wird in der Liste als aktiv
                    angezeigt.\\
                \hline
                    Sender starten &
                    Im Hirschmann Tool einen Sender an die Gruppe "`225.1.1.1"'
                    und den Port "`12345"' am gewählten Netzwerkadapter anlegen
                    und aktivieren. &
                    Der Sender wird als aktiv in der Liste angezeigt.\\
                \hline
                    Empfang verifizieren &
                    Im Multicast Test Tool den Empfang des Datenstroms
                    verifizieren &
                    Der Datenstrom des Hirschmann Tools erscheint als
                    Unterpunkt des zuvor angelegten Empfängers. Die
                    gemessene Datenrate entspricht weitgehend der im Hirschmann
                    Tool angegebenen.\\
                \hline
            \end{tabular}
        \end{center}
    \end{table}

    \begin{table}[h]
        \caption{/TC5006/}
        \label{tab:TC5006}
        \begin{center}
            \begin{tabular}{|p{3.5cm}|p{12cm}|}
                \hline
                    \textbf{Testfall Id} & /TC5006/\\
                \hline
                    \textbf{Testfall Name} & Abonnierte Gruppen empfangen
                    \\
                \hline
                    \textbf{Requirement ID} & /VA0100/, /VA0200/\\
                \hline
                    \textbf{Beschreibung} &  Dieser Test verifiziert die
                    korrekte Umsetzung des Multicast-Protokolls. Er überprüft
                    den Gruppen-Abonnement-Mechanismus und stellt sicher, dass
                    tatsächlich nur die Abonnierten Gruppen empfangen und
                    ausgewertet werden.\\
                \hline
            \end{tabular}
            \begin{tabular}{|p{3.5cm}|p{5cm}|p{6.55cm}|}
                \multicolumn{3}{|c|}{\textbf{Einzelschritte des Testfalls}} \\
                \hline
                    \textbf{Schritt} & \textbf{Aktion} & \textbf{Ergebnis}\\
                \hline
                    Test-Setup 1 &
                    Zwei Instanzen des Multicast Test Tools mit grafischer
                    Oberfläche starten &
                    Die Tools sind bereit für Benutzereingaben.\\
                \hline
                    Sender erzeugen &
                    In der sendenden Instanz drei Sender mit aufsteigenden
                    Gruppen- und Portnummern (225.1.1.1-225.1.1.3, 12345-12347)
                    anlegen und aktivieren &
                    Die drei Datenströme werden als aktiv und mit einer
                    Senderate größer null in der Liste angezeigt.\\
                \hline
                    Empfänger anlegen &
                    In der empfangenden Instanz einen Empfänger für die Gruppe
                    225.1.1.1 und den Port 12345, sowie einen für die Gruppe
                    225.1.1.2 und den Port 12346 anlegen &
                    Beide Empfänger werden als aktiv in der Liste angezeigt.\\
                \hline
                    Empfangende Datenströme verifizieren &
                    Die tatsächlich empfangenden Datenströme identifizieren & 
                    Es werden nur die Datenströme in Gruppe 225.1.1.1 und
                    225.1.1.2 empfangen und angezeigt. Gruppe 225.1.1.3 wird
                    nicht empfangen.\\
                \hline
            \end{tabular}
        \end{center}
    \end{table}