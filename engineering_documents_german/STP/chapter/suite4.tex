\chapter{Testsuite /TS40/}
Dieser Anhang enthält die Testfälle der Testsuite /TS40/.\\
Die zugehörige Requirement ID lautet /UC20/.\\
\newline
Die Testfälle dieser Testsuite sollen die Automatisierung des Programms über die CLI sicherstellen.\\
Dazu zählen zum einen das Laden von Profilen anhand eines Pfades oder dem Namen eines zuletzt verwendeten Profils und der Automatisierte Start aller geladenen Sender und Empfänger.
Zusätzlich soll das Programm zum einen mit CLI und ohne GUI oder mit CLI und GUI startbar sein.
Das Testen der Ausgabe von Logging Daten auf der CLI und in eine Datei ist implizit gefordert.

\begin{table}[h]
\caption{/TC4001/}
\label{tab:TC4001}
\begin{center}
\begin{tabular}{|p{3.5cm}|p{9cm}|}
\hline
\textbf{Testfall Id} & /TC4001/\\
\hline
\textbf{Testfall Name} & CLI-Konfiguration\\
\hline
\textbf{Requirement ID} & VA0900 \& VA0600\\
\hline
\textbf{Beschreibung} & Dieser Testfall verifiziert die korrekte
Funktionsweise der "`Konfiguration Laden"-Funktion des Commandline Interfaces.\\
\hline
\end{tabular}
\begin{tabular}{|p{2.5cm}|p{5cm}|p{4.55cm}|}
\multicolumn{3}{|c|}{\textbf{Einzelschritte des Testfalls}} \\
\hline
\textbf{Schritt} & \textbf{Aktion} & \textbf{Ergebnis}\\
\hline
Test-Setup & Installiere das MC-Tool &  MC-Tool ist per
Commandline aufrufbar\\
\hline
Starten und Konfiguration laden - korrekte Pfadangabe & Eingabe des Befehls zum
Starten und Laden einer Konfiguration mit korrekter Pfadangabe zu einer
Konfigurationsdatei & Anlegen aller Sender und Empfänger im selben Zustand, wie
zum Zeitpunkt des Speicherns \& Ausgabe auf der Commandline, die den Erfolg des
Ladens bestätigt.
\\
\hline
Konfiguration laden - fehlerhafte Pfadangabe & Eingabe des Befehls zum
Starten und Laden einer Konfiguration mit fehlerhafter Pfadangabe zu einer
Konfigurationsdatei & Ausgabe einer Fehlermeldung, dass keine Datei gefunden wurde\\
\hline
Konfiguration laden - fehlerhafte Datei & Eingabe des Befehls zum Starten
und Laden einer Konfiguration mit Pfadangabe zu einer fehlerhaften/nicht
kompatiblen Datei & Ausgabe einer Fehlermeldung, dass die gelesene Datei fehlerhaft ist\\
\hline
\end{tabular}
\end{center}
\label{default}
\end{table}

\begin{table}[h]
\caption{/TC4002/}
\label{tab:TC4002}
\begin{center}
\begin{tabular}{|p{3.5cm}|p{9cm}|}
\hline
\textbf{Testfall Id} & /TC4002/\\
\hline
\textbf{Testfall Name} & Starten aller Sender und Empfänger per CLI\\
\hline
\textbf{Requirement ID} & VA0900 \& VA0600\\
\hline
\textbf{Beschreibung} & Dieser Testfall verifiziert die korrekte
Funktionsweise des Startens aller Sender und Empfänger via CLI.\\
\hline
\end{tabular}
\begin{tabular}{|p{2.5cm}|p{5cm}|p{4.55cm}|}
\multicolumn{3}{|c|}{\textbf{Einzelschritte des Testfalls}} \\
\hline
\textbf{Schritt} & \textbf{Aktion} & \textbf{Ergebnis}\\
\hline
Test-Setup & Installiere das MC-Tool & MC-Tool ist via
Commandline startbar\\
\hline
Eingabe des gültigen Start-Befehls & Eingabe des Befehls zum Starten des
Programms, sowie zum Laden und Starten aller Sender Empfänger & Ausgabe auf der
Commandline, die den Erfolg des Startens bestätigt
\\
\hline
Eingabe des ungültigen Start-Befehls & Eingabe eines ungültigen Befehls zum
Starten des Programms, sowie zum Laden und Starten aller Empfänger & Ausgabe auf
CLI, die die ungültige Eingabe quittiert
\\
\hline
\end{tabular}
\end{center}
\label{default}
\end{table}

\begin{table}[h]
\caption{/TC4003/}
\label{tab:TC4003}
\begin{center}
\begin{tabular}{|p{3.5cm}|p{9cm}|}
\hline
\textbf{Testfall Id} & /TC4003/\\
\hline
\textbf{Testfall Name} & Starten des Programms ohne Starten der Sender und
Empfänger per CLI\\
\hline
\textbf{Requirement ID} & VA0900 \& VA0600\\
\hline
\textbf{Beschreibung} & Dieser Testfall verifiziert die korrekte
Funktionsweise des Startens des Programms ohne Starten der Sender und
Empfänger via CLI.\\
\hline
\end{tabular}
\begin{tabular}{|p{2.5cm}|p{5cm}|p{4.55cm}|}
\multicolumn{3}{|c|}{\textbf{Einzelschritte des Testfalls}} \\
\hline
\textbf{Schritt} & \textbf{Aktion} & \textbf{Ergebnis}\\
\hline
Test-Setup & Installiere das MC-Tool & MC-Tool ist via
Commandline startbar\\
\hline
Eingabe des gültigen Start-Befehls & Eingabe des Befehls zum Starten des
Programms, sowie zum Laden und Starten keiner Sender und Empfänger & Ausgabe auf
der Commandline, die den Erfolg des Startens des Programms bestätigt
\\
\hline
Eingabe des ungültigen Start-Befehls & Eingabe eines ungültigen Befehls zum
Starten des Programms, sowie zum Laden und Starten keiner Sender und Empfänger &
Ausgabe auf CLI, die die ungültige Eingabe quittiert
\\
\hline
\end{tabular}
\end{center}
\label{default}
\end{table}

\begin{table}[h]
\caption{/TC4004/}
\label{tab:TC4004}
\begin{center}
\begin{tabular}{|p{3.5cm}|p{9cm}|}
\hline
\textbf{Testfall Id} & /TC4004/\\
\hline
\textbf{Testfall Name} & Ausgabe der Daten im CLI\\
\hline
\textbf{Requirement ID} & VA0900 \& VA0600\\
\hline
\textbf{Beschreibung} & Dieser Testfall verifiziert die korrekte
Funktionsweise der Datenausgabe auf dem Commandline Interface.\\
\hline
\end{tabular}
\begin{tabular}{|p{2.5cm}|p{5cm}|p{4.55cm}|}
\multicolumn{3}{|c|}{\textbf{Einzelschritte des Testfalls}} \\
\hline
\textbf{Schritt} & \textbf{Aktion} & \textbf{Ergebnis}\\
\hline
Test-Setup & Installiere das MC-Tool & MC-Tool ist via
Commandline startbar\\
\hline
Test-Setup Teil 2 & Starten des MC-Tools mit gültigem Pfad und Start-Befehl &
Bestätigung auf CLI\\
\hline
Beobachten der Ausgabe & Auf Update der Ausgabe warten & Ausgabe der Statistik
auf dem CLI
\\
\hline
\end{tabular}
\end{center}
\label{default}
\end{table}

\begin{table}[h]
\caption{/TC4005/}
\label{tab:TC4005}
\begin{center}
\begin{tabular}{|p{3.5cm}|p{9cm}|}
\hline
\textbf{Testfall Id} & /TC4005/\\
\hline
\textbf{Testfall Name} & Loggen der Statistik\\
\hline
\textbf{Requirement ID} & VA0900 \& VA0600\\
\hline
\textbf{Beschreibung} & Dieser Testfall verifiziert die korrekte
Funktionsweise der Logger-Funktionalität.\\
\hline
\end{tabular}
\begin{tabular}{|p{2.5cm}|p{5cm}|p{4.55cm}|}
\multicolumn{3}{|c|}{\textbf{Einzelschritte des Testfalls}} \\
\hline
\textbf{Schritt} & \textbf{Aktion} & \textbf{Ergebnis}\\
\hline
Test-Setup & Installiere das MC-Tool & MC-Tool Commandline Interface mit Bestätigung wird angezeigt\\
\hline
Test-Setup Teil 2 & Starten des MC-Tool via Commandline mit gültigem
Start-Befehl und gestarteten Sendern/Receivern & Bestätigung in Commandline
Interface\\
\hline
Korrekte Log-Daten & Öffnen der Log-Datei und Vergleich mit Ausgabe auf dem
CLI & Logger funktioniert fehlerfrei - geloggte Daten sind korrekt
\\
\hline
Inkorrekte Log-Daten & Öffnen der Log-Datei und Vergleich mit Ausgabe auf dem
CLI & Logger funktioniert fehlerhaft - geloggte Daten stimmen nicht mit dem CLI
überein
\\
\hline
Inkorrektes Log-Format & Öffnen der Log-Datei und betrachten des Log-Formats &
Daten werden nicht als XML gespeichert.
\\
\hline
\end{tabular}
\end{center}
\label{default}
\end{table}

\begin{table}[h]
\caption{/TC4006/}
\label{tab:TC4006}
\begin{center}
\begin{tabular}{|p{3.5cm}|p{9cm}|}
\hline
\textbf{Testfall Id} & /TC4006/\\
\hline
\textbf{Testfall Name} & Starten des MC-Tools ohne GUI\\
\hline
\textbf{Requirement ID} & VA0900 \& VA0600\\
\hline
\textbf{Beschreibung} & Dieser Testfall verifiziert die korrekte
Funktionsweise des Startens des MC-Tools ohne GUI.\\
\hline
\end{tabular}
\begin{tabular}{|p{2.5cm}|p{5cm}|p{4.55cm}|}
\multicolumn{3}{|c|}{\textbf{Einzelschritte des Testfalls}} \\
\hline
\textbf{Schritt} & \textbf{Aktion} & \textbf{Ergebnis}\\
\hline
Test-Setup & Installiere das MC-Tool & MC-Tool Commandline Interface mit Bestätigung wird angezeigt\\
\hline
Starten des MC-Tools 2 & Starten des MC-Tool via Commandline mit gültigem
Start-Befehl und -nogui Parameter & Programm wird ohne GUI gestartet.\\
\hline
\end{tabular}
\end{center}
\label{default}
\end{table}