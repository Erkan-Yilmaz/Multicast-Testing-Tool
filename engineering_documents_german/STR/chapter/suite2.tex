\chapter{Testsuite /TS20/}
Dieses Kapitel enthält die Testergebnisse der Testsuite /TS20/.\\
Die zugehörige Requirement ID lautet /UC10/.

\begin{table}[h]
\caption{/TC2001/}
\label{tab:TC2001}
\begin{center}
\begin{tabular}{|p{3.5cm}|p{11cm}|}
\hline
\textbf{Testfall Id} & /TC2001/\\
\hline
\textbf{Testfall Name} & Anlegen eines Empfängers\\
\hline
\textbf{Requirement ID} & /VA0700/, /VA0800/, /VA0200/\\
\hline
\textbf{Testfall Setup} & Starten des Computers unter Linux. Durchführen des Tests.
Anschließendes Starten des Computers unter Windows XP. Erneutes Durchführen des Tests. Schlägt ein Testschritt unter mindestens einem System fehl, wird er als 'FAIL' markiert.\\
\hline
\end{tabular}
\begin{tabular}{|p{4cm}|p{7.8cm}|p{2.3cm}|}
\multicolumn{3}{|c|}{\textbf{Einzelschritte des Testfalls}} \\
\hline
\textbf{Schritt} & \textbf{Erwartetes Ergebnis} & \textbf{Ergebnis}\\
\hline
Programmstart (0) & GUI erscheint. & PASS\\
\hline
Dialog öffnen (1) & Dialog zum Hinzufügen eines neuen Empfängers öffnet sich. & PASS\\
\hline
IP-Adressen-Test (2) & Neuer Empfänger auf gewählter Gruppe erscheint im
Hauptfenster im Empfängerbereich. & PASS\\
\hline
Alle validen Gruppen prüfen (3) & Die Empfänger werden korrekt angelegt. & PASS\\
\hline
Dialog öffnen (4) & Dialog zum Hinzufügen eines neuen Empfängers öffnet sich. & PASS\\
\hline
Port-Test (5) & Neuer Empfänger auf gewähltem Port erscheint im
Hauptfenster im Empfängerbereich. & PASS\\
\hline
Alle validen Ports prüfen (3) & Die Empfänger werden korrekt angelegt. & PASS\\
\hline
\end{tabular}
\begin{tabular}{|p{3.5cm}|p{11cm}|}
\textbf{Tester} & DH\\
\hline
\textbf{Datum} & 13.05.2011\\
\hline
\textbf{Ergebnis} & PASS\\
\hline
\end{tabular}
\end{center}
\end{table}

\begin{table}[h]
\caption{/TC2002/}
\label{tab:TC2002}
\begin{center}
\begin{tabular}{|p{3.5cm}|p{11cm}|}
\hline
\textbf{Testfall Id} & /TC2002/\\
\hline
\textbf{Testfall Name} & Aktivieren von Empfängern\\
\hline
\textbf{Requirement ID} & /VA0700/, /VA0800/, /VA0200/\\
\hline
\textbf{Testfall Setup} & Starten des Computers unter Linux. Durchführen des Tests.
Anschließendes Starten des Computers unter Windows XP. Erneutes Durchführen des Tests. Schlägt ein Testschritt unter mindestens einem System fehl, wird er als 'FAIL' markiert.\\
\hline
\end{tabular}
\begin{tabular}{|p{4cm}|p{7.8cm}|p{2.3cm}|}
\multicolumn{3}{|c|}{\textbf{Einzelschritte des Testfalls}} \\
\hline
\textbf{Schritt} & \textbf{Erwartetes Ergebnis} & \textbf{Ergebnis}\\
\hline
TC2001 (1)& Hauptfenster mit angelegten,
deaktivierten Empfänger befindet sich auf dem Bildschirm. & PASS\\
\hline
Einzelne Aktivierung (2) & Empfänger wird aktiv, erkennbar am
Wechsel des roten Vierecks zu einem grünen Pfeil. & PASS\\
\hline
Multiple Aktivierung (3) & Empfänger werden aktiv,
erkennbar am Wechsel des roten Vierecks zu einem grünen Pfeil. & PASS\\
\hline
\end{tabular}
\begin{tabular}{|p{3.5cm}|p{11cm}|}
\textbf{Tester} & DH\\
\hline
\textbf{Datum} & 13.05.2011\\
\hline
\textbf{Ergebnis} & FAIL\\
\hline
\end{tabular}
\end{center}
\end{table}

\begin{table}[h]
\caption{/TC2003/}
\label{tab:TC2003}
\begin{center}
\begin{tabular}{|p{3.5cm}|p{11cm}|}
\hline
\textbf{Testfall Id} & /TC2003/\\
\hline
\textbf{Testfall Name} & Bearbeiten von Empfängern\\
\hline
\textbf{Requirement ID} & /VA0700/, /VA0800/, /VA0200/\\
\hline
\textbf{Testfall Setup} & Starten des Computers unter Linux. Durchführen des Tests.
Anschließendes Starten des Computers unter Windows XP. Erneutes Durchführen des Tests. Schlägt ein Testschritt unter mindestens einem System fehl, wird er als 'FAIL' markiert.\\
\hline
\end{tabular}
\begin{tabular}{|p{4cm}|p{7.8cm}|p{2.3cm}|}
\multicolumn{3}{|c|}{\textbf{Einzelschritte des Testfalls}} \\
\hline
\textbf{Schritt} & \textbf{Erwartetes Ergebnis} & \textbf{Ergebnis}\\
\hline
TC2001 (1)& Hauptfenster mit angelegten,
deaktivierten Empfänger befindet sich auf dem Bildschirm. & PASS\\
\hline
Öffnen des Bearbeiten-Dialogs (2) & Bearbeiten Dialog erscheint. & PASS\\
\hline
Bearbeiten eines inaktiven Empfängers (3) & Fenster schließt sich. & PASS\\
\hline
Öffnen des Bearbeiten-Dialogs & Bearbeiten Dialog erscheint. & PASS\\
\hline
Änderung überprüfen (4) & Analyse Verhalten ist auf 'lazy' eingestellt. & PASS\\
\hline
Öffnen des Bearbeiten-Dialogs (5) & Bearbeiten Dialog erscheint. & PASS\\
\hline
Bearbeiten eines aktiven Empfängers (6) & Fenster schliesst sich. & PASS\\
\hline
Wiederhole (5) & Bearbeiten Dialog erscheint. & PASS\\
\hline
Überprüfen (7) & Analyse Verhalten ist 'eager'. & PASS\\
\hline
\end{tabular}
\begin{tabular}{|p{3.5cm}|p{11cm}|}
\textbf{Tester} & DH\\
\hline
\textbf{Datum} & 13.05.2011\\
\hline
\textbf{Ergebnis} & PASS\\
\hline
\end{tabular}
\end{center}
\end{table}

\begin{table}[h]
\caption{/TC2004/}
\label{tab:TC2004}
\begin{center}
\begin{tabular}{|p{3.5cm}|p{11cm}|}
\hline
\textbf{Testfall Id} & /TC2004/\\
\hline
\textbf{Testfall Name} & Deaktivieren von Empfängern\\
\hline
\textbf{Requirement ID} & /VA0700/, /VA0800/, /VA0200/\\
\hline
\textbf{Testfall Setup} & Starten des Computers unter Linux. Durchführen des Tests.
Anschließendes Starten des Computers unter Windows XP. Erneutes Durchführen des Tests. Schlägt ein Testschritt unter mindestens einem System fehl, wird er als 'FAIL' markiert.\\
\hline
\end{tabular}
\begin{tabular}{|p{4cm}|p{7.8cm}|p{2.3cm}|}
\multicolumn{3}{|c|}{\textbf{Einzelschritte des Testfalls}} \\
\hline
\textbf{Schritt} & \textbf{Erwartetes Ergebnis} & \textbf{Ergebnis}\\
\hline
TC2002 & Hauptfenster mit angelegten,
aktiven Empfängern befindet sich auf dem Bildschirm. & PASS\\
\hline
Einzelne Deaktivierung (1) & Empfänger wird inaktiv,
erkennbar am Wechsel des grünen Pfeils zu einem roten Viereck. & PASS\\
\hline
Multiple Deaktivierung (2) & Empfänger werden
inaktiv, erkennbar am Wechsel des grünen Pfeils zu einem roten Viereck. & PASS\\
\hline
\end{tabular}
\begin{tabular}{|p{3.5cm}|p{11cm}|}
\textbf{Tester} & DH\\
\hline
\textbf{Datum} & 13.05.2011\\
\hline
\textbf{Ergebnis} & PASS\\
\hline
\end{tabular}
\end{center}
\end{table}

\begin{table}[h]
\caption{/TC2005/}
\label{tab:TC2005}
\begin{center}
\begin{tabular}{|p{3.5cm}|p{11cm}|}
\hline
\textbf{Testfall Id} & /TC2005/\\
\hline
\textbf{Testfall Name} & Entfernen von Empfängern\\
\hline
\textbf{Requirement ID} & /VA0700/, /VA0800/, /VA0200/\\
\hline
\textbf{Testfall Setup} & Starten des Computers unter Linux. Durchführen des Tests.
Anschließendes Starten des Computers unter Windows XP. Erneutes Durchführen des Tests. Schlägt ein Testschritt unter mindestens einem System fehl, wird er als 'FAIL' markiert.\\
\hline
\end{tabular}
\begin{tabular}{|p{4cm}|p{7.8cm}|p{2.3cm}|}
\multicolumn{3}{|c|}{\textbf{Einzelschritte des Testfalls}} \\
\hline
\textbf{Schritt} & \textbf{Erwartetes Ergebnis} & \textbf{Ergebnis}\\
\hline
TC2001 & Hauptfenster mit angelegten,
inaktiven Empfängern befindet sich auf dem Bildschirm. & PASS\\
\hline
Einzelnes Entfernen (1) & Empfänger ist nicht mehr
in der Liste vorhanden. & PASS\\
\hline
Multiples Entfernen (2) & Empfänger sind nicht mehr
in der Liste vorhanden. & PASS\\
\hline
\end{tabular}
\begin{tabular}{|p{3.5cm}|p{11cm}|}
\textbf{Tester} & DH\\
\hline
\textbf{Datum} & 13.05.2011\\
\hline
\textbf{Ergebnis} & PASS\\
\hline
\end{tabular}
\end{center}
\end{table}

\begin{table}[h]
\caption{/TC2006/}
\label{tab:TC2006}
\begin{center}
\begin{tabular}{|p{3.5cm}|p{11cm}|}
\hline
\textbf{Testfall Id} & /TC2004/\\
\hline
\textbf{Testfall Name} & Abfangen invalider Angaben\\
\hline
\textbf{Requirement ID} & /VA0700/, /VA0800/, /VA0200/\\
\hline
\textbf{Testfall Setup} & Starten des Computers unter Linux. Durchführen des Tests.
Anschließendes Starten des Computers unter Windows XP. Erneutes Durchführen des Tests. Schlägt ein Testschritt unter mindestens einem System fehl, wird er als 'FAIL' markiert.\\
\hline
\end{tabular}
\begin{tabular}{|p{4cm}|p{7.8cm}|p{2.3cm}|}
\multicolumn{3}{|c|}{\textbf{Einzelschritte des Testfalls}} \\
\hline
\textbf{Schritt} & \textbf{Erwartetes Ergebnis} & \textbf{Ergebnis}\\
\hline
Programmstart (0) & GUI
erscheint. & PASS\\
\hline
Dialog öffnen (1) & Dialog zum Hinzufügen eines neuen Empfängers öffnet sich. & PASS\\
\hline
IP-Adressen-Test (2) & Dialog bleibt offen und Fehlermeldung über falsche
Gruppen-Adresse wird angezeigt. & PASS\\
\hline
Alle validen Gruppen prüfen (3) & Dialog bleibt offen und Fehlermeldung über falsche
Gruppen-Adresse wird angezeigt. & PASS\\
\hline
Dialog öffnen (4) & Dialog zum Hinzufügen eines neuen Empfängers öffnet sich. & PASS\\
\hline
Port-Test (5) & Dialog bleibt offen und Fehlermeldung über falschen
Port erscheint. & PASS\\
\hline
Alle invaliden Ports prüfen (3) & Dialog bleibt offen und Fehlermeldung über falschen
Port erscheint. & PASS\\
\hline
\end{tabular}
\begin{tabular}{|p{3.5cm}|p{11cm}|}
\textbf{Tester} & DH\\
\hline
\textbf{Datum} & 13.05.2011\\
\hline
\textbf{Ergebnis} & PASS\\
\hline
\end{tabular}
\end{center}
\end{table}