\chapter{Testsuite /TS30/}
Dieses Kapitel enthält die Testergebnisse der Testsuite /TS30/.\\
Die zugehörige Requirement ID lautet /UC10/.

\begin{table}[h]
\caption{/TC3001/}
\label{tab:TC3001}
\begin{center}
\begin{tabular}{|p{3.5cm}|p{11cm}|}
\hline
\textbf{Testfall ID} & /TC3001/\\
\hline
\textbf{Testfall Name} & Laden valider Konfigurationsdateien\\
\hline
\textbf{Requirement ID} & /VA0500/ (/LVA0600/), /VA0600/ (/LVA0800/)\\
\hline
\textbf{Testfall Setup} & Starten des Computers unter Linux. Durchführen des Tests.
Anschließendes Starten des Computers unter Windows XP. Erneutes Durchführen des Tests. Schlägt ein Testschritt unter mindestens einem System fehl, wird er als 'FAIL' markiert.\\
\hline
\end{tabular}
\begin{tabular}{|p{4cm}|p{7.8cm}|p{2.3cm}|}
\multicolumn{3}{|c|}{\textbf{Einzelschritte des Testfalls}} \\
\hline
\textbf{Schritt} & \textbf{Erwartetes Ergebnis} & \textbf{Ergebnis}\\
\hline
Test Setup & Das Tool ist erfolgreich installiert. & PASS\\
\hline
Konfigurationen laden &  Das Tool lädt die Konfiguration ohne Probleme. &
PASS\\
\hline
Überprüfen geladener Werte & In der GUI des Tools werden alle Daten aus der
Konfiguration richtig angezeigt. & PASS\\
\hline
\end{tabular}

\begin{tabular}{|p{3.5cm}|p{11cm}|}
\textbf{Tester} & TSC\\
\hline
\textbf{Datum} & 13.05.2011\\
\hline
\textbf{Ergebnis} & PASS\\
\hline
\end{tabular}
\end{center}
\label{default}
\end{table}

\begin{table}[h]
\caption{/TC3002/}
\label{tab:TC3002}
\begin{center}
\begin{tabular}{|p{3.5cm}|p{11cm}|}
\hline
\textbf{Testfall ID} & /TC3002/\\
\hline
\textbf{Testfall Name} & Laden invalider Konfigurationsdateien\\
\hline
\textbf{Requirement ID} & /VA0600/ (/LVA0800/), /QZ10/ (/LQZ10/)\\
\hline
\textbf{Testfall Setup} & Starten des Computers unter Linux. Durchführen des Tests.
Anschließendes Starten des Computers unter Windows XP. Erneutes Durchführen des Tests. Schlägt ein Testschritt unter mindestens einem System fehl, wird er als 'FAIL' markiert.\\
\hline
\end{tabular}
\begin{tabular}{|p{4cm}|p{7.8cm}|p{2.3cm}|}
\multicolumn{3}{|c|}{\textbf{Einzelschritte des Testfalls}} \\
\hline
\textbf{Schritt} & \textbf{Erwartetes Ergebnis} & \textbf{Ergebnis}\\
\hline
Test-Setup &  Das Tool ist erfolgreich installiert. & PASS\\
\hline
Konfigura- tionen laden & Das Tool lädt die Konfiguration und gibt eine
Fehlermeldung aus die mindestens eines der Probleme in der Konfiguration wieder
spiegelt. & PASS\\
\hline
Reaktion des Programms & Das Tool startet und geht sinnvoll mit Fehlern um. Das
Tool kann entscheiden wie viele Informationen es aus der Konfiguration In der
GUI des Tools werden entweder sinnvollen Daten aus der Konfiguration richtig
angezeigt oder das Tool beendet sich. & PASS\\
\hline
\end{tabular}

\begin{tabular}{|p{3.5cm}|p{11cm}|}
\textbf{Tester} & TSC\\
\hline
\textbf{Datum} & 13.05.2011\\
\hline
\textbf{Ergebnis} & PASS\\
\hline
\end{tabular}
\end{center}
\label{default}
\end{table}

\begin{table}[h]
\caption{/TC3003/}
\label{tab:TC3003}
\begin{center}
\begin{tabular}{|p{3.5cm}|p{11cm}|}
\hline
\textbf{Testfall ID} & /TC3003/\\
\hline
\textbf{Testfall Name} & Profil Lebenszyklus Test\\
\hline
\textbf{Requirement ID} & /VA0500/ (/LVA0600/), /AVA0600/ (/VA0600/), /VA0700/ (/LVA0900/)\\
\hline
\textbf{Testfall Setup} & Starten des Computers unter Linux. Durchführen des Tests.
Anschließendes Starten des Computers unter Windows XP. Erneutes Durchführen des Tests. Schlägt ein Testschritt unter mindestens einem System fehl, wird er als 'FAIL' markiert.\\
\hline
\end{tabular}
\begin{tabular}{|p{4cm}|p{7.8cm}|p{2.3cm}|}
\multicolumn{3}{|c|}{\textbf{Einzelschritte des Testfalls}} \\
\hline
\textbf{Schritt} & \textbf{Erwartetes Ergebnis} & \textbf{Ergebnis}\\
\hline
                    Test-Setup &
                    Das Tool ist bereit für das Anlegen von Datenströmen &
                    PASS\\
                \hline
                    Konfigurationen Anlegen &
                    Sender und Receiver werden angelegt &
                    PASS\\
                \hline
                    Speichern des Profiles &
                    Profil wird gespeichert.&
                    PASS\\
                \hline
                    Tool beenden &
                    Tool wird beenden, der java Prozess wird beendet &
                    PASS\\
                \hline
                    Tool erneut starten & 
                    Das Tool ist bereit für das Anlegen von Datenströmen &
                    PASS\\
                \hline
                    Startzustand überprüfen & 
                    Es ist kein Sender oder Receiver angelegt &
                    PASS\\
                \hline
                    Profil laden & 
                    Das Profil wird erfolgreich geladen. Der Fenstertitel der 
                    Applikation spiegelt den Namen des Profils wieder. &
                    PASS\\
                 \hline
                    Konfigurationen überprüfen & 
                    In der Liste des Senders und Receivers ist jeweils ein Eintrag.
                    Die Werte in den Edit Dialogen sind korrekt. &
                    PASS\\
\hline
\end{tabular}

\begin{tabular}{|p{3.5cm}|p{11cm}|}
\textbf{Tester} & TSC\\
\hline
\textbf{Datum} & 13.05.2011\\
\hline
\textbf{Ergebnis} & PASS\\
\hline
\end{tabular}
\end{center}
\label{default}
\end{table}

\begin{table}[h]
\caption{/TC3004/}
\label{tab:TC3004}
\begin{center}
\begin{tabular}{|p{3.5cm}|p{11cm}|}
\hline
\textbf{Testfall ID} & /TC3004/\\
\hline
\textbf{Testfall Name} & Laden einer zuletzt verwendeten Konfigurationsdatei \\
\hline
\textbf{Requirement ID} & \\
\hline
\textbf{Testfall Setup} & Starten des Computers unter Linux. Durchführen des Tests.
Anschließendes Starten des Computers unter Windows XP. Erneutes Durchführen des Tests. Schlägt ein Testschritt unter mindestens einem System fehl, wird er als 'FAIL' markiert.\\
\hline
\end{tabular}
\begin{tabular}{|p{4cm}|p{7.8cm}|p{2.3cm}|}
\multicolumn{3}{|c|}{\textbf{Einzelschritte des Testfalls}} \\
\hline
\textbf{Schritt} & \textbf{Erwartetes Ergebnis} & \textbf{Ergebnis}\\
\hline
Test-Setup & Das Tool ist erfolgreich installiert. & PASS\\
\hline
Konfigura- tionen speichern & Konfiguration wird gespeichert. & PASS\\
\hline
Applikation beenden & Applikation beendet sich. & PASS\\
\hline
Applikation starten & Applikation startet & PASS\\
\hline
Letzte Konfiguration laden & Letzte Konfiguration wird geladen. & PASS\\
\hline
Letzte Konfiguration überprüfen & Die letzte Konfiguration wurde erfolgreich
geladen. & PASS\\
\hline
Konfiguration speichern & Konfiguration wird gespeichert. & PASS\\
\hline
Applikation beenden & Applikation beendet sich. & PASS\\
\hline
Applikation starten & Applikation startet & PASS\\
\hline
Letzte Konfigurationen überprüfen & Letzte Konfiguration und vorletzte
Konfiguration stehen zur Auswahl. & PASS\\
\hline
\end{tabular}

\begin{tabular}{|p{3.5cm}|p{11cm}|}
\textbf{Tester} & TSC\\
\hline
\textbf{Datum} & 13.05.2011\\
\hline
\textbf{Ergebnis} & PASS\\
\hline
\end{tabular}
\end{center}
\label{default}
\end{table}