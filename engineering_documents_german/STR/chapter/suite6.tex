\chapter{Testsuite /TS60/}
Dieses Kapitel enthält die Testergebnisse der Testsuite /TS60/.\\
Die zugehörige Requirement Id lautet /UC40/.

\begin{table}[h]
        \caption{/TC6001/}
        \label{tab:TC6001}
        \begin{center}
            \begin{tabular}{|p{3.5cm}|p{12cm}|}
                \hline
                    \textbf{Testfall Id} & /TC6001/\\
                \hline
                    \textbf{Testfall Name} & Sendestatistik eines Sender\\
                \hline
                    \textbf{Requirement ID} & /VA1000/, /VA1200/, /OA0300/\\
                \hline
                    \textbf{Beschreibung} & Dieser Testfall verifiziert die
                    korrekte Erstellung von Statistiken für einen einzelnen
                    Sender.\\
                \hline
            \end{tabular}
            \begin{tabular}{|p{2.5cm}|p{7.55cm}|p{5cm}|}
                \multicolumn{3}{|c|}{\textbf{Einzelschritte des Testfalls}} \\
                \hline
                    \textbf{Schritt} & \textbf{Erwartetes
                    Ergebnis} & \textbf{Ergebnis}\\
                \hline
                    Test-Setup & Das Tool
                    wird ausgeführt und ist bereit zum Starten von Sendern. &
                    Pass
                    \\
                \hline
                    Sender starten & Der Sender
                    sendet und wird angezeigt. Die maximale, minimale und
                    durchschnittliche Senderate liegt bei 10 PPS. &
                    Pass\\
                \hline
                    Erhöhen der Senderate & Die minimale Senderate
                    liegt bei 10 PPS, die maximale bei 1000 PPS, die
                    durchschnittliche bei 1000 PPS. &
                    Pass\\
                \hline
                    Pausieren des Senders & Die Statistiken bleiben erhalten.
                    Die minimale Senderate wird nicht verändert. &
                    Pass\\
                \hline
                    Aktivieren des Senders & Die
                    Statistiken bleiben erhalten. &
                    Pass\\
                \hline
                    Hinzufügen eines neuen Senders & Die durchschnittliche, maximale und minimale
                    Senderate liegen bei 1000 PPS. &
                    Pass\\
                \hline
                    Überlasten von Sendern & Die minimale Senderate wird jeweils bei niedrigerer
                    durchschnittlicher Senderate angepasst. &
                    Pass\\
                \hline
                    Detailansicht & Die angezeigten Statistiken
                    stimmen mit den Statistiken der Tabelle überein. Der Graph
                    wird um die jeweilige durchschnittliche Senderate ergänzt.
                    & Pass\\
                \hline
            \end{tabular}
            \begin{tabular}{|p{3.5cm}|p{11cm}|}
                \textbf{Tester} & RS\\
                \hline
                \textbf{Datum} & 13.05.2011\\
                \hline
                \textbf{Ergebnis} & PASS\\
                \hline
            \end{tabular}
        \end{center}
    \end{table}
    
    \begin{table}[h]
        \caption{/TC6002/}
        \label{tab:TC6002}
        \begin{center}
            \begin{tabular}{|p{3.5cm}|p{12cm}|}
                \hline
                    \textbf{Testfall Id} & /TC6002/\\
                \hline
                    \textbf{Testfall Name} & Globale Sender-Statistiken\\
                \hline
                    \textbf{Requirement ID} & /VA1000/\\
                \hline
                    \textbf{Beschreibung} & Dieser Testfall verifiziert die
                    korrekte Erstellung von globalen Sender-Statistiken.\\
                \hline
            \end{tabular}
            \begin{tabular}{|p{2.5cm}|p{5cm}|p{7.55cm}|}
                \multicolumn{3}{|c|}{\textbf{Einzelschritte des Testfalls}} \\
                \hline
                    \textbf{Schritt} &  & \textbf{Ergebnis}\\
                \hline
                    Test-Setup & Das Tool
                    wird ausgeführt und ist bereit zum Starten von Sendern. \\
                \hline
                    Sender starten & Der Sender
                    sendet und wird angezeigt. Die Senderate liegt bei 10 PPS.
                    Die globale Senderate ist identisch. Der Zähler für die
                    Anzahl insgesamt gesendeter Pakete erhöht sich pro Sekunde
                    im Schnitt um 10 Pakete. & Pass\\
                \hline
                    Sender hinzufügen & Die globale Senderate
                    liegt bei 100 PPS. Der Zähler für die Anzahl insgesamt
                    gesendeter Pakete erhöht sich pro Sekunde im Schnitt um 100
                    Pakete. & Pass\\
                \hline
                    Sender hinzufügen & Die globale Senderate
                    liegt bei 1000 PPS. Der Zähler für die Anzahl insgesamt
                    gesendeter Pakete erhöht sich pro Sekunde im Schnitt um 1000
                    Pakete. & Pass\\
                \hline
                    Sender pausieren & Die globale Senderate liegt bei 100 PPS. Der
                    Zähler für die Anzahl insgesamt gesendeter Pakete erhöht
                    sich pro Sekunde im Schnitt um 100 Pakete. & Pass\\
                \hline
                    Sender löschen & Die globale Senderate liegt bei 10 PPS. Der Zähler für die
                    Anzahl insgesamt gesendeter Pakete erhöht sich pro Sekunde
                    im Schnitt um 100 Pakete.  & Pass\\
                \hline
                    Sender pausieren & Die globale Senderate liegt bei 0 PPS. Der
                    Zähler für die Anzahl insgesamt gesendeter Pakete erhöht
                    sich nicht. & Pass\\
                \hline
                    Sender hinzufügen & Die globale Senderate
                    liegt bei 500 PPS. Der Zähler für die Anzahl insgesamt
                    gesendeter Pakete erhöht sich pro Sekunde im Schnitt um 500
                    Pakete. & Pass\\
                \hline
                    Sender überlasten & Die
                    globale Senderate schwankt um einen Wert. & Pass\\
                \hline
            \end{tabular}
             \begin{tabular}{|p{3.5cm}|p{11cm}|}
                \textbf{Tester} & RS\\
                \hline
                \textbf{Datum} & 13.05.2011\\
                \hline
                \textbf{Ergebnis} & PASS\\
                \hline
            \end{tabular}
        \end{center}
    \end{table}
    
    \begin{table}[h]
        \caption{/TC6003/}
        \label{tab:TC6003}
        \begin{center}
            \begin{tabular}{|p{3.5cm}|p{12cm}|}
                \hline
                    \textbf{Testfall Id} & /TC6003/\\
                \hline
                    \textbf{Testfall Name} & Paketraten-Statistik auf
                    Empfängerseite\\
                \hline
                    \textbf{Requirement ID} & /VA1000/, /VA1200/, /VA1300/\\
                \hline
                    \textbf{Beschreibung} & Dieser Testfall verifiziert die
                    korrekte Erstellung von Statistiken über die
                    gemessene Paketrate für einen empfangenen Datenstrom.\\
                \hline
            \end{tabular}
            \begin{tabular}{|p{2.5cm}|p{7.55cm}|p{5cm}|}
                \multicolumn{3}{|c|}{\textbf{Einzelschritte des Testfalls}} \\
                \hline
                    \textbf{Schritt} & \textbf{Erwartetes Ergebnis} &
                    \textbf{Ergebnis}\\
                \hline
                    Test-Setup & Das Tool
                    wird ausgeführt und ist bereit zum Starten von Sendern. &
                    Pass\\
                \hline
                    Sender starten & Der
                    Sender sendet und wird angezeigt. & Pass\\
                \hline
                    Empfänger hinzufügen & Der Empfänger empfängt einen Datenstrom mit 100 PPS.\\
                \hline
                    Sender hinzufügen & Der Empfänger empfängt die Datenströme zweier Sender mit 100 bzw. 1000
                    PPS (jeweils konfigurierte und gemessene Paketrate). &
                    Pass\\
                \hline
                    Verändern der Senderate & Der Empfänger empfängt die Datenströme
                    zweier Sender mit je 100 PPS als konfigurierte und
                    gemessene Paketrate. & Pass\\
                \hline
                    Sender pausieren & Der Empfänger
                    zeigt den Datenstrom als pausiert an. Die Statistiken
                    bleiben erhalten. & Pass\\
                \hline
                    Sender aktivieren & Der Empfänger zeigt den Sender als aktiv an. Die Statistiken
                    werden fortgesetzt. & Pass\\
                \hline
                    Überlasten der Sender & Pakete gehen verloren. Die Empfangsrate der Datenströme schwankt analog
                    zu den Sendern. & Pass\\
                \hline
            \end{tabular}
        \begin{tabular}{|p{3.5cm}|p{11cm}|}
                \textbf{Tester} & RS\\
                \hline
                \textbf{Datum} & 13.05.2011\\
                \hline
                \textbf{Ergebnis} & PASS\\
                \hline
            \end{tabular}
        \end{center}
    \end{table}
    
    \begin{table}[h]
        \caption{/TC6004/}
        \label{tab:TC6004}
        \begin{center}
            \begin{tabular}{|p{3.5cm}|p{12cm}|}
                \hline
                    \textbf{Testfall Id} & /TC6004/\\
                \hline
                    \textbf{Testfall Name} & Globale Empfänger-Statistiken\\
                \hline
                    \textbf{Requirement ID} & /VA1000/, /VA1300/\\
                \hline
                    \textbf{Beschreibung} & Dieser Testfall verifiziert die
                    korrekte Erstellung globaler Empfänger-Statistiken.\\
                \hline
            \end{tabular}
            \begin{tabular}{|p{2.5cm}|p{7.55cm}|p{5cm}|}
                \multicolumn{3}{|c|}{\textbf{Einzelschritte des Testfalls}} \\
                \hline
                    \textbf{Schritt} &  & \textbf{Ergebnis}\\
                \hline
                    Test-Setup & Das Tool
                    wird ausgeführt und ist bereit zum Starten von Sendern. &
                    Pass\\
                \hline
                    Sender Starten & Der
                    Sender sendet und wird angezeigt. & Pass\\
                \hline
                    Empfänger hinzufügen &
                    Der Empfänger empfängt einen Datenstrom mit 100 PPS. Die
                    globale Empfangsrate liegt bei 100 PPS. Der Zähler für
                    insgesamt empfangene Pakete wird pro Sekunde um 100
                    Pakete erhöht. & Pass\\
                \hline
                    Sender hinzufügen & Der Empfänger empfängt zwei Datenströme
                    mit 100 bzw. 900 PPS. Die globale Empfangsrate liegt bei 1000 PPS. Der Zähler für
                    insgesamt empfangene Pakete wird pro Sekunde im Schnitt um
                    1000 Pakete erhöht. & Pass\\
                \hline
                    Sender pausieren & Die globale Empfangsrate liegt bei 100 PPS. Der Zähler für
                    insgesamt empfangene Pakete wird pro Sekunde im Schnitt um
                    1000 Pakete erhöht. & Pass\\
                \hline
                    Sender aktivieren & Die globale Empfangsrate liegt bei 1000 PPS. Der Zähler für
                    insgesamt empfangene Pakete wird pro Sekunde im Schnitt um
                    1000 Pakete erhöht. & Pass\\
                \hline
                    Überlasten der Sender Pakete gehen verloren. Die Summe der verlorenen Pakete der einzelnen
                    Datenströme stimmt mit den global verlorenen Paketen
                    überein. & Pass\\
                \hline
            \end{tabular}
            \begin{tabular}{|p{3.5cm}|p{11cm}|}
                \textbf{Tester} & RS\\
                \hline
                \textbf{Datum} & 13.05.2011\\
                \hline
                \textbf{Ergebnis} & PASS\\
                \hline
            \end{tabular}
        \end{center}
    \end{table}
    
    \begin{table}[h]
        \caption{/TC6005/}
        \label{tab:TC6005}
        \begin{center}
            \begin{tabular}{|p{3.5cm}|p{12cm}|}
                \hline
                    \textbf{Testfall Id} & /TC6005/\\
                \hline
                    \textbf{Testfall Name} & Detaillierte Statistik eines
                    empfangenen Datenstromes\\
                \hline
                    \textbf{Requirement ID} & /VA1000/, /VA1200/, /VA1300/,
                    /OA0300/\\
                \hline
                    \textbf{Beschreibung} & Dieser Testfall verifiziert die
                    korrekte Erstellung von Statistiken für einen einzelnen
                    empfangenen Datenstrom sowie für den einzelnen Empfänger,
                    der diesen Datenstrom empfängt.\\
                \hline
            \end{tabular}
            \begin{tabular}{|p{2.5cm}|p{5cm}|p{7.55cm}|}
                \multicolumn{3}{|c|}{\textbf{Einzelschritte des Testfalls}} \\
                \hline
                    \textbf{Schritt} & \textbf{Ergebnis}\\
                \hline
                    Test-Setup & Das Tool wird auf beiden Rechnern
                    ausgeführt und ist bereit zum Starten von Sendern. & Pass\\
                \hline
                    Zeit Synchronisierung & Die Zeiten
                    beider Computer sind hinreichend synchron. & Pass\\
                \hline
                    Sender Starten & Der Sender sendet und wird
                    angezeigt. & PAss\\
                \hline
                    Empfänger anlegen & Der Empfänger empfängt den Sender aus
                    dem Netzwerk. & Pass\\
                \hline
                    Detailansicht Paketraten & Die konfigurierte und gemessene Senderate
                    stimmen mit den Daten des Senders überein. Die gemessene
                    Empfangsrate stimmt mit der Tabellenangabe überein. & Pass\\
                \hline
                    Detailansicht Übertragungszeit & Die angezeigte Übetragungszeit stimmt
                    mit der Hälfte der Ping-Zeit überein. & Pass\\
                \hline
                    Detailansicht maximaler Versatz & Maximaler Versatz wird
                    angezeigt. & Pass\\
                \hline
            \end{tabular}
            \begin{tabular}{|p{3.5cm}|p{11cm}|}
                \textbf{Tester} & RS\\
                \hline
                \textbf{Datum} & 13.05.2011\\
                \hline
                \textbf{Ergebnis} & PASS\\
                \hline
            \end{tabular}
        \end{center}
    \end{table}