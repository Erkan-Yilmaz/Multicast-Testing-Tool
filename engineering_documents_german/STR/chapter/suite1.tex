\chapter{Testsuite /TS10/}
Dieses Kapitel enthält die Testergebnisse der Testsuite /TS10/.\\
Die zugehörige Requirement ID lautet /UC10/.

\begin{table}[h]
\caption{/TC1001/}
\label{tab:TC1001}
\begin{center}
\begin{tabular}{|p{3.5cm}|p{11cm}|}
\hline
\textbf{Testfall ID} & /TC1001/\\
\hline
\textbf{Testfall Name} & Testen auf valide Werte\\
\hline
\textbf{Requirement ID} & /VA0100/, /VA0500/, /VA0700/\\
\hline
\textbf{Beschreibung} & Der Test stellt sicher, dass ein Sender mit validen
Werten über die GUI erstellt werden kann. \\
\hline
\end{tabular}
\begin{tabular}{|p{4cm}|p{7.8cm}|p{2.3cm}|}
\multicolumn{3}{|c|}{\textbf{Einzelschritte des Testfalls}} \\
\hline
\textbf{Schritt} & \textbf{Erwartetes Ergebnis} & \textbf{Ergebnis}\\
\hline
Programm starten (0) & GUI erscheint & PASS
 \\
\hline
Dialog öffnen (1) & Der Dialog
 zum Hinzufügen eines Senders erscheint. & PASS
\\
\hline
IP-Adressen-Test (2) & Ein neuer Sender mit den zuvor eingestellten Werten erscheint im Bereich
 Empfänger. & PASS
 \\
\hline
Alle validen Gruppen testen (3) & & PASS
\\
\hline
Dialog öffnen (4) & Der Dialog
 zum Hinzufügen eines Senders erscheint. & PASS
\\
\hline
Port Test (5) & Ein neuer Sender mit den eingestellten Werten erscheint im
 Senderbereich. & PASS
\\
\hline
Testen aller validen Ports & & PASS
\\
\hline
Dialog öffnen (6) & Der Dialog
 zum Hinzufügen eines Senders erscheint. & PASS
\\
\hline
\end{tabular}
\begin{tabular}{|p{3.5cm}|p{11cm}|}
\textbf{Tester} & JJ\\
\hline
\textbf{Datum} & 13.05.2011\\
\hline
\textbf{Ergebnis} & PASS\\
\hline
\end{tabular}
\end{center}
\end{table}

\begin{table}[h]
\caption{/TC1002/}
\label{tab:TC1002}
\begin{center}
\begin{tabular}{|p{3.5cm}|p{11cm}|}
\hline
\textbf{Testfall ID} & /TC1002/\\
\hline
\textbf{Testfall Name} & Testen Aktivierung der Sender.
\\
\hline
\textbf{Requirement ID} & /VA0100/, /VA0500/, /VA0700/\\
\hline
\textbf{Beschreibung} & Der Test stellt sicher, dass angelegte Sender aktiviert
werden können.
\\
\hline
\end{tabular}
\begin{tabular}{|p{4cm}|p{7.8cm}|p{2.3cm}|}
\multicolumn{3}{|c|}{\textbf{Einzelschritte des Testfalls}} \\
\hline
\textbf{Schritt} & \textbf{Erwartetes Ergebnis} & \textbf{Ergebnis}\\
\hline
TC1001 (1) & Die angelegten Sender
 erscheinen. & PASS
\\
\hline
Aktivieren eines Senders (2) & Das rote Viereck wird mit einem grünen Dreieck
 ersetzt, was bedeutet, dass der Sender aktiv ist. & PASS
\\
\hline
Aktivieren mehrerer Sender gleichzeitig (3) & Die Sender werden aktiviert,
was die GUI durch das Ersetzen des roten Vierecks mit einem grünen Dreieck
anzeigt. & PASS
\\
\hline
Aktivieren bei Erstellung (4) & Ein bereits
 aktivierter Sender erscheint neu in der Senderliste. Dies ist an dem grünen
 Dreieck vor der Zeile des Senders zu erkennen. & PASS\\
\hline
\end{tabular}
\begin{tabular}{|p{3.5cm}|p{11cm}|}
\textbf{Tester} & JJ\\
\hline
\textbf{Datum} & 13.05.2011\\
\hline
\textbf{Ergebnis} & PASS\\
\hline
\end{tabular}
\end{center}
\end{table}

\begin{table}[h]
\caption{/TC1003/}
\label{tab:TC1003}
\begin{center}
\begin{tabular}{|p{3.5cm}|p{11cm}|}
\hline
\textbf{Testfall ID} & /TC1003/\\
\hline
\textbf{Testfall Name} & Testen auf Veränderbarkeit der Werte
\\
\hline
\textbf{Requirement ID} & /VA0100/, /VA0500/, /VA0700/\\
\hline
\textbf{Beschreibung} & Der Test stellt sicher, dass die Eigenschaften der
Sender verändert werden können.
\\
\hline
\end{tabular}
\begin{tabular}{|p{4cm}|p{7.8cm}|p{2.3cm}|}
\multicolumn{3}{|c|}{\textbf{Einzelschritte des Testfalls}} \\
\hline
\textbf{Schritt} & \textbf{Erwartetes Ergebnis} & \textbf{Ergebnis}\\
\hline
Wiederholen von TC1002 Schritte 1-3 & 
Eine Liste von Sendern ist
 angelegt. Der erste Sender ist aktiviert, alle anderen sind deaktiviert. & PASS
\\
\hline
Öffnen des Bearbeiten-Dialogs (1) & Der Dialog zum Bearbeiten erscheint. & PASS
 \\
\hline
Bearbeiten eines inaktiven Senders (2) & Der Dialog schließt sich. & PASS
 \\
\hline
Wiederhole 1 & & PASS.
\\
\hline
Änderung überprüfen (3) & Das Analyseverhalten ist 'lazy'. & PASS
\\
\hline
Öffnen des Bearbeiten-Dialogs (4) & Der Dialog zum Bearbeiten erscheint. & PASS
\\
\hline
Bearbeiten eines aktiven Senders (5) & Der Dialog schließt sich. & PASS
\\
\hline
Wiederhole 4 & & PASS
\\
\hline
Änderung überprüfen (6) & Das Analyseverhalten ist 'eager'. & PASS
\\
\hline
\end{tabular}
\begin{tabular}{|p{3.5cm}|p{11cm}|}
\textbf{Tester} & JJ\\
\hline
\textbf{Datum} & 13.05.2011\\
\hline
\textbf{Ergebnis} & PASS\\
\hline
\end{tabular}
\end{center}
\end{table}

\begin{table}[h]
\caption{/TC1004/}
\label{tab:TC1004}
\begin{center}
\begin{tabular}{|p{3.5cm}|p{11cm}|}
\hline
\textbf{Testfall ID} & /TC1004/\\
\hline
\textbf{Testfall Name} & Testen auf Deaktivieren von Sendern.
\\
\hline
\textbf{Requirement ID} & /VA0100/, /VA0500/, /VA0700/\\
\hline
\textbf{Beschreibung} & Der Test stellt sicher, dass aktive Sender deaktiviert
werden können.
\\
\hline
\end{tabular}
\begin{tabular}{|p{4cm}|p{7.8cm}|p{2.3cm}|}
\multicolumn{3}{|c|}{\textbf{Einzelschritte des Testfalls}} \\
\hline
\textbf{Schritt} & \textbf{Erwartetes Ergebnis} & \textbf{Ergebnis}\\
\hline
TC1001 (1) & Eine Liste
 mit aktiven Sendern liegt vor. & PASS
\\
\hline
Deaktivieren eines Senders & Das grüne Dreieck wird mit einem roten Viereck
 ersetzt, was bedeutet, dass der Sender inaktiv ist. & PASS
 \\
\hline
Deaktivieren mehrerer Sender gleichzeitig (3) & Die Sender werden
deaktiviert, was die GUI durch das Ersetzen des grünen Dreiecks mit einem roten
Viereck anzeigt. & PASS
 \\
\hline
\end{tabular}
\begin{tabular}{|p{3.5cm}|p{11cm}|}
\textbf{Tester} & JJ\\
\hline
\textbf{Datum} & 13.05.2011\\
\hline
\textbf{Ergebnis} & PASS\\
\hline
\end{tabular}
\end{center}
\end{table}

\begin{table}[h]
\caption{/TC1005/}
\label{tab:TC1005}
\begin{center}
\begin{tabular}{|p{3.5cm}|p{11cm}|}
\hline
\textbf{Testfall ID} & /TC1005/\\
\hline
\textbf{Testfall Name} & Testen auf Entfernen von Sendern
\\
\hline
\textbf{Requirement ID} & /VA0100/, /VA0500/, /VA0700/\\
\hline
\textbf{Beschreibung} & Der Test stellt sicher, dass Sender aus der Senderliste
entfernt werden können.
\\
\hline
\end{tabular}
\begin{tabular}{|p{4cm}|p{7.8cm}|p{2.3cm}|}
\multicolumn{3}{|c|}{\textbf{Einzelschritte des Testfalls}} \\
\hline
\textbf{Schritt} & \textbf{Aktion} & \textbf{Ergebnis}\\
\hline
TC1001 (1) & Eine Liste
 von Sendern liegt vor. & PASS
\\
\hline
Entfernen eines Senders & Der entsprechende Sender ist nicht mehr in der Liste
 vorhanden. & PASS
\\
\hline
Entfernen mehrerer Sender gleichzeitig (3) & Die Sender
werden entfernt und die Senderliste ist leer. & PASS
\\
\hline
\end{tabular}
\begin{tabular}{|p{3.5cm}|p{11cm}|}
\textbf{Tester} & JJ\\
\hline
\textbf{Datum} & 13.05.2011\\
\hline
\textbf{Ergebnis} & PASS\\
\hline
\end{tabular}
\end{center}
\end{table}

\begin{table}[h]
\caption{/TC1006/}
\label{tab:TC1006}
\begin{center}
\begin{tabular}{|p{3.5cm}|p{11cm}|}
\hline
\textbf{Testfall ID} & /TC1006/\\
\hline
\textbf{Testfall Name} & Testen auf invalide Eingaben
\\
\hline
\textbf{Requirement ID} & /VA0100/, /VA0500/, /VA0700/\\
\hline
\textbf{Beschreibung} & Der Test stellt sicher, dass das Programm invalide
Eingaben erkennt und abfängt.\\
\hline
\end{tabular}
\begin{tabular}{|p{4cm}|p{7.8cm}|p{2.3cm}|}
\multicolumn{3}{|c|}{\textbf{Einzelschritte des Testfalls}} \\
\hline
\textbf{Schritt} & \textbf{Aktion} & \textbf{Ergebnis}\\
\hline
Programm starten (0) & GUI erscheint & PASS
\\
\hline
Dialog öffnen (1) & Der Dialog
 zum Hinzufügen eines Senders erscheint. & PASS
\\
\hline
IP Adressen Test (2) & Es erscheint zusätzlich ein Dialog, welcher auf eine Fehleingabe hinweist.
 Der 'Erstellen'-Dialog bleibt offen. & PASS
\\
\hline
Weitere invaliden Adressen testen (3) & Es erscheint zusätzlich ein Dialog, welcher auf eine Fehleingabe hinweist.
 Der 'Erstellen'-Dialog bleibt offen. & PASS
\\
\hline
Dialog öffnen (4) & Der Dialog
 zum Hinzufügen eines Senders erscheint. & PASS
\\
\hline
\end{tabular}
\end{center}
\end{table}

\begin{table}[h]
\begin{center}
\begin{tabular}{|p{4cm}|p{7.8cm}|p{2.3cm}|}
\hline
Port Test (5) & Es erscheint zusätzlich ein Dialog, welcher auf eine Fehleingabe
 hinweist. Der Erstellen Dialog bleibt offen. & PASS
\\
\hline
Testen weiterer invalider Ports & & PASS
\\
\hline
Dialog öffnen (6) & Der Dialog
 zum Hinzufügen eines Senders erscheint. & PASS
\\
\hline
Paketraten Test (7) & Es erscheint zusätzlich ein Dialog, welcher auf eine Fehleingabe
 hinweist. Der Erstellen Dialog bleibt offen. & PASS
\\
\hline
Testen weiterer invalider Paketraten & & PASS
\\
\hline
Dialog öffnen (8) & Der Dialog
 zum Hinzufügen eines Senders erscheint. & PASS
\\
\hline
Paketgrößentest (9) & Es erscheint zusätzlich ein Dialog, welcher auf eine Fehleingabe
 hinweist. Der Erstellen Dialog bleibt offen. & PASS
\\
\hline
Testen weiterer invalider Paketgrößen & & PASS
\\
\hline
Dialog öffnen (10) & Der Dialog
 zum Hinzufügen eines Senders erscheint. & PASS
\\
\hline
\end{tabular}
\begin{tabular}{|p{3.5cm}|p{11cm}|}
\textbf{Tester} & JJ\\
\hline
\textbf{Datum} & 13.05.2011\\
\hline
\textbf{Ergebnis} & PASS\\
\hline
\end{tabular}
\end{center}
\end{table}