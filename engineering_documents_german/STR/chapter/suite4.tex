\chapter{Testsuite /TS40/}
Dieses Kapitel enthält die Testergebnisse der Testsuite /TS40/.\\
Die zugehörige Requirement ID lautet /UC20/.

\begin{table}[h]
    \caption{/TC4001/}
    \label{tab:TC4001}
    \begin{center}
        \begin{tabular}{|p{3.5cm}|p{11cm}|}
            \hline
                \textbf{Testfall ID} & /TC4001/\\
            \hline
                \textbf{Testfall Name} & CLI-Konfiguration\\
            \hline
                \textbf{Requirement ID} & VA0900 \& VA0600\\
            \hline
                \textbf{Testfall Setup} & Installiere das MC-Tool.\\
            \hline
        \end{tabular}
        \begin{tabular}{|p{2cm}|p{3.9cm}|p{3.9cm}|p{3.8cm}|}
            \multicolumn{4}{|c|}{\textbf{Einzelschritte des Testfalls}} \\
            \hline
                \textbf{Schritt} & \textbf{Aktion} & \textbf{Erwartetes
                Ergebnis} & \textbf{Ergebnis}\\
            \hline
                Starten und Konfiguration laden - korrekte Pfadangabe & Eingabe
                des Befehls zum Starten und Laden einer Konfiguration mit
                korrekter Pfadangabe zu einer Konfigurationsdatei & Anlegen
                aller Sender und Empfänger im selben Zustand, wie zum Zeitpunkt
                des Speicherns \& Ausgabe auf der Commandline, die den Erfolg
                des Ladens bestätigt. & PASS \\
            \hline
                Konfiguration laden - fehlerhafte Pfadangabe & Eingabe des
                Befehls zum Starten und Laden einer Konfiguration mit
                fehlerhafter Pfadangabe zu einer Konfigurationsdatei & Ausgabe
                einer Fehlermeldung, dass keine Datei gefunden wurde &
                PASS \\
            \hline
                Konfiguration laden - fehlerhafte Datei & Eingabe des Befehls
                zum Starten und Laden einer Konfiguration mit Pfadangabe zu
                einer fehlerhaften/nicht kompatiblen Datei & Ausgabe einer
                Fehlermeldung, dass die gelesene Datei fehlerhaft ist &
                PASS \\
            \hline
        \end{tabular}
        \begin{tabular}{|p{3.5cm}|p{11cm}|}
                \textbf{Tester} & TST\\
            \hline
                \textbf{Datum} & 14.05.2011\\
            \hline
                \textbf{Ergebnis} & PASS\\
            \hline
        \end{tabular}
    \end{center}
\end{table}

\begin{table}[h]
    \caption{/TC4002/}
    \label{tab:TC4002}
    \begin{center}
        \begin{tabular}{|p{3.5cm}|p{11cm}|}
            \hline
                \textbf{Testfall ID} & /TC4002/\\
            \hline
                \textbf{Testfall Name} & Starten aller Sender und Empfänger per
                CLI\\
            \hline
                \textbf{Requirement ID} & VA0900 \& VA0600\\
            \hline
                \textbf{Testfall Setup} & Installiere das MC-Tool\\
            \hline
        \end{tabular}
        \begin{tabular}{|p{2cm}|p{3.9cm}|p{3.9cm}|p{3.8cm}|}
            \multicolumn{4}{|c|}{\textbf{Einzelschritte des Testfalls}} \\
            \hline
                \textbf{Schritt} & \textbf{Aktion} & \textbf{Erwartetes
                Ergebnis} & \textbf{Ergebnis}\\
            \hline
                Eingabe des gültigen Start-Befehls & Eingabe des Befehls zum
                Starten des Programms, sowie zum Laden und Starten aller Sender
                und Empfänger & Ausgabe auf der Commandline, die den Erfolg des
                Startens bestätigt & PASS \\
            \hline
                Eingabe eines ungültigen Start-Befehls & Eingabe eines
                ungültigen Befehls zum Starten des Programms, sowie zum Laden
                und Starten aller Empfänger & Ausgabe auf CLI, die die ungültige
                Eingabe quittiert & PASS \\
            \hline
        \end{tabular}
        \begin{tabular}{|p{3.5cm}|p{11cm}|}
                \textbf{Tester} & TST\\
            \hline
                \textbf{Datum} & 14.05.2011\\
            \hline
                \textbf{Ergebnis} & PASS\\
            \hline
        \end{tabular}
    \end{center}
\end{table}

\begin{table}[h]
    \caption{/TC4003/}
    \label{tab:TC4003}
    \begin{center}
        \begin{tabular}{|p{3.5cm}|p{11cm}|}
            \hline
                \textbf{Testfall ID} & /TC4003/\\
            \hline
                \textbf{Testfall Name} & Starten des Programms ohne Starten der
                Sender und Empfänger per CLI \\
            \hline
                \textbf{Requirement ID} & VA0900 \& VA0600\\
            \hline
                \textbf{Testfall Setup} & Installiere das MC-Tool\\
            \hline
        \end{tabular}
        \begin{tabular}{|p{2cm}|p{3.9cm}|p{3.9cm}|p{3.8cm}|}
            \multicolumn{4}{|c|}{\textbf{Einzelschritte des Testfalls}} \\
            \hline
                \textbf{Schritt} & \textbf{Aktion} & \textbf{Erwartetes
                Ergebnis} & \textbf{Ergebnis}\\
            \hline
                Eingabe des gültigen Start-Befehls & Eingabe des Befehls zum
                Starten des Programms, sowie zum Laden und Starten keiner Sender
                und Empfänger & Ausgabe auf der Commandline, die den Erfolg des
                Startens des Programms bestätigt & PASS\\
            \hline
                Eingabe eines ungültigen Start-Befehls & Eingabe eines ungültigen
                Befehls zum Starten des Programms, sowie zum Laden und Starten
                keiner Sender und Empfänger & Ausgabe auf CLI, die die ungültige
                Eingabe quittiert & PASS \\
            \hline
        \end{tabular}
        \begin{tabular}{|p{3.5cm}|p{11cm}|}
                \textbf{Tester} & TST\\
            \hline
                \textbf{Datum} & 14.05.2011\\
            \hline
                \textbf{Ergebnis} & PASS\\
            \hline
        \end{tabular}
    \end{center}
\end{table}

\begin{table}[h]
    \caption{/TC4004/}
    \label{tab:TC4004}
    \begin{center}
        \begin{tabular}{|p{3.5cm}|p{11cm}|}
            \hline
                \textbf{Testfall ID} & /TC4004/\\
            \hline
                \textbf{Testfall Name} & Ausgabe der Daten im CLI\\
            \hline
                \textbf{Requirement ID} & VA0900 \& VA0600\\
            \hline
                \textbf{Testfall Setup} & Installiere das MC-Tool\\
            \hline
        \end{tabular}
        \begin{tabular}{|p{2cm}|p{3.9cm}|p{3.9cm}|p{3.8cm}|}
            \multicolumn{4}{|c|}{\textbf{Einzelschritte des Testfalls}} \\
            \hline
                \textbf{Schritt} & \textbf{Aktion} & \textbf{Erwartetes
                Ergebnis} & \textbf{Ergebnis}\\
            \hline
                Test-Setup Teil 2 & Starten des MC-Tools mit gültigem Pfad und
                Start-Befehl & Bestätigung auf CLI & PASS\\
            \hline
                Beobachten der Ausgabe & Auf Update der Ausgabe warten & Ausgabe
                der Statistik auf dem CLI & PASS\\
            \hline
        \end{tabular}
        \begin{tabular}{|p{3.5cm}|p{11cm}|}
                \textbf{Tester} & TST\\
            \hline
                \textbf{Datum} & 14.05.2011\\
            \hline
                \textbf{Ergebnis} & PASS\\
            \hline
        \end{tabular}
    \end{center}
\end{table}

\begin{table}[h]
    \caption{/TC4005/}
    \label{tab:TC4005}
    \begin{center}
        \begin{tabular}{|p{3.5cm}|p{11cm}|}
            \hline
                \textbf{Testfall ID} & /TC4005/\\
            \hline
                \textbf{Testfall Name} & Loggen der Statistik\\
            \hline
                \textbf{Requirement ID} & VA0900 \& VA0600\\
            \hline
                \textbf{Testfall Setup} &Installiere das MC-Tool\\
            \hline
        \end{tabular}
        \begin{tabular}{|p{2cm}|p{3.9cm}|p{3.9cm}|p{3.8cm}|}
            \multicolumn{4}{|c|}{\textbf{Einzelschritte des Testfalls}} \\
            \hline
                \textbf{Schritt} & \textbf{Aktion} & \textbf{Erwartetes
                Ergebnis} & \textbf{Ergebnis}\\
            \hline
                Test-Setup Teil 2 & Starten des MC-Tool via Commandline mit
                gültigem Start-Befehl und gestarteten Sendern/Receivern &
                Bestätigung in Commandline Interface & PASS\\
            \hline
                Korrekte Log-Daten & Öffnen der Log-Datei und Vergleich mit
                Ausgabe auf dem CLI & Logger funktioniert fehlerfrei -
                geloggte Daten sind korrekt & PASS\\
            \hline
                Inkorrekte Log-Daten & Öffnen der Log-Datei und Vergleich mit
                Ausgabe auf dem CLI & Logger funktioniert fehlerhaft - geloggte
                Daten stimmen nicht mit dem CLI überein & 
                PASS\\
            \hline
                Inkorrektes Log-Format & Öffnen der Log-Datei und betrachten des
                Log-Formats & Daten werden nicht als XML gespeichert. &
                PASS\\
            \hline
        \end{tabular}
        \begin{tabular}{|p{3.5cm}|p{11cm}|}
                \textbf{Tester} & TST\\
            \hline
                \textbf{Datum} & 14.05.2011\\
            \hline
                \textbf{Ergebnis} & PASS\\
            \hline
        \end{tabular}
    \end{center}
\end{table}

\begin{table}[h]
    \caption{/TC4006/}
    \label{tab:TC4006}
    \begin{center}
        \begin{tabular}{|p{3.5cm}|p{11cm}|}
            \hline
                \textbf{Testfall ID} & /TC4006/\\
            \hline
                \textbf{Testfall Name} & Starten des MC-Tools ohne GUI\\
            \hline
                \textbf{Requirement ID} & VA0900 \& VA0600\\
            \hline
                \textbf{Testfall Setup} & Installiere das MC-Tool\\
            \hline
        \end{tabular}
        \begin{tabular}{|p{2cm}|p{3.9cm}|p{3.9cm}|p{3.8cm}|}
            \multicolumn{4}{|c|}{\textbf{Einzelschritte des Testfalls}} \\
            \hline
                \textbf{Schritt} & \textbf{Aktion} & \textbf{Erwartetes
                Ergebnis} & \textbf{Ergebnis}\\
            \hline
                Starten des MC-Tools & Starten des MC-Tool via Commandline mit
                gültigem Start-Befehl und nogui Parameter & Programm wird ohne
                GUI gestartet. & PASS\\
            \hline
        \end{tabular}
        \begin{tabular}{|p{3.5cm}|p{11cm}|}
                \textbf{Tester} & TST\\
            \hline
                \textbf{Datum} & 14.05.2011\\
            \hline
                \textbf{Ergebnis} & \cellcolor{green} PASS \\
            \hline
        \end{tabular}
    \end{center}
\end{table}