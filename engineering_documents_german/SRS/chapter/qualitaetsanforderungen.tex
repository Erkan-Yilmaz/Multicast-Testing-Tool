\chapter{Qualit"atsanforderungen}
\label{cha:qual}


\begin{table}[htdp]
\caption{Definition der Qualit"atsanforderungen}
\label{tab:quality}
\begin{center}
\begin{tabular}{|l|c|c|c|c|}
\hline
\textbf{Systemqualit"at} & \textbf{sehr gut} & \textbf{gut} & \textbf{normal} & \textbf{nicht relevant} \\
\hline
Funktionalit"at & & x & & \\
\hline
Zuverl"assigkeit & & x & & \\
\hline
Benutzbarkeit & & x & & \\
\hline
Effizienz & & x & & \\
\hline
Wartbarkeit & & & x & \\
\hline
Portabilit"at & x & & &  \\
\hline
\end{tabular}
\end{center}
\label{default}
\end{table}

\section{Funktionalit"at}

\paragraph{/QF10/ (/LQF10/)} Die Funktionalität des Systems soll in einer
einzelnen Programmdatei zusammengefasst werden.

\paragraph{/QF20/ (/LQF20/)} Die vollständige Funktionalität des
Multicast Test Tool der Hirschmann Automation GmbH soll auch im neuen
Produkt verfügbar sein.

\paragraph{/QF30/ (/LQF30/)} Die vollständige Funktionalität des Systems muss
auch über die Kommandozeile erreichbar sein.

\paragraph{/QF40/ (/LQF40/)} Das Produkt soll empfangsseitig kompatibel zum
Multicast Test Tool der Hirschmann Automation GmbH sein. Siehe hierzu auch /VA0500/
 
\section{Zuverl"assigkeit}

\paragraph{/QZ10/ (/LQZ10/)} Die Software darf bei fehlerhafter Konfiguration
durch den Benutzer nicht abstürzen. Darunter fallen doppelt vergebene Ports, ungültige
Multicast-Adressen, ungültige Absenderadressen oder Syntaxfehler in
Konfigurationsdateien.

\paragraph{/QZ20/ (/LQZ20/)} Die Software darf nicht durch fehlerhafte
Implementation von Multicasting-Protokollen in den Netzwerkkomponenten zum Absturz gebracht werden.
Es muss damit gerechnet werden, dass Pakete nicht, oder beim falschen Empfänger
ankommen. Außerdem muss die Anwendung damit zurecht kommen, dass Pakete oder
deren Header unterwegs verändert werden.

\paragraph{/QZ30/ (/LQZ30/)} Alle gesendeten Datenpakete müssen unter allen
Umständen dem Protokoll der Anwendung gehorchen. Dies muss mit Anwendungen, wie
Wireshark belegt werden können. Analog müssen alle unbeschädigt empfangenen
Datenpakete korrekt nach dem Protokoll der Anwendung weiterverarbeitet werden.
Das Protokoll sollte hierfür keinen Interpretationsspielraum lassen.

\paragraph{/QZ40/ (/LQZ40/)} Beschädigte Pakete, oder solche, die dem
Anwendungsprotokoll nicht gehorchen, sollen trotzdem, soweit es die enthaltenen Daten noch zulassen,
weiterverarbeitet und ausgewertet werden. Unter keinen Umständen darf hierbei
aber die Stabilität oder gar Sicherheit der Anwendung gefährdet werden.
Pufferüberläufe durch korrumpierte
Pakete sind dringend zu vermeiden.

\paragraph{/QZ50/ (/LQZ50/)} Innerhalb der Messdatenverarbeitung dürfen keine
Rundungs- und Berechnungsfehler auftreten, die eine Messgenauigkeit von 5
Millisekunden beeinträchtigen.

\section{Benutzbarkeit}

\paragraph{/QU10/ (/LQU10/)} Die grafische Oberfläche soll sich an das
Multicast Test Tool der Hirschmann Automation GmbH anlehnen.

\paragraph{/QU20/ (/LQU20/)} Die Multicasting-Ströme sollen übersichtlich in
einer Liste angezeigt werden.

\paragraph{/QU30/ (/LQU30/)} Ein Multicasting-Strom soll mit einem Mausklick
direkt in der Liste aus /LQU20/ aktiviert und deaktiviert werden können.

\paragraph{/QU40/ (/LQU40/)} Die grafische Oberfläche soll gesendete
und empfangene Multicast-Strömen eindeutig optisch trennen. Unabhängig von der
konkreten Darstellung der einzelnen Ströme muss \emph{auf einen Blick}
ersichtlich sein, welche davon gerade senden und welche empfangen werden.

\paragraph{/QU50/ (/LQU50/)} Die Parameter der Datenströme sollen intuitiv
direkt bearbeitbar sein. Dies soll auch für mehrere Datenströme gleichzeitig möglich
sein.

\paragraph{/QU60/ (/LQU60/)} Messergebnisse und ihr Zweck müssen eindeutig
präsentiert werden. Eine numerische, wenn auch klar beschriftete, Anzeige ist hierfür
vorerst ausreichend.

\paragraph{/QU70/ (/LQU70/)} Das Produkt soll von einem Anwender, der mit dem
Netzwerkhintergrund vertraut ist, schnell erlernbar sein. Eine logisch
strukturierte Menüführung, sowie eindeutige Befehlsbeschriftungen sind hierfür
ausschlaggebend. Mindestens für komplexere Befehle sollten auch erläuternde
Tooltips eingeblendet werden. Fachbegriffe sollten nicht künstlich vermieden
werden, da einem erfahrenen Nutzer sonst die wahren Auswirkungen eines Befehls
unnötig verschleiert werden.

\section{Effizienz}

\paragraph{/QE10/ (/LQE10/)} Auf einem gewöhnlichen Arbeitsrechner (CPU-Takt >
1GHz, RAM-Kapazität > 1GB, 100Mbit-Netzwerkadapter) müssen mindestens 30
Multicast-Ströme zeitgleich gesendet und empfangen werden können.

\section{Wartbarkeit}

\paragraph{/QW10/ (/LQW10/)} Die Software soll klar modularisiert sein.
Modulgrenzen müssen eindeutig definiert und die Module selbst austauschbar sein. Als
verbindliches Dokument wird hierfür eine detaillierte Moduldokumentation
geliefert.

\paragraph{/QW20/ (/LQW20/)} Die Module müssen einzeln testbar, sowie
Fehlerzustände analysierbar sein. Ein Testplan, sowie die Testergebnisse sind schriftlich
vorzulegen. Bei einem Fehlerzustand, der im laufenden Betrieb auftritt, ist
abzuwägen, ob es sinnvoll ist, den Benutzer direkt mit einer Meldung zu
konfrontieren, oder ob ein Eintrag ins Protokoll zur späteren Analyse
ausreichend ist.

\paragraph{/QW30/ (/LQW30/)} Schnittstellen zwischen Modulen sind als Interfaces
zu definieren und nach Javadoc-Konventionen zu dokumentieren.

\paragraph{/QW40/ (/LQW40/)} Schnittstellen für spätere Erweiterungen sollen im
Produkt vorhanden und dokumentiert sein.

\section{Portabilität}

\paragraph{/QP10/ (/LQP10/)} Die Software muss sowohl unter Linux, als auch
unter Windows (ab Windows XP) funktionieren.

\paragraph{/QP20/ (/LQP20/)} Die Software soll das Netzwerk von verschiedenen
Knoten aus analysierbar machen, darum muss sie auch auf unterschiedlicher Hardware ihre
volle Funktionalität bereitstellen. Vor Allem sollte die Software mit gängigen
Netzwerkkarten kompatibel sein.

\paragraph{/QP30/ (/LQP30/)} Das Programm soll von einem erfahrenen Benutzer
ohne besondere Vorbereitungen installierbar sein. Hierfür wird ein
Installationsskript oder ein Installer zur Verfügung gestellt. (<-- ist das
noch im Zeitplan?)

\paragraph{/QP40/ (/LQP40/)} Die Software darf, wenn sie gemeinsam mit anderen
Programmen auf demselben Gerät betrieben wird, keine Konflikte erzeugen. Dies gilt
insbesondere dann, wenn die Fremdsoftware ähnliche Aufgaben, wie das Produkt
erfüllt. Namentlich dürfen keine Konflikte mit Wireshark oder dem
Hirschmann-Tool auftreten.
