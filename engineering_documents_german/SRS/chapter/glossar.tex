\chapter{Glossar}
\label{cha:glos}
%TODO Gloassar des guten Willen wegen erstellen :)
%z.B. Multicast, ipv4, IGMP, traversierungszeit

\paragraph{Host} Ein an ein Netzwerk angeschlossener Computer.

\paragraph{IGMP} Internet Group Management Protocol. Basiert auf dem Internet Protocol und ermöglicht IPv4 Multicasting im Internet. \\

\paragraph{IPv4} Internet Protocol in der Version 4. Bildet eine wichtige Grundlage bei der Datenübertragung im Internet. Bis 2010+ der Standard. \\

\paragraph{IPv6} Internet Protocol in der Version 6. Der Nachfolger von IPv4. Ermöglicht eine Adressierung von mehr Geräten. Schrittweise Einführung ab 2010/2011.\\

\paragraph{LAN} Local Area Network. 

\paragraph{Multicast} Nachrichtenübertragung von einem Punkt zu einer Gruppe in der Telekommunikation.\\

\paragraph{OSI model} Open Systems Interconnection model. Ein Referenzdesign für
Netzwerkanwendungen, das die Kommunikation auf mehrere Schichten aufteilt. Für
dieses Projekt von Relevanz sind vor Allem die OSI-Schicht 3 (Network) mit IP
und IGMP, sowie die OSI-Schicht 4 (Transport) mit UDP.

\paragraph{TCP} Transmission Control Protocol. Ein Transportprokoll, das
sicherstellt, dass alle Pakete vollständig und in der richtigen Reihenfolge
das Ziel erreichen. Für dieses Projekt nicht weiter von Relevanz.

\paragraph{Traversierungszeit} Entspricht in diesem Projekt der gesamten Übertragungszeit eines Multicasts. \\

\paragraph{UDP} User Datagram Protocol. Ein einfaches Transportprotokoll zur
Paketübertragung. Eignet sich für Multicasting, da es keine nachvollziehbare
Verbindung zwischen Sender und Empfänger voraussetzt.