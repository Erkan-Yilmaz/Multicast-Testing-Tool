\chapter{Abnahmekriterien}
\label{cha:abna}
%TODO Abnahmekriterien f�r funktionale und Qualit�tsanforderungen
% -> Baltzer Seite 471-473!

\section{Abnahmekriterien f"ur funktionale Anforderungen}

%------
%\paragraph{/A0100/ (/VA0100/)}
%\texttt{Beschreibung:} %entspricht titel
%\texttt{Abnahmekriterium:} %abstrakt->ohne zahlen, konkret->mit zahlen
%\texttt{Ausgangssituation:} %Situation vor dem Test, was muss gegeben sein?
%\texttt{Ereignis:} %Wie verl�uft der test?
%\texttt{Erwartetes Ergebnis:} %Wie soll das System reagieren?

\paragraph{/AVA0100/ (/VA0100/)} Multicast-Sendefähigkeit\\% Komponententest der Sendekomponente%
\texttt{Abnahmekriterium:} \\(abstrakt)
\texttt{Ausgangssituation:} \\Im Netzwerk laufen mehrere Multicast-Tools auf
verschiedenen Hosts.
\texttt{Ereignis:} Es wird in einem laufenden Tool eine Multicast-Gruppe
angelegt.In dieser werden Pakete an die restlichen Hosts verschickt. Dieser Test
wird mit  verschiedenen Paketgrößen und in unterschiedlichen Intervallen
wiederholt. 
\texttt{Erwartetes Ergebnis:} Die anderen Tools zeigen das Eingehen von Paketen
mit den eingestellten Parametern an.

\paragraph{/AVA0200/ (/VA0200/)} Multicast-Empfangsfähigkeit\\% Komponententest der Empfangskomponente %
\texttt{Abnahmekriterium:} \\ (abstrakt)
\texttt{Ausgangssituation:} Ein Sender im funktionsfähigen
Netzwerk der Pakete als Multicast (über IGMP Protokoll) an einen bestimmten
Port schicken kann. Auf einem anderen Host läuft der zu testende Empfänger, der
die Pakete auf dem definierten Port empfängt. 
\texttt{Ereignis:} \\Der Sender wird aktiviert. 
\texttt{Erwartetes Ergebnis:}\\Der Empfänger zeigt auf der Benutzeroberfläche und in den Statistiken an, dass er Pakete empfangen hat.

\paragraph{/AVA0300/ (/VA0300/)} Gleichzeitigkeit\\  %Komponententest der Empfangs- und Sendekomponente%
\texttt{Abnahmekriterium:} \\(abstrakt)
zahlen \texttt{Ausgangssituation:} \\Im funktionsfähigen Netzwerk laufen mehrere
Multicast-Tools auf verschiedenen Hosts. Es sind 30 Streams in einem
Sender angelegt. 
\texttt{Ereignis:} \\Die Streams werden gestartet.
und empfangen. 
\texttt{Erwartetes Ergebnis:} \\Die Empfänger erhalten Pakete von allen Streams.

\paragraph{/AVA0400/ (/VA0400/)} Kompatibilität\\ %Test der Sende- und Empfangskomponenten in Verbindung mit dem ursprünglichen Software-System.%
\texttt{Abnahmekriterium:} \\(abstrakt)
\texttt{Ausgangssituation:} \\Im funktionsfähigen Netzwerk existiert ein altes
Hirschmann-Tool, welches als Sender fungiert. Auf einem anderen Host
läuft das neue Multicast-Tool. 
\texttt{Ereignis:} \\Der Sender wird aktiviert.
\texttt{Erwartetes Ergebnis:} \\Das neue Multicast-Tool zeigt auf der
Benutzeroberfläche und in den Statistiken an, dass es Pakete empfangen hat.

\paragraph{/AVA0500/ (/VA0500/)} Konfigurierbarkeit\\ % Komponententest der Sendekomponente%
\texttt{Abnahmekriterium:} \\ %abstrakt->ohne zahlen, konkret->mit zahlen
\texttt{Ausgangssituation:} \\Es laufen 2 Tools auf 2 verschiedenen Hosts. 
Es sind  Paketgröße, Paket-Senderate, Time-to-live,
Multicast-Gruppe, Port, Nutzdaten, Paketformat festgelegt. Es läuft ein Stream
zwischen Sender und einem Empfänger. Der Empfänger zeigt die Eingestellten Daten
an. 
\texttt{Ereignis:} \\Der Stream wird gestoppt. Die festgelegten Parameter werden
verändert. Der Sender wird wieder gestartet.
\texttt{Erwartetes Ergebnis:} \\Der Empfänger zeigt die Veränderungen der
Parameter an.

\paragraph{/AVA0600/ (/VA0600/)}  Konfigurationsdatei\\ %Komponententest der Konfigurations-Persistierungs-Komponente.%
\texttt{Abnahmekriterium:} \\ (abstrakt)
\texttt{Ausgangssituation:} \\In dem laufenden Multicast-Tool sind konfigurierte
Gruppen vorhanden?
\texttt{Ereignis:} \\Der Benutzer klickt auf den Speichern-Button. Danach wird
das Tool neu gestartet und die gespeicherte Datei wird geladen.
\texttt{Erwartetes Ergebnis:} \\Nach dem speichern muss eine Datei im XML-Format
vorhanden sein. Nach dem laden der Datei müssen alle Gruppen mit den
Einstellungen wiederhergestellt sein.

\paragraph{/AVA0700/ (/VA0700/)} Grafische Nutzeroberfläche\\%Verteiltes Testen der Nutzeroberfläche.%
\texttt{Abnahmekriterium:} \\(abstrakt)
\texttt{Ausgangssituation:} \\Auf einem Linux-Rechner und einem Windows-Rechner
ist jeweils ein Multicast-Tool installiert.  
 \texttt{Ereignis:} \\Es werden über die grafische Oberfläche alle Einstellungen
 für die Streams geändert.
 \texttt{Erwartetes Ergebnis:} \\Es lassen sich alle Funktionen über die GUI
 erreichen und alle Einstellungen (Paketrate,TTL \ldots) der Streams verändern.

\paragraph{/AVA0900/ (/VA0900/)} Textbasierte Nutzeroberfläche\\%Verteiltes Testen, Testeinbindung in Skripten.%
\texttt{Abnahmekriterium:} \\(abstrakt)
\texttt{Ausgangssituation:} \\Das Multicast-Tool ist auf 2 verschiedenen Hosts
installiert und läuft auf einem Host. Außerdem existieren auf dem anderen Host
ein Testskript, das alle Befehle (starten,Konfiguration laden, Streams starten)
enthält und eine Konfigurationsdatei. Die Konfiguratiosdatei enthält einen
Stream auf den ersten Host 
\texttt{Ereignis:}\\Das Testskript wird ausgeführt. 
\texttt{Erwartetes Ergebnis:} \\Das System soll die Befehle des Skriptes
ausführen, dies kann mithilfe des anderen Hosts überprüft werden. Es ist
außerdem eine Logdatei vorhanden.

\paragraph{/AVA1000/ (/VA1000/)} Sendestatistik\\ %Komponententest der Sende-, Logging- und Oberflächenkomponente%
\texttt{Abnahmekriterium:} \\%abstrakt->ohne zahlen, konkret->mit zahlen
\texttt{Ausgangssituation:} \\Es existiert ein Datenstrom zwischen 2
Multicast-Tools.
\texttt{Ereignis:} \\Die Sendestatistik wird geöffnet.
\texttt{Erwartetes Ergebnis:} \\Es werden Azahl der empfangenen Pakete,
Paketrate (beides sowohl pro Sendegruppe als auch insgesamt) und die Sender-ID
angezeigt.

\paragraph{/AVA1100/ (/VA1100/)} Darstellung der Sendeströme\\%Verteiltes Testen der Nutzeroberfläche.%
\texttt{Abnahmekriterium:} \\(abstrakt)
\texttt{Ausgangssituation:} \\Es laufen mehrere Empfänger im Netzwerk. Die
Testinstanz ist so konfiguriert, dass es für jeden Empfänger einen Stream gibt.
\texttt{Ereignis:} \\Die Streams werden gestartet. 
\texttt{Erwartetes Ergebnis:} 
\\ Die Daten der Streams werden auf der Benutzeroberfläche angezeigt und
aktualisiert.

\paragraph{/AVA1200/ (/VA1200/)} Messwertanzeige\\%Verteiltes Testen der Nutzeroberfläche.%
\texttt{Abnahmekriterium:} \\(abstrakt)
\texttt{Ausgangssituation:} \\ Es laufen mehrere Empfänger im Netzwerk. Die
Testinstanz ist so konfiguriert, dass es für jeden Empfänger einen Stream gibt.
\texttt{Ereignis:} \\Die Streams werden gestartet.
\texttt{Erwartetes Ergebnis:} \\Die Empfänger zeigen Messwerte für die
empfangenen Streams an.

\paragraph{/AVA1300/ (/VA1300/)} Datenauswertung\\%Komponententest der Empfangskomponente%
\texttt{Abnahmekriterium:} \\(abstrakt)
\texttt{Ausgangssituation:} \\Im Netzwerk existiert ein Sender und ein
Empfänger. 
\texttt{Ereignis:} \\Der Sender wird aktiviert, sodass ein Datenstrom zwischen
beiden Tools vorhanden ist. 
\texttt{Erwartetes Ergebnis:} \\Der Empfänger zeigt Anzahl und Dauer von Unterbrechungen, Paketrate , Anzahl empfangener Pakete (gesamt, pro MC-Gruppe), Anzahl verlorener
der Pakete und die Anzahl fehlerhafter Pakete an.

\section{Abnahmekriterien f"ur Qualitätsanforderungen}

%------
\paragraph{/ALQF10/ (/LQF10/)} Programm besteht aus einer Datei\\ 
\texttt{Abnahmekriterium:} Die Programmdatei wird auf ein
weiteres System kopiert. Das Programm ist auf dem
Zielsystem weiterhin voll Funktionsfähig.\\

\paragraph{/ALQF20/ (/LQF20/)} Hirschmann-Funktionalität\\ 
\texttt{Abnahmekriterium:} Es werden einige oder alle Funktionalitäten des
Hirschmann-Tools ausgewählt und es wird dann überprüft ob diese auch im
Multicast-Tool vorhanden sind. \\

\paragraph{/ALQF30/ (/LQF30/)} Volle Funktionalität via Kommandozeile\\
\texttt{Abnahmekriterium:} Es werden einige oder alle Funktionalitäten des
Multicast-Tools im GUI-Modus ausgewählt und es wird dann überprüft ob diese auch
im CLI-Modus vorhanden sind.\\

\paragraph{/ALQF40/ (/LQF40/)} Empfangen von Hirschmann-Paketen\\
\texttt{Abnahmekriterium:} Es werden Datenpankete von der
Sendekomponente des Hirschmann-Tools gesendet. Diese Pakete
werden nach Empfang vom Multicast-Tool korrekt erkannt und
verarbeitet.\\

\paragraph{/ALQZ10/ (/LQZ10/)} Absturzsicherheit\\
\texttt{Abnahmekriterium:} Das Programm wird wissentlich
fehlerhaft konfiguriert und läuft dabei trotzdem weiter.\\

\paragraph{/ALQZ20/ (/LQZ20/)} Unabhängigkeit der Programmstabilität von den
Netzwerkkomponenten\\
\texttt{Abnahmekriterium:} Das Multicast-Tool sendet oder empfängt Datenströme
in einem belieben Netzwerk. Das Programm stürzt nicht ab.\\

\paragraph{/ALQZ30/ (/LQZ30/)} Protokollabhängigkeit der Pakete\\
\texttt{Abnahmekriterium:} Es werden gesendete Pakete
stichprobenartig mittels eines entsprechenden Tools betrachtet.
Dabei muss Protokollkonformität herrschen.\\

\paragraph{/ALQZ40/ (/LQZ40/)} Verarbeitung beschädigter Pakete\\
\texttt{Abnahmekriterium:} Es werden wissentlich
beschädigte/nicht-Protokoll-konforme Pakete an das Multicast-Tool
gesendet. Das Multicast-Tool verarbeitet diese bei hinreichender
Ähnlichkeit.\\

\paragraph{/ALQZ50/ (/LQZ50/)} Messgenauigkeit\\
\texttt{Abnahmekriterium:} Es werden Messergebnisse ausgewertet. Die
Messergebnisse sind mit hinreichend genauen Zeitangaben aufgeführt.\\

\paragraph{/ALQU10/ (/LQU10/)} Anlehnung der grafischen Oberfläche an das
Hirschmann-Tool\\ %SRS Verweis!
\texttt{Abnahmekriterium:} Es werden die GUI des Multicast-Tools
sowie die GUI des Hirschmann-Tools gleichzeitig betrachtet. Dabei
erkennt der Betrachter eine hinreichende Ähnlichkeit.\\

\paragraph{/ALQU20/ (/LQU20/)} Darstellung der Multicast-Ströme in einer Liste\\
\texttt{Abnahmekriterium:} Es wird die GUI des Multicast-Tools
betrachtet. Dabei erkennt der  Betrachter eine Auflistung der
Multicast-Ströme.\\

\paragraph{/ALQU30/ (/LQU30/)} De-/Aktivierung der Datenströme per Mausklick\\
\texttt{Abnahmekriterium:} Der Benutzer benötigt nur einen Klick
zur De- bzw. Aktivierung eines Datenstroms\\

\paragraph{/ALQU40/ (/LQU40/)} Optische Trennung Senden/Empfangen\\
\texttt{Abnahmekriterium:} Es wird die GUI des Multicast-Tools
betrachtet. Dabei erkennt der Betrachter eindeutig eine Trennung
zwischen den empfangenen und den gesendeten Multicast-Strömen.\\

\paragraph{/ALQU50/ (/LQU50/)} Intuitive Bearbeitbarkeit einzelner und mehrerer
Datenstromparameter\\
\texttt{Abnahmekriterium:} Der Benutzer kann ohne langes Suchen
die Parameter eines oder mehrerer Datenströme bearbeiten.\\

\paragraph{/ALQU60/ (/LQU60/)} Zweckmäßige Darstellung der Messergebnisse\\
\texttt{Abnahmekriterium:} Der Benutzer betrachtet die
Messergebnisse. Hierbei erkennt der Benutzer hinreichend schnell,
welches Messergebnis welchen Zweck erfüllt.\\

\paragraph{/ALQU70/ (/LQU70/)} Schnelle Erlernbarkeit für erfahrene Benutzer\\
\texttt{Abnahmekriterium:} Ein Benutzer besitzt Wissen zum Netzwerkhintergrund.
Solch ein Benutzer kann sich ohne große Schwierigkeiten mit dem
Programm vertraut machen.\\

\paragraph{/ALQE10/ (/LQE10/)} Auslastung des Arbeitssystems\\
\texttt{Abnahmekriterium:} Es werden 30 Multicast-Ströme zeitgleich
gesendet und empfangen. Dabei darf sich das Arbeitssystem nicht
so langsam verhalten, dass Daten eventuell verloren gehen.\\

\paragraph{/ALQW10/ (/LQW10/)} Modularisierung der Software\\
\texttt{Abnahmekriterium:} Es liegt eine Moduldokumentation vor.\\

\paragraph{/ALQW20/ (/LQW20/)} Modultestbarkeit\\
\texttt{Abnahmekriterium:} Es liegen Modultestberichte vor.\\

\paragraph{/ALQW30/ (/LQW30/)} Definition von Schnittstellen via Interfaces
und Dokumentation in Javadoc-Konvention\\
\texttt{Abnahmekriterium:} Es liegt eine Dokumentation der
Schnittstellen in Javadoc-Konvention vor. Der Quellcode des
Multicast-Tools beinhaltet Interfaces zur
Schnittstellendefinition\\

\paragraph{/ALQW40/ (/LQW40/)} Erweiterbarkeit des Produkts\\
\texttt{Abnahmekriterium:} Es wird eine weitere Funktionalität für das Produkt
gewünscht. Zur Erweiterung wird der Grundaufbau des Programmes nicht
verändert.\\

\paragraph{/ALQP10/ (/LQP10/)} Systemunabhängigkeit\\
\texttt{Abnahmekriterium:} Ein Transfer der Programmdatei
zwischen Windows (ab Windows XP) und Linux findet statt. Die
Programmdatei ist auf dem Quell- sowie auf dem Zielsystem
ausführbar.\\

\paragraph{/ALQP20/ (/LQP20/)} Hardwareunabhängigkeit\\
\texttt{Abnahmekriterium:} Ein Transfer der Programmdatei
zwischen Systemen mit unterschiedlicher Hardwareausstattung
findet statt. Die Programmdatei ist auf beiden Systemen
gleichermaßen ausführbar.\\

\paragraph{/ALQP30/ (/LQP30/)} Einfache Installierbarkeit\\ 
\texttt{Abnahmekriterium:} Die Installation des Tools kann ohne
zusätzliche Software erfolgen. (Java Runtime Environment ausgenommen)\\

\paragraph{/ALQP40/ (/LQP40/)} Konfliktvermeidung mit Software gleicher oder
anderer Funktionalität\\
\texttt{Abnahmekriterium:} Das Multicast-Tool läuft mit anderer
Software auf dem gleichen System. Die andere Software erfüllt
eventuell ähnliche oder gleiche Aufgaben wie das Multicast-Tool.
Keines der beiden Softwareprodukte wird an der Ausführung seiner
Aufgabe durch das andere Softwareprodukt gehindert.\\
