%TODO Faseln
\section{Persistenz}
\label{sec:6:pers}
\subsection{Konfiguration}
Gemäß VA0050 und VA0060 (siehe SRS) muss das System alle konfigurierbaren
Parameter in einer XML-Datei sichern. Für diese XML-Datei wird ein eigener DTD
erstellt. Die Semanthik muss ausdrucksstark genug sein, um eine einfache,
manuelle Bearbeitung zu ermöglichen.

\subsection{Logging}
Hinsichtlich VA0900 und VA1300 (siehe SRS) bietet das System die Möglichkeit
eine Logger-Komponente zu aktivieren, die alle definierten Statistiken über die
Datenströme in einer XML-Datei speichert.

\section{Nutzeroberfläche}
\label{sec:6:ober}
\subsection{Graphisch}
Nach VA0700 und VA0800 muss das gesamte System über eine graphische
Nutzeroberfläche gesteuert werden können. Diese wird in Swing entweder manuell
oder mit GUI-Buildern wie Mantisse erstellt. Ein Prototyp mit den minimalen
Anforderungen ist in Anhang \ref{a:gui} ersichtlich.

\subsection{Textuell}
Das System muss nach VA0900 die Möglichkeit bieten, über die Kommandozeile eine
Konfigurationsdatei zu laden und im Hintergrund zu arbeiten.

\section{Ergonomie}
\label{sec:6:ergo}
\subsection{Performanz}
Keine Nutzerinteraktion darf mehr als 100ms benötigen, bis dass der Nutzer eine
Rückmeldung vom System bekommt.\\
Beim gleichzeitigen Betrieb von 30 Datenströmen dürfen
bei einem handelsüblichen Rechner mit 2,4GHz Intel Core 2 Duo Prozessor und 2GB RAM nicht mehr als 80\%
des Systems ausgelastet werden.

\subsection{Benutzerfreundlichkeit}
Die graphische Nutzeroberfläche muss es ermöglichen, mehre Datenströme
gleichzeitig zu aktivieren, zu deaktiveren und in sinnvollen (nicht
Datenstrom-spezifischen wie Port, Gruppe, \ldots) Attributen zu ändern.\\
Die definierten Daten, die über jeden Datenstrom erhoben werden, müssen auf
einen Blick ersichtlich sein.

\subsection{Automatisierbarkeit}
Das System muss per Kommandozeilenschnittstelle in Skripten eingebunden werden
und im Hintergrund arbeiten können.

\section{Verteilung, Kommunikation mit anderen Systemen, Migration}
\label{sec:6:komm}
Mehrere, voneinander funktional-unabhängige Instanzen des Systems können auf
vielen verschiedenen Knoten eines Netzwerks in Betrieb genommen werden können. Diese
kommunizieren dabei untereinander oder mit dem alten
System der Hirschmann Automation GmbH.

\section{Parallelisierung}
\label{sec:6:para}
Die einzelnen Datenströme werden mit den Mitteln der Java-Plattform
parallelisiert. Die Java-Virtual-Machine entscheidet weiterführend, wie die
Parallelisierung auf Rechner-Ebene umgesetzt wird.

\section{Internationalisierung}
\label{sec:6:inter}
Die Nutzeroberflächen und Ausgaben müssen international betextbar sein. Mit der
Endversion der Software werden die Sprachen Deutsch und Englisch ausgeliefert.
Weiter Sprachen müssen einfach mit Klartext-Dateien hinzugefügt werden können.

\section{Codeverwaltung, Build-Management und Testing}
\label{sec:6:bm}
Der Quelltext des Projektes wird auf dem Unternehmensserver in einem zentralen
Mercurial-Repository verwaltet. Gebaut wird nach Continuous-Integration Prinzip
mit Apache Maven und Apache Continuum als Weboberfläche. Komponententests
werden automatisch mit jedem Build per JUnit V4.8.* durchgeführt, ebenfalls
durch Maven.
