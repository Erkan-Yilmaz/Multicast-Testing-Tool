% This documents structure is based on a template from the following website
% http://www.arc42.de/

\section{Aufgabenstellung}
\label{sec:1:aufg}


Da die Multicast-Fähigkeit nur optional im weit verbeiteten IP-Standard
verankert ist, wird sie von vielen Hard- und Softwarekomponenten nicht richtig
implementiert.\\
Das Multicast-Test-Tool wird es ermöglichen, die Multicasting-Fähigkeiten eines
Netzwerks über IPv4 und IPv6 hinsichtlich Funktionalität, Synchronizität und
Traversierungsdauer auf einen Blick mess- und verifizierbar zu machen.\\
Zu diesem Zweck bietet das Werkzeug die Fähigkeit, mindestens 30
Multicast-Datenströme gleichzeitig zu senden und zu empfangen. Zu jedem dieser
Streams sollen Statistiken über folgende Daten erhoben und verfügbar gemacht
werden:
\begin{itemize}
  \item Empfangene Pakete
  \item Intervalle zwischen dem Empfang zweier aufeinander
folgender Pakete
\item Verlorene Pakete
\item Verzögerungen mit maximaler und
durchschnittlicher Verzögerungszeit
\item Optional: Traversierungszeit von Sender zu Empfänger mit NTP
Synchronisierung
\end{itemize}
Außerdem können geringe Menge von Nutzdaten in Form einer Zeichenkette
übertragen werden.\\

\section{Architekturziele}
\label{sec:1:arch}

Die hier vorgestellte Architektur der Software ist ausschlaggebend für die
folgenden Merkmale:

\paragraph{Effizienz} Das Programm stellt die geforderte Funktionalität durch
lineare und direkte Datenverarbeitungswege ohne unverhältsnismäßig hohe
Rechnerbelastung zur Verfügung. Zur Verhältnismäßigkeit muss beachtet werden,
dass das Senden und Empfangen von vielen Datenströmen speziell mit hohen
Frequenzen durchaus viel Rechenleistung erfordern kann.

\paragraph{Benutzbarkeit} Obwohl sich der Kontext des Programms eher an den
Fachanwender richtet, wird es einfach und übersichtlich zu verwenden sein.

\paragraph{Migrierbarkeit} Um eine schnelle Einführung des Tools zu ermöglichen,
wird zum einen für Kompatibilität mit dem alten Tool der Hirschmann Automation
GmbH gesorgt, zum anderen eine einfache Deployment-Methode gewählt.

\paragraph{Stabilität} Die Architektur beachtet die Tatsache, dass in einem
offenen Netzwerk nicht nur wohlgeformte Pakete an den Empfänger gelangen können.
Das Paket-Dekodierungs System wird diese, nicht dem Protokoll
entsprechenden Pakete erkennen und verwerfen ohne das es zu Fehlern im System
kommt.

\paragraph{Wartbarkeit} Jeder Teil der Architektur strebt hohe
Kohäsion und geringe Kopplung an. Somit sind alle wichtigen Komponenten
individuell austausch- und testbar. Durch häufige Verwendung des
Beobachter-Entwurfsmusters wird eine sehr einfache Erweiterbarkeit des Systems
ermöglicht.

\paragraph{Portabilität} Um das Tool auf allen großen Plattformen verfügbar zu
machen, wird als Implementierungssprache Java in der Version 6 gewählt. Das Tool
basiert komplett auf der Java API sowie Java-nativen, externen Bibliotheken.
Dadurch müssen keine plattform-spezifischen Anpassungen durchgeführt werden.

\clearpage

\section{Stakeholder}
\label{sec:1:stake}

Die folgenden Stakeholder wurden während der Analyse- und Designphase
identifiziert.

\begin{table}[htdp]
\caption{Stakeholder}
\label{tab:stakeholder}
\begin{center}
\begin{tabular}{|c|c|}
\hline
\textbf{Name} & \textbf{Kurzbeschreibung}\\
\hline
Andreas Stuckert & Mitarbeiter Fa. Net-Tools \\
\hline
Markus Rentschler & Mitarbeiter Fa. Net-Tools \\
\hline
Fa. Net-Tools & Auftraggebende Firma \\
\hline
Fa. SPAM & Entwickelnde Firma \\
\hline
Hildenbrand, David (DH) & Projekt Manager\\
\hline
Jedele, Jeffrey (JJ) & Leitender Ingenieur\\
\hline
Safarpour, Ramin (RS) & Systemtest\\
\hline
Schoknecht, Tobias (TSC) & Dokumentation\\
\hline
Stöckel, Tobias (TST) & Produkt Manager\\
\hline
Weitz, Konstantin (KW) & Leitender Ingenieur\\
\hline
\end{tabular}
\end{center}
\label{default}
\end{table}
