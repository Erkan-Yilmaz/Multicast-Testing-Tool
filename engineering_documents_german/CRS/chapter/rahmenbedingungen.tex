%owner jeff

\chapter{Rahmenbedingungen}
\label{cha:rahm}

\section{Organisatorische Rahmenbedingungen}
\label{sec:orga}

\paragraph{Anwendungsbereich:} Netzwerkaufbau, -analyse, -verifikation

\paragraph{Zielgruppen:} Systemtester, Analysten von Netzwerk Hardware bzw.
Netzwerk Software. Die Software wird eingesetzt um die
Multicasting-Fähigkeit des Netzwerks zu analysieren und Aussagen über
Gleichlauf und Zuverlässigkeit machen zu können.

\paragraph{Betriebsbedingungen:} Computer-Netzwerke jeder Ausprägung auf
IP-Basis

\section{Technische Rahmenbedingungen}
\label{sec:tech}

\paragraph{Software:} Die Software unterstützt Windows (ab XP) und Linux. 
Verwendung unter anderen Betriebssystemen mit kompatiblem Java
Runtime Environment und IP-Stack ist prinzipiell möglich, wird aber nicht
garantiert. Ein Java Runtime Environment der Version 6 muss verfügbar sein. Für
die Nutzung der graphischen Nutzeroberfläche ist ebenfalls eines der
Standardfenstersysteme der unterstützen Betriebssysteme notwendig (Windows
Fenstersystem, X.org, Qwartz).

\paragraph{Hardware:} Minimale Anforderung ist ein handels"ublicher Desktop
Computer bzw. Notebook mit Netzwerkkarte(mind. 100mbit/s). Der Computer muss
lokal oder per Netzwerk steuerbar sein.

\paragraph{Orgware:} Ein Anwender kann sowohl Client als auch Server zur selben Zeit darstellen. Eine Netzwerkverbindung zum selben LAN vom Sender auch als Empf"anger ist f"ur das Testen erforderlich.

\section{Anforderungen an die Entwicklungsumgebung}
\label{sec:anf}

\paragraph{Software} Zur Entwicklung eignet sich jedes aktuelle Betriebssystem
mit verfügbarem Java Development Kit der Version 6. Als Entwicklungsumgebung
muss Eclipse in aktueller Version verfügbar sein. In Betriebssystem sowie
Entwicklungsumgebung sollte das verteilte Versionierungssystem Mercurial zum
Verwalten von Systemkomponenten und -dokumenten verfügbar sein. Zugang zum
Internet, sowie ein aktueller Webbrowser sind notwendig, um auf Versionierung
und Projektverwaltung zuzugreifen. Die UML-Software VisualParadigm ist notwendig zur graphischen
Modellierung des Systems.

\paragraph{Hardware} Anforderungen an die Hardware sind identisch zu den
oben genannten.

\paragraph{Orgware} Aufgaben werden online in dem Projekt-Organisations-Werkzeug
Redmine verwaltet. Zugang zu diesem ist für produktive Entwicklung notwendig.
Die graphische Modellierung des Systems erfolgt in UML. Alle Dokumente und
Softwareteile des Systems werden mit dem verteilen Versionierungssystem
Mercurial verwaltet.
