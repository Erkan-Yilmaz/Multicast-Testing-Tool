% owner: jeff
\chapter{Kontext und "Uberblick}
\label{cha:kont}

\paragraph{Systemumgebung}
Das System wird in einem bereits bestehenden Computer-Netzwerk verwendet. Zu
diesem gehören sowohl die einzelnen Computer als Netzwerkknoten, als auch
sämtliche Peripherie zum Verbinden dieser Knoten untereinander. Das System
ist auf Empfangsseite mit älteren Versionen
der MultiCast-Testing-Software kompatibel.

\paragraph{Systemgrenze}
Das System ist klar auf die Anwendungsschicht beschränkt. Es sollen weder auf
Hardware- noch Betriebssystemebene neue Multicasting-Fähigkeiten implementiert, sondern ausschließlich bereits vorhande Implementierungen genutzt und
getestet werden. Interaktion mit anderen Softwarekomponenten auf
Schnittstellenebene ist nich vorgesehen.

\paragraph{Kontext}
\begin{figure}
\includegraphics[width=10cm]{images/kontext-usecase.png}
\centering
\caption{Grundsätzlicher Anwendungsfall}
\end{figure}
Das System wird auschließlich in lokalen Netzwerken (LAN) verwendet - eine
Verwendung über Internetverbindungen hinweg ist nicht vorgesehen.
Empfangsintervall- und Traversierungszeiten sind in Auflösung von Millisekunden
zu ermitteln. Da Multicast fest im IP-Protokoll spezifiziert ist, wird davon
ausgegangen, dass alle am Netzwerk beteiligten Komponenten dessen Methoden
vollständig unterstützen. Dazu gehört auch auch das
Multicast-Gruppenverwaltungsprotokoll IGMP. Auch bei diesem wird davon
ausgegangen, dass alle beteiligten Netzwerkkomponenten es unterstützen. Sollte
dies nicht der Fall sein, wird kein Mutlicastverkehr stattfinden.
