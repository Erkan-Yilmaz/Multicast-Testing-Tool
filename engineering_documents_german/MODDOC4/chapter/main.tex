
\chapter{Scope}
Die Moduldokumentation beschreibt die Architektur, die Schnittstellen 
und die Hauptmerkmale des Moduls. Außerdem werden die Modul bzw. Komponententests 
einschließlich der Ergebnisse beschrieben und dokumentiert. 
Die MOD dient bei Bedarf auch als Programmier- oder Integrationshandbuch für das 
Modul. Wenn bestimmte Risiken direkt mit der Verwendung des Moduls verknüpft sind,
so sind sie in diesem Dokument zu benennen und zu kommentieren.

\section{Abkürzungen}

	\begin{description}
	\item[GUI] Graphical User Interface
	\item[CLI] Command Line Interface
	\end{description}

\chapter{Analyse}
%Dieses Kapitel muss alles beinhalten, was noch nicht in anderen Dokumenten enthalten ist.
Die CLI wird verwendet um automatisiert tests durchführen zu können. Dies
bedeutet, dass der User des Testtool wahrscheinlich mittels Skripten startet und
diese dann mithilfe vordefinierter Profile Tests durchführen. Um die Ergebnisse
nachzuvollziehen müssen diese in die Konsole ausgegeben werden. Außerdem werdeen
zur Fehleranalyse die Ereignisse (insbesondere Fehler) und Statistiken geloggt.\\
\\
Das Modul reagiert lediglich auf Ereignisse des Controllers(Exceptions,
Statistik Updates) und schreibt die Ergebnisse in ein Logfile sowie in die
Konsole. 

\section{Voruntersuchungen}
%Müssen/mussten Prototypen gemacht werden?
%Versuche, Test und Abschätzungen 

\subsection{Buffered Writer + Standardoutput}
Der erste Versuch war eine Kombination der BufferedWriter Klasse, welche zum
loggen der Ereignisse zuständig war und des Standardoutputs zur Ausgabe in die
Konsole. Diese Kombination hat sich als äußerst ungünstig herausgestellt, da
z.B. die Formatierung der Ausgaben jedes mal erneut in einen String geschrieben
werden muss. Außerdem muss je Ausgabe (hier in Log-File und Konsole) ein
Funktionsaufruf getätigt werden, was zu schlechtem Code führt.

\subsection{Log4J}
Log4J ist ein Logging-Framework der Apache Foundation und eine sehr elegante
Lösung für die Aufgaben des Moduls. Jede Ausgabe kann über Appender hinzugefügt
werden. Diese werden dann alle durch den Aufruf einer Methode geschrieben.
Durch Angabe eines Pattern Layouts ist die Ausgabe auch immer Einheitlich
formatiert. Dies führt zu übersichtlicheren Log-Files.

\subsection{Internationalisierung}
Für die Internationalisierung greift der Logger auf denselben Mechanismus
zurück, auf den auch die GUI zugreift. Dies geschieht zum einen um
Einheitlichkeit zu gewährleisten aber auch, da der Mechanismus für das Modul
optimal ist.

\section{Systemanalyse}

\subsection{Rahmenbedingung}
Das Modul ist Schnittstelle zum User, indem es Ausgaben bereitstellt. Die
Ereignisse, welche augegeben werden müssen erhält das Modul als Observer vom
Controller, welche auch Nutzereingaben umsetzt. Das Modul muss über alle
wichtigen Ereignisse, besonders Fehler Auskunft geben. Zur Fehlerbehebung gibt
das Tool auch exakten Zeitpunkt und die zugehörige Session an.

\subsection{Problemstellung}
Die einzige Funktion des Moduls ist das Ausgeben der Ereignisse an den User. Die
Verarbeitung von Usereingaben ist Aufgabe des Controllers. Dies kann sowohl in
deutscher Sprache als auch auf Englisch erfolgen. Mögliche Eingaben werden vom
Controller bearbeitet und vom Logger/CLI ausgegeben.

\subsection{Abhängigkeiten}
Die Funktionalität des Moduls ist abhängig von den Ereignissen, welche der
Controller kommuniziert. Es ist das einzige Modul, mit dem der Logger/CLI in
Verbindung steht.

\chapter{Design}

\subsubsection{Folgerungen aus der Analyse}
Für die Ausgaben wird Log4J verwendet, was eine übersichtliche Implementierung
der Fähigkeiten des Modus ermöglicht.

\subsubsection{Systemarchitektur}
Die Architektur besteht aus einer einzigen Klasse, welche die durch den
Controller gemeldeten Events ausgibt. Sie kontrolliert die Ausgabe in den
Log-File und in die Konsole.

\subsection{Schnittstellen}
Die Klasse hat zwei Methoden, welche
vom Controller aufgerufen werden um das CLI zu starten/beenden. Damit
registriert sich die Klasse als Observer bzw. entfernt sich als Observer.
Außerdem gibt es für jedes festzuhaltende Ereignis eine Methode, welche die
entsprechenden Informationen ausgibt.

\chapter{Implementierung}

\subsection{Initializing}
Die init Methode registriert die Klasse als Observer für alle wichtigen Events.
Dies geschieht über dafür vorhergesehene Methoden des Controllers. Bereits hier
gibt das Modul die erste Nachricht aus.

\subsection{Event Listener}
Für alle relevanten Ereignisse existiert ein Event Listener, welcher die Art des
Events und eine Nachricht bzw. Verursacherobjekt (z.B. hinzugefügten Sender)
ausgibt.

\subsection{Statistic Update}
Hierbei handelt es sich eigentlich ebenfalls um Listener. Der Unterschied zu den
anderen Listenern ist, dass die Ausgabe nur in festen Zeitintervallen erfolgt,
also nicht bei jedem Aufruf. Grund hierfür ist, dass die Updates nahezu
sekündlich eintreffen und eine so hohe Auflösung eher störend als hilfreich ist.

\subsection{Beenden}
Die Kill Methode dient dazu den Logger/CLI geordnet zu beenden. Dies ist zum
einen das entfernen als Listener im Controller über dafür vorgesehene Methoden
und andererseits das loggen des Sessionendes.

\subsection{Codedokumentation}
Die Klasse wurde mittels Javadocs dokumentiert.

\chapter{Komponententest}

Das Modul wurde mittels JUnit Tests und manueller Kontrolle getestet. Hierbei
wurden Ereignisse im Controller über JUnit simuliert und anschließend der
Logfile und die Konsle überprüft.

\chapter{Zusammenfassung}

\section{Beurteilung der Komponente}
%Hat die Komponente Schwächen und Stärken, 
%Besonderheiten?
%Übertragbarkeit der Komponente und Lösungen
%Was ist verbesserungswürdig?
Das CLI/Logger Modul ist ein kleines Modul. Dadurch und, da es ausschließlich
mit dem Controller kommuniziert ist der Code automatisch sehr übersichtlich
geblieben. 

Dennoch ist das Modul stark erweiterungsfähig. (s. Ausblick für die
Weiterentwicklung)



\section{Ausblick für die Weiterentwicklung}
%Sind schon Dinge vorbereitet?
%Erweiterungsmöglichkeiten?
Bisher nicht betrachtet aber dennoch relativ einfach zu Implementieren wären
mehr Angaben in Fehlerfällen bzw. zu den Ereignissen. Diese Informationen stehen
dem Modul größtenteils zur Verfügung müssten aber noch konsequenter und in
Übersichtlicher Form ausgegeben werden.