
\chapter{Scope}
Die Moduldokumentation beschreibt die Architektur, die Schnittstellen 
und die Hauptmerkmale des Moduls. Außerdem werden die Modul bzw. Komponententests 
einschließlich der Ergebnisse beschrieben und dokumentiert. 
Die MOD dient bei Bedarf auch als Programmier- oder Integrationshandbuch für das 
Modul. Wenn bestimmte Risiken direkt mit der Verwendung des Moduls verknüpft sind,
so sind sie in diesem Dokument zu benennen und zu kommentieren.

\chapter{Definitionen}

    Auf der Empfängerseite der Anwendung herrscht eine gewisse Diskrepanz der
    Begrifflichkeiten zwischen Benutzer- und Entwicklersicht, die in Tabelle
    \ref{tab:receiver_vs_group} klargestellt wird. In diesem Dokument wird
    in der Regel auf die Benutzersicht Bezug genommen
    
    \begin{table}[h]
        \caption{ReceiverGroup/Empfänger vs. Receiver/Datenstrom}
        \label{tab:receiver_vs_group}
        \begin{center}
            \begin{tabular}{|p{3.2cm}|p{3.5cm}|p{7.5cm}|}
                \hline
                    \textbf{Entwicklersicht} &
                    \textbf{Benutzersicht} &
                    \textbf{Definition}\\
                \hline
                    ReceiverGroup & Empfänger & Ein auf Empfängerseite
                    eröffneter Multicast-Socket, der genau einer Gruppe
                    beigetreten ist und an einem bestimmten Port lauscht.\\
                \hline
                    Receiver & (eingehender) Datenstrom & Das Gegenstück zu
                    einem von \emph{einem} Sender ausgehenden Datenstrom. Ein
                    Empfänger kann mehrere Datenströme empfangen, da ja
                    meherere Sender in die selbe Gruppe senden können.\\
                \hline
            \end{tabular}
        \end{center}
    \end{table}

%\section{Abkürzungen}

%\section{Definitionen}
    %Wichtige Begriffe und Worte

\section{Terminologien}
    %Besondere Begrifflichkeiten und deren Verwendung.
    %Festgelegte Terminologien und Regeln für Schreibweise z.B. Notationsegeln
    % für CLI Kommandos
    
    \begin{description}
        \item[View-Controller] Der View-Controller ist die GraphicalView-Klasse
        und wird in Moduldokumentation "`GUI-Logik"' näher beschrieben.
    \end{description}


\chapter{Analyse}
    %Dieses Kapitel muss alles beinhalten, was noch nicht in anderen Dokumenten
    % enthalten ist.

    Die grafische Oberfläche (GUI) stellt die wichtigste Schnittstelle des
    Multicast Test Tools zum Endanwender dar. Die wahrgenommene Qualität der
    gesamten Anwendung hängt in hohem Maß von der Bedienbarkeit über die GUI ab.
    Alle vom Endanwender formulierten Use Cases müssen auf möglichst intuitive
    Weise abgebildet werden.
    
    Der Analyse der Designanforderungen an das Multicast Test Tools lagen die in
    CRS und SRS explizit spezifizierten Use Cases zugrunde. Um die
    Praxistauglichkeit der Anwendung zu maximieren, lag hierbei ein besonderers
    Augenmerk auf der übersichtlichen Darstellung von Messergebnissen.
    
    Die hier vorgestellten Voruntersuchungen bauen größtenteils auf die
    im Dokument ``Moduldokumentation GUI-Logik'' vorgestellten Voruntersuchungen
    auf.

    \section{Voruntersuchungen}
        %Müssen/mussten Prototypen gemacht werden?
        %Versuche, Test und Abschätzungen 
        
        Die grafische Oberfläche wurde mithilfe des Swing-Toolkits umgesetzt.
        Besonders die Funktionsweise und der korrekte Einsatz der
        Swing-Layout-Manager erforderte gründliche Voruntersuchungen und
        Analysen. Layout-Manager sind dafür verantwortlich, GUI-Komponenten
        anzuordnen. Dazu gehört insbesondere, dass die grafische Oberfläche auch
        bei Größenänderungen der Fenster ein korrektes und ausgewogenenes
        Erscheinungsbild beibehält.
        
        Für das GUI-Prototyping und die Erzeugung eines Großteils des GUI-Codes
        wurde der Netbeans GUI-Designer eingesetzt. Dessen Funktionsweise und
        Einsatzpotential war ebenfalls Objekt gründlicher Voruntersuchungen.
        Insbesondere die Fähigkeit, den generierten Code automatisch zu
        internationalisieren, führte zur Entscheidung, den Netbeans GUI-Designer
        einzusetzen.
        
        Um die oben erwähnte übersichtliche Anzeige der Datenströme zu
        gewährleisten, wurde ein Großteil der Voruntersuchungen auf die
        Arbeitsweise und Anpassungsmöglichkeiten der Swing-JTable verwendet.
     
    \section{Systemanalyse}
    
        \subsection{Rahmenbedingungen}
        
            Die zentrale GUI-Logik wurde in die GraphicalView-Klasse
            ausgelagert. Diese Klasse ist auch die Schnittstelle zwischen den
            Design-Elementen und der Funktionalität der Anwendung.
            
            Die Struktur des Codes für die grafische Oberfläche ist weitgehendst
            durch den Netbeans GUI-Designer vorgegeben. Große Teile dieses Codes
            dürfen auch von Hand gar nicht geänder werden, um die Kompatibilität
            mit dem Designer aufrechtzuerhalten.
            
            Besonders beim Code für die visuellen Komponenten der Oberfläche,
            ist die Plattformunabhängigkeit zu beachten. Im Allgemeinen kann
            nicht von einer einheitlichen Darstellung der Komponenten auf
            verschiedenen Betriebssystemen ausgegangen werden. Dies gilt
            besonders dann, wenn jeweils das systemeigene ``Look-and-Feel''
            verwendet wird. Für eine ausführliche Diskussion der
            Swing-Look-and-Feels sei hier auf die offizielle Java-Dokumentation
            verwiesen.
        
        \subsection{Problemstellung}
    
            %Analyse der Problemstellung 
            
            \subsubsection{Grundsätzliche Anforderungen}
            
                Hauptaufgabe der GUI ist die übersichtliche Darstellung von
                Sende- und Empfangsströmen. Dazu kommt die Darstellung globaler
                Statistikdaten.
                
                Neben der korrekten Darstellung ist auch die Bedienbarkeit der
                GUI ein entscheidendes Erfolgskriterium. Die Oberfläche muss von
                einem technisch orientierten Anwender intuitiv bedienbar sein.
                
                Auch plattformübergreifend muss eine angemessene Darstellung
                erreicht werden.
                
            \subsubsection{Tabellen}
            
                Das Kernstück der GUI sind die beiden Tabellen, die jeweils
                Sende- und Empfangsströme darstellen. Gemäß CRS muss hier der
                Status, sowohl eines einzelnen Senders oder Empfängers, als auch
                im groben Überblick der des gesamten Systems, auf einen Blick
                erkennbar sein.
                
                Beide Tabellen müsse sich nahtlos in die restlichen,
                ``einfachen'' Swing-Komponenten eingliedern.
            
            \subsubsection{Dialoge}
           
           		Eine Erweiterung zum Hauptfenster der GUI stellen die Dialoge dar,
           		die Teilaufgaben (z.B. Speichern/Laden einer Konfiguration)
           		übernehmen und diese Übersichtlich darstellen müssen.

        %Architektur und Gliederung der Komponente 
        
        \subsection{Abhängigkeiten}
            %Interaktionsanalyse / Abhängigkeiten 
            Der gesamte GUI-Design-Code ist stark von der GUI-Logik abhängig und
            verschmilzt an vielen Stellen auch damit. Direkte Abhängigkeiten zum
            Model wurden aber durch das Zwischenschalten eines View-Controllers
            deutlich gesenkt. In der Funktionalität spiegelt sich diese
            Abhängigkeit dann aber naturgemäß wieder, da die View ja den
            gesamten Funktionsumfang des Models abzudecken hat.
            
        
 


\chapter{Design}
    %Folgerungen aus der Analyse 
        
    \section{Lösungen für den Problembereich der Analyse}
        %Lösungen für den Problembereich der Analyse 
        
        \subsection{Hauptfenster}
        
            Um die geforderte Übersicht zu gewährleisten, ist das Hauptfenster
            eindeutig zweigeteilt. Die obere Hälfte dient der Darstellung der
            Sendefunktionalität der Anwendung, die Untere der Darstellung der
            Empfangsfunktionalität. Diese logische Trennung wird durch optische
            Merkmale (ein-/ausgehender Pfeil, JSeparator) unterstrichen.
            
            Jeder der beiden Bereich ist wiederum zweigeteilt: Eine große
            Tabelle zur Darstellung der Datenströme und ein Teilbereich am Rand,
            der globale Statistiken anzeigt.
            
            Um die Ergonomie zu verbessern werden für einzelne Benutzeraktionen
            verschiedene Zugriffsmöglichkeiten bereitgestellt. So ist
            beispielsweise das Entfernen eines Datenstroms je nach
            Benutzervorliebe entweder über Buttons im Hauptfenster, über einen
            Kontextmenüeintrag, oder durch Druck der Entf-Taste möglich.
            Zusätzlich kann die Oberfläche des Hauptfensters auch an den
            momentanen Einsatzzweck angepasst werden, indem z.B. die
            Sendertabelle komplett ausgeblendet wird, wenn die Anwendung nur als
            Empfänger fungiert.
            
            Um die Plattformunabhängigkeit der Darstellung sicherzustellen, wird
            auf die ``Look-and-Feel''-Mechanismen des Swing-Toolkits
            zurückgegriffen. Mit Einbinden des jeweils plattformspezifischen
            System-Look-and-Feel wird zwar eine hundertprozentig einheitliche
            Darstellungen auf allen Zielsystemen bewusst verhindert, dafür fügt
            sich das Tool dann ergonomisch besser in das jeweilige
            Betriebssystem ein. Das Swing-Toolkit sorgt dabei für eine
            eindeutige und verlustfreie  Übertragung der Bedeutung einzelner
            GUI-Komponenten auf die jweilige Plattform.
            
            Um die intuitive Bedienbarkeit der Oberfläche aufzuwerten, wurde
            allen wichtigen Steuerelementen ein erläuternder Tooltip
            hinzugefügt. Ebenso wurden die Spaltenköpfe der Tabellen mit
            Tooltips versorgt, da sich deren Bedeutung nicht immer eindeutig,
            aus dem Spaltentitel erschließt.
        
        \subsection{Tabellen}
        
            Um eine möglichst intuitive Darstellung der Datenströme zu
            gewährleisten, werden in beiden Tabellen Icons verwendet, die den
            Status eines Datenstromes widerspiegeln. Zur besseren Lesbarkeit
            werden außerdem die Zeilen mit alternierenden Hintergrundfarben
            dargestellt. Beide Tabellen werden durch ihre Data Models regelmäßig
            mit aktuellen Daten versorgt.
            
            Als Besonderheit bietet die Empfängertabelle die Möglichkeit ganze
            Empfänger ``einzuklappen'', also die Datenströme, die ein Empfänger
            empfängt, auszublenden. Der Mechanismus, der für den konsistenten
            Zustand der Tabellendaten, auch bei eingeklappten Zeilen, sorgt,
            wird größtenteils vom Data Model umgesetzt. 
        
        \subsection{Dialoge}
        
        	Zur Gewährleistung einer übersichtlichen und intuitiven Darstellung
        	werden Teilfunktionen des Programms in Dialog-Fenster ausgelagert.
        	Dies beinhaltet z.B. Sender oder Empfänger hinzufügen, entfernen oder
        	bearbeiten.
        	Dabei werden in dem entsprechenden Fenster nur die relevanten
        	Textfelder angezeigt, die sich z.B. zwischen Sender und Empfänger
        	unterscheiden, da Sender eine größe Anzahl an
        	Konfigurationsmöglichkeiten bieten.
        	
        	Im Falle der Statistik-Dialog-Fenster werden dann nur die Informationen
        	angezeigt, die zu dem entsprechenden Datenstrom gehören.
        	Weiterhin ist anzumerken, dass die Statistik-Fenster zur visuellen
        	Darstellung zusätzlich eine Box enthalten, auf der ein Graph der
        	aktuell gemessenen Paketrate dargestellt wird.

\chapter{Implementierung}

    \section{Hauptfenster}
    
        Das Hauptfenster wurde als vom Swing-JFrame abgeleitete Klasse
        implementiert. Die Layout-Manager wurden im Fall des Hauptfensters nicht
        vom Netbeans GUI-Designer, sondern von Hand konfiguriert. Durch diese
        Vorgehensweise konnte eine höhere Präzision in der Darstellung
        hinsichtlich Größenänderungen und Plattformunabhängigkeit erreicht
        werden.
        
        Alle ActionListener der GUI-Komponenten wurden als anonyme innere
        Klassen implementiert. Dadurch bleibt der Code kompakt und der
        Eventfluss kann besser nachvollzogen werden. Alle Benutzerereignisse
        werden somit zunächst in der Hauptfensterklasse selbst behandelt und
        gegebenenfalls an den View-Controller, oder die Dialoge weitergeleitet.
        
        Die zentralen Komponenten des Hauptfensters sind die Sender- und
        Empfängertablle, die aufgrund ihrer Komplexität jeweils in ein eigenes
        Package ausgelagert wurden.
        
        Das Hauptfenster stellt auch das Datei- und Hilfe-Menü, sowie das
        Kontextmenü von Sender- und Empfängertabelle zur Verfügung.
    
    \section{Sendertabelle}
    
        Die Sendertabelle ist die einfachere der beiden Tabellen. Sie beerbt die
        JTable-Klasse, um die spezielle Konfiguration der Tabelle zu kapseln.
        Dazu gehört das Setzen des Zellen-Renderers für die Icon-Darstellung der
        Status-Spalte und für die alternierenden Hintergrundfarben der
        Tabellenzeilen.
        
        Auch die Spaltendefinitionen und deren internationalisierte Spaltentitel
        und Tooltips sind in diesem Package als Enum-Typ hinterlegt.
        
        Der Sendertabelle liegt ein eigenes TableModel zugrunde, welches die
        Verwaltung der anzuzeigenden Sendeströme vornimmt.
        
        Die Sendertabelle ermöglicht das Auswählen eines oder mehrerer Sender,
        wovon dann auch die im Hauptfenster zur Verfügung stehenden Aktionen
        abhängen (Sender de-/aktivieren/löschen/\ldots).
        
        Eine Übersicht über die Architektur der Sendertabelle liefert Diagramm
        \ref{jsendertable}.
    
    \section{Empfängertabelle}
    
        Als größere Herausforderung stellte sich die Implementierung der
        Empfängertabelle dar. Auch sie ist von der JTable-Klasse abgeleitet, vor
        Allem um zeilenweise unterschiedliche Renderer definieren zu können.
        Dies ist notwendig, um eine Empfängerzeile anders darstellen zu können,
        als eine Datenstromzeile.
        
        Auch in der Empfängertabelle wird die Datenhaltung von einem eigenen
        TableModel verwaltet. Dieses ist in diesem Fall von der
        AbstractTableModel-Klasse abgeleitet, um eine größere Flexibilität in
        der Umsetzung zu erreichen. Das TableModel verwaltet sowohl die Daten
        der dargestellten Empfänger, als auch die Ein- und
        Ausklappfunktionalität. Sämtliche Logik (bis auf das Durchreichen von
        Mausklicks) wurde damit aus der darstellenden Komponente entfernt und in
        das TableModel verlagert. Die JReceiverTable greift damit transparent
        per Spalten- und Zeilenangaben auf des TableModel zu und entscheidet je
        nach zurückgegebenem Zeilentyp, welcher Renderer zum Einsatz kommt.
        
        Wie bei der Sendertabelle sind auch hier die Spaltendefinitionen und deren internationalisierte Spaltentitel
        und Tooltips sind in diesem Package als Enum-Typ hinterlegt.
        
        Etwas tiefer in die Swing-Trickkiste greift der Renderer für die
        Empfängerzeile (ReceiverGroupRow). Er manipuliert gezielt die
        Swing-Zeichenmethode, um ein Spaltenübergreifendes Aussehen zu
        erreichen.
        
        Auch die Empfängertabelle ermöglicht das Auswählen einer oder mehrerer
        Zeilen. Abhängig davon, ob nach einer Selektion, nur Empfänger, nur
        Datenströme, oder beides markiert ist, werden die entsprechenden
        Aktionen im Hauptfenster und im Kontextmenü de-/aktiviert.
        
        Eine Übersicht über die Architektur der Empfängertabelle liefert
        Diagramm \ref{jreceivertable}.
    
    \section{Dialoge}
    
    	Die Dialoge sind von der JDialog-Klasse abgeleitet und erhalten die
    	angezeigten Daten (sofern es sich nicht um Standarderte handelt) bei
    	erstellen des jeweiligen Dialogs über eine Referenz zum gewünschten Objekt
    	(Sender/Empfänger).
    	
    	Die "`Bearbeiten"'- und "`Hinzufügen"'-Dialoge zeigen die Daten dabei
    	in Textfeldern oder aufklappbaren Listen an um eine Bearbeitung der Daten
    	zu ermöglichen, wohin gegen die Statistik-Dialoge lediglich eine
    	Anzeige-Funktion mittels JLabels besitzen. 
    	
    	Gesondert hierzu muss der "`Traffic-Graph"' erwähnt werden, der eine
    	grafische Darstellung der gemessenen Paketraten darstellt.
    	Der "`Traffic-Graph"' ist von JPanel abgeleitet und überschreibt darin die
    	paintComponent-Methode zum Zeichnen eines grauen Rasters, eines schwarzen
    	Hintergrundes und eines grünen Graphs.
    	Die aktuellen Messdaten werden hierbei über das
    	Statistik-Fenster übergeben, wenn dieser ein DataChanged-Event erhält und
    	damit die Felder auf der GUI aktualisiert.
    	
    	Der Optionen-Dialog bietet die Möglichkeit die Sprache umzustellen. Nach
    	Bestätigen der Auswahl wird die Benutzeroberfläche dann mit der
    	ausgewählten Sprache neu geladen.
    	
    	Weiterhin gibt es Dialoge zum Laden und Speichern von Konfigurationen, die
    	über einen einfachen oder einen (zum Speichern mit Profilnamen) erweiterten
    	JFileChooser umgesetzt wurden. Außerdem existiert ein AboutDialog, der das
    	MCTOOL als Software von SPAM ausweist.
    	
    	Eine Besonderheit im Vergleich zu den anderen Dialogen stellt der
    	Error-Dialog dar. Dieser wird nicht wie die anderen Dialoge durch den
    	Benutzer geöffnet, sondern resultierend z.B. durch die Eingabe eines
    	Benutzers vom Programm selbst im Fehlerfall aufgerufen. Hierbei wird der
    	aktuelle Fehler in einer Liste angezeigt. Sollten weitere Fehler während
    	des Ausführens des Dialogs auftreten, so werden diese an die Liste
    	angehängt ohne einen weiteren Dialog zu öffnen.
    
    \section{Codedokumentation}
    
        Mit Ausnahme der automatisch generierten Methoden, sind die
        Design-Komponenten nach Javadoc-Konvention dokumentiert.

% \chapter{Komponententest}
%     
%     Die korrekte Funktionsweise der Design-Komponenten wurde im Rahmen der
%     Continouus Integration projektbegleitend von Hand getestet. Automatisierte
%     Testverfahren wurden zwar evaluiert, deren konsequenter Einsatz erwies sich
%     jedoch als nicht wirtschaftlich.
%     
% \section{Komponententestplan}
% 
% \begin{table}[h]
% \caption{Komponententestplan}
% \label{tab:ktp}
% \begin{center}
% \begin{tabular}{|p{2cm}|p{3cm}|p{9cm}|}
% \hline
% \textbf{Test Nr.} & \textbf{Feature ID} & \textbf{Test Spezifikation}\\
% \hline
%  & & \\
% \hline
%  & & \\
% \hline
%  & & \\
% \hline
%  & & \\
% \hline
% \end{tabular}
% \end{center}
% \end{table}
% 
% \section{Komponententestreport}
% 
% \begin{table}[h]
% \caption{Komponententestreport}
% \label{tab:ktr}
% \begin{center}
% \begin{tabular}{|p{2cm}|p{2cm}|p{6cm}|p{1.5cm}|p{1.5cm}|}
% \hline
% \textbf{Test Nr.} & \textbf{Pass/Fail} & \textbf{Test Ergebnis (wenn Fail)} & \textbf{Datum} & \textbf{Tester}\\
% \hline
%  & & & &\\
% \hline
%  & & & &\\
% \hline
%  & & & &\\
% \hline
%  & & & &\\
% \hline
% \end{tabular}
% \end{center}
% \end{table}

\chapter{Zusammenfassung}

\section{Beurteilung der Komponente}

    %Hat die Komponente Schwächen und Stärken, 
    Der Einsatz eines Code-Generators bringt naturgemäß einen großen Umfang an
    häufig unübersichtlichen und schlecht wartbaren Code mit sich. Auch in
    diesem Fall leidet die Komponente sichtbar unter diesem Problem. Zusätzlich
    dazu ist der Entwickler auch in seiner Flexibilität eingeschränkt, da große
    Codepassagen nicht von Hand nachgebessert werden können. Das den im Moment
    verwendeten Listener-Methoden überlegene Konzept der Java-Actions wurde vom
    Entwickler leider zu spät "`entdeckt"'. Gerade Benutzeraktionen, die an
    mehereren Stellen der grafischen Oberfläche ausgelöst werden können, hätten
    damit sehr elegant umgesetzt werden können.
    
    Auf der anderen Seite hat sich der Einsatz des Netbeans GUI-Designers aber
    eindeutig bezahlt gemacht. Trotz der oben genannten Einschränkung erwies er
    sich als sehr flexibles Werkzeug. An vielen Stellen kann in die
    Codegenerierung eingegriffen werden, das Refactoring gestaltet sich sehr
    angenehm und auch eigene, komplexe Komponenten, wie die Sender- oder
    Empfängertabellen, lassen sich ohne Probleme in das Design integrieren. Auch
    die automatische Internationalisierungsfunktion hat den Entwicklungsprozess
    deutlich effizienter gemacht.
    
    Auch wenn die Empfängertabelle nun auf die Darstellung von Multicaststreams
    zugeschnitten wurde, so ist das mit ihr entwickelte Konzepte einer Tabelle
    mit ein- und ausklappbaren Untergruppen doch mit Sicherheit auch auf
    zukünftige Komponenten übertragbar.
    
    
    

\section{Ausblick für die Weiterentwicklung}
%Sind schon Dinge vorbereitet?
%Erweiterungsmöglichkeiten?

    Im Gesamtkonzept ist die View und vor Allem ihr Design eine
    Anwendungskomponente, die auf komplette Austauschbarkeit setzt. Wenn nur
    gerfingfügige Änderungen vorgenommen werden sollen, ist es aber dennoch
    wichtig, dass der Code trotzdem wart- und erweiterbar bleibt. Vor diesem
    Hintergrund ist positiv zu vermerken, dass der gesamte
    Design-Code nach wie vor Netbeans-GUI-Designer-tauglich ist. Eine eventuelle
    Weiterentwicklung des Designs wird damit deutlich vereinfacht, da eine
    aufwändige Einarbeitung in den GUI-Code nahezu komplett entfällt.
    
    Deutlich aufgewertet würde die Verknüpfung zwischen GUI-Design und -Logik
    durch eine Umstellung der anonymen ActionListener auf ein Action-basiertes
    System.


%\chapter{Referenzen und Standards}
